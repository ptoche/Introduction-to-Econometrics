% !TEX root = ../../MakeBeamer.tex
\title[Multiple Regression]{Multiple Regression: Multicollinearity}
\date{}


\begin{document}


\inputfile{../Section-Cover-Problems}


%%%%%%%%%%%%%%%%%%%%%%%%%%%%%%%%%%%%%%%%%%%%%%%%%%%%%%%%%
\begin{frame}
\frametitle{Problems and Applications}
\exercise{Stock \& Watson, Introduction (4th), Chapter~6, Exercise~4.}
Consider the regression of average hourly earnings $\vn{AHE}$ (in dollars) on $\vn{Age}$ (in years) and several binary variables for characteristics such as sex, education, and region of employment:
\begin{align*}
\verywidehat{\vn{AHE}} 
  = 0.33 & + 10.42\,\vn{College} - 4.57\,\vn{Female} + 0.61\,\vn{Age}\\ 
         & + 0.74\,\vn{Northeast} - 1.54\,\vn{Midwest} - 0.44\,\vn{South}\\
  \quad R^2 = 0.185, &
  \quad \vn{SER} = 12.01, 
  \quad n = 7178
\end{align*}
\vspace*{-2ex}
\begin{enumerate}
\item Do there appear to be important regional differences?
\item Why is the regressor $\vn{West}$ omitted from the regression? What would happen if it were included?
\item Juanita is a $28$-year-old female college graduate from the $\vn{South}$. Jennifer is a $28$-year-old female college graduate from the $\vn{Midwest}$. Calculate the expected difference in earnings between Juanita and Jennifer.
\end{enumerate}
\end{frame}
%%%%%%%%%%%%%%%%%%%%%%%%%%%%%%%%%%%%%%%%%%%%%%%%%%%%%%%%%


%%%%%%%%%%%%%%%%%%%%%%%%%%%%%%%%%%%%%%%%%%%%%%%%%%%%%%%%%
\begin{frame}
\frametitle{Problems and Applications}

\begin{enumerate}\setcounter{enumi}{0}

\item Do there appear to be important regional differences?

\begin{answer}
Since the variables for $West$ is omitted from the regression, it is the reference group to which the other regional variables can be compared to. 
On average, and controlling for other variables in the regression, workers in the $Northeast$ earn $\$0.74$ more per hour than workers in the $West$; while workers in the $Midwest$ earn $\$1.54$ less than workers in the West; and workers in the $South$ earn $\$0.44$ less than workers in the $West$.
\end{answer}

\end{enumerate}
\end{frame}
%%%%%%%%%%%%%%%%%%%%%%%%%%%%%%%%%%%%%%%%%%%%%%%%%%%%%%%%%


%%%%%%%%%%%%%%%%%%%%%%%%%%%%%%%%%%%%%%%%%%%%%%%%%%%%%%%%%
\begin{frame}
\frametitle{Problems and Applications}

\begin{enumerate}\setcounter{enumi}{1}

\item Why is the regressor $West$ omitted from the regression? What would happen if it were included?

\begin{answer}
The regressor $West$ is omitted to avoid perfect multicollinearity. Perfect multicollinearity would arise because the data is divided into exactly $4$ groups: $West$, $Midwest$, $Northeast$, and $South$. Since the $4$ categories are exchaustive and mutually exclusive, by construction, they add up to $1$ for every observation in the dataset. This is known as the ``dummy variable trap''. Perfect multicollinearity among regressors is usually easy to detect. Some software will produce an error, others will drop one of the perfectly multicollinear regressors and issue a warning. Imperfect multicollinearity is another issue, much less easy to deal with. 
\end{answer}

\end{enumerate}
\end{frame}
%%%%%%%%%%%%%%%%%%%%%%%%%%%%%%%%%%%%%%%%%%%%%%%%%%%%%%%%%


%%%%%%%%%%%%%%%%%%%%%%%%%%%%%%%%%%%%%%%%%%%%%%%%%%%%%%%%%
\begin{frame}
\frametitle{Problems and Applications}

\begin{enumerate}\setcounter{enumi}{2}

\item Juanita is a $28$-year-old female college graduate from the South. Jennifer is a $28$-year-old female college graduate from the Midwest. Calculate the expected difference in earnings between Juanita and Jennifer.

\begin{answer}
The expected difference in earnings between Juanita and Jennifer is:
\begin{align*}
AHE_{\text{Juanita}} & - AHE_{\text{Jennifer}} \\[1ex]
  = &~ (AHE | College=1, Female=1, Age=28, \\
    & \hspace*{4em} Northeast=0, Midwest=0, South=1)\\ 
  - &~ (AHE | College=1, Female=1, Age=28, \\
    & \hspace*{4em} Northeast=0, Midwest=1, South=0)\\ 
  = &~ (-0.44) - (-1.54) \\
  = &~ + 1.10
%  1.54-0.44
\end{align*}
The expected difference in earnings between Juanita and Jennifer, based on the information used in the regression, is $\$1.10$ per hour.
\end{answer}

\end{enumerate}
\end{frame}
%%%%%%%%%%%%%%%%%%%%%%%%%%%%%%%%%%%%%%%%%%%%%%%%%%%%%%%%%


\end{document}

