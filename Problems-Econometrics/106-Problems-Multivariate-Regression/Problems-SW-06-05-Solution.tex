% !TEX root = ../../MakeBeamer.tex
\title[Multiple Regression]{Multiple Regression: Home Sales}
\date{}


\begin{document}


\inputfile{../Section-Cover-Problems}


%%%%%%%%%%%%%%%%%%%%%%%%%%%%%%%%%%%%%%%%%%%%%%%%%%%%%%%%%
\begin{frame}[shrink=5]
\frametitle{Problems and Applications}
\exercise{Stock \& Watson, Introduction (4th), Chapter~6, Exercise~5.}
Data were collected from a random sample of $220$ home sales from a community in 2013. Let $\vn{Price}$ denote the selling price (in $\$1000$s), $\vn{BDR}$ denote the number of bedrooms, $\vn{Bath}$ denote the number of bathrooms, $\vn{Hsize}$ denote the size of the house (in square feet), $\vn{Lsize}$ denote the lot size (in square feet), $\vn{Age}$ denote the age of the house (in years), and $\vn{Poor}$ denote a binary variable that is equal to $1$ if the condition of the house is reported as ``poor.'' An estimated regression yields:
\begin{align*}
\verywidehat{\vn{Price}} 
  = & 119.2 + 0.485\,\vn{BDR} + 23.4\,\vn{Bath} + 0.156\,\vn{Hsize} \\
    & \quad + 0.002\,\vn{Lsize} + 0.090\,\vn{Age} - 48.8\,\vn{Poor}
  \quad \bar{R}^2 = 0.72,
  \quad \vn{SER} = 41.5
\end{align*}
\vspace*{-3ex}
\begin{enumerate}
\item Suppose a homeowner converts part of an existing family room in her house into a new bathroom. What is the expected increase in the value of the house?
\item Suppose a homeowner adds a new bathroom to her house, which increases the size of the house by $100$ square feet. What is the expected increase in the value of the house?
\item What is the loss in value if a homeowner lets his house run down, so that its condition becomes ``poor''?
\item Compute the $R^{2}$ for the regression.
\end{enumerate}
\end{frame}
%%%%%%%%%%%%%%%%%%%%%%%%%%%%%%%%%%%%%%%%%%%%%%%%%%%%%%%%%


%%%%%%%%%%%%%%%%%%%%%%%%%%%%%%%%%%%%%%%%%%%%%%%%%%%%%%%%%
\begin{frame}
\frametitle{Problems and Applications}

\begin{enumerate}\setcounter{enumi}{0}

\item Suppose a homeowner converts part of an existing family room in her house into a new bathroom. What is the expected increase in the value of the house?

\begin{answer}
The number of bathrooms $Bath$ increases by $1$. Since the other regressors are unchanged, only the slope coefficient on $Bath$ affects the expected increase in the house value:
\begin{align*}
\Delta\verywidehat{Price} 
  & = 23.4\,\Delta BATH\\
  & = 23.4 \cdot 1\\
  & = 23.4
\end{align*}
The expected increase in house value is $\$23,400$.
\end{answer}

\end{enumerate}
\end{frame}
%%%%%%%%%%%%%%%%%%%%%%%%%%%%%%%%%%%%%%%%%%%%%%%%%%%%%%%%%


%%%%%%%%%%%%%%%%%%%%%%%%%%%%%%%%%%%%%%%%%%%%%%%%%%%%%%%%%
\begin{frame}
\frametitle{Problems and Applications}

\begin{enumerate}\setcounter{enumi}{1}

\item Suppose a homeowner adds a new bathroom to her house, which increases the size of the house by $100$ square feet. What is the expected increase in the value of the house?

\begin{answer}
The number of bathrooms $Bath$ increases by $1$ and the size of the house $Hsize$ increases by $100$ square feet. The slope coefficients on $Bath$ and $Hsize$ both matter:
\begin{align*}
\Delta\verywidehat{Price} 
  & = 23.4\,\Delta BATH + 0.156\,\Delta Hsize\\
  & = 23.4 \cdot 1 + 0.156 \cdot 100\\
  & = 39.0
\end{align*}
The expected increase in house value is $\$39,000$.
\end{answer}

\end{enumerate}
\end{frame}
%%%%%%%%%%%%%%%%%%%%%%%%%%%%%%%%%%%%%%%%%%%%%%%%%%%%%%%%%


%%%%%%%%%%%%%%%%%%%%%%%%%%%%%%%%%%%%%%%%%%%%%%%%%%%%%%%%%
\begin{frame}
\frametitle{Problems and Applications}

\begin{enumerate}\setcounter{enumi}{2}

\item What is the loss in value if a homeowner lets his house run down, so that its condition becomes ``poor''?

\begin{answer}
The categorical variable $Poor$ changes from $0$ to $1$. Other regressors are unchanged. 
\begin{align*}
\Delta\verywidehat{Price} 
  & = - 48.8\,\Delta Poor\\
  & = - 48.8 \cdot 1\\
  & = - 48.8
\end{align*}
The expected loss in house value is $\$48,800$.
\end{answer}

\end{enumerate}
\end{frame}
%%%%%%%%%%%%%%%%%%%%%%%%%%%%%%%%%%%%%%%%%%%%%%%%%%%%%%%%%


%%%%%%%%%%%%%%%%%%%%%%%%%%%%%%%%%%%%%%%%%%%%%%%%%%%%%%%%%
\begin{frame}
\frametitle{Problems and Applications}

\begin{enumerate}\setcounter{enumi}{3}

\item Compute the $R^{2}$ for the regression.

\begin{answer}
The adjusted $\bar{R}^{2}$ is given, $\bar{R}^{2}=0.72$. The formula for the adjusted $\bar{R}^{2}$ in terms of the raw $R^{2}$ may be inverted to solve for $R^{2}$. The sample size is $n=220$ and the number of regressors is $p=6$.
\begin{align*}
\bar{R}^2 & = 1 - \frac{n-1}{n-1-p}\,(1-R^{2})\\
\implies 
R^2 & = 1 - \frac{n-1-p}{n-1}\,(1-\bar{R}^{2})\\
    & = 1 - \frac{220-1-6}{220-1}\,(1-0.72)\\
    & \approx 0.72767 
% 1-(220-1-6)/(220-1)*(1-0.72)
% 0.7276712
\end{align*}
The raw $R^{2}$ is about $0.73$, not very different from the adjusted $\bar{R}^{2}$. 
\end{answer}

\end{enumerate}
\end{frame}
%%%%%%%%%%%%%%%%%%%%%%%%%%%%%%%%%%%%%%%%%%%%%%%%%%%%%%%%%


\end{document}

