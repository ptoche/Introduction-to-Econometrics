% !TEX root = ../../MakeBeamer.tex
\title[Multivariate Assessment]{Multivariate Assessment: Earnings and Children}
\date{}


\begin{document}


\inputfile{../Section-Cover-Problems}


%%%%%%%%%%%%%%%%%%%%%%%%%%%%%%%%%%%%%%%%%%%%%%%%%%%%%%%%%
\begin{frame}
\frametitle{Problems and Applications}
\exercise{Stock \& Watson, Introduction (4th), Chapter~9, Exercise~3.}
Labor economists studying the determinants of women's earnings discovered a puzzling empirical result. Using randomly selected employed women, they regressed earnings on the women's number of children and a set of control variables (age, education, occupation, and so forth). They found that women with more children had higher wages, controlling for these other factors. Explain how sample selection might be the cause of this result. 
% Hint: Notice that women who do not work outside the home are missing from the sample.
\end{frame}
%%%%%%%%%%%%%%%%%%%%%%%%%%%%%%%%%%%%%%%%%%%%%%%%%%%%%%%%%


%%%%%%%%%%%%%%%%%%%%%%%%%%%%%%%%%%%%%%%%%%%%%%%%%%%%%%%%%
\begin{frame}
\frametitle{Problems and Applications}
They found that women with more children had higher wages, controlling for these other factors. Explain how sample selection might be the cause of this result. \begin{answer}
The sample selection bias is obvious: Only employed women are surveyed. Are employed women's attitude to child bearing likely to differ from the attitude of unemployed women? Two effects seem relevant. Women with higher wages are women less likely to become unemployed and more likely to find employment, which, other things equal, reduces labor market risk and the cost of child-bearing; on the other hand, women with higher wages have a greater opportunity cost of pregnancy, which would discourage child-bearing. While it is an empirical question that deserves to be tested directly, common sense suggests the latter effect is likely more important. The more children a woman has, the greater the wage offers must be to attract them away from their children, potentially explaining why the regression coefficient between the number of children and the woman's earnings is positive.
\end{answer}
\end{frame}
%%%%%%%%%%%%%%%%%%%%%%%%%%%%%%%%%%%%%%%%%%%%%%%%%%%%%%%%%


\end{document}
