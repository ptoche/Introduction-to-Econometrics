% Options for packages loaded elsewhere
\PassOptionsToPackage{unicode}{hyperref}
\PassOptionsToPackage{hyphens}{url}
%
\documentclass[
  11pt,
  ignorenonframetext,
  svgnames, handout, t]{beamer}
\usepackage{pgfpages}
\setbeamertemplate{caption}[numbered]
\setbeamertemplate{caption label separator}{: }
\setbeamercolor{caption name}{fg=normal text.fg}
\beamertemplatenavigationsymbolsempty
% Prevent slide breaks in the middle of a paragraph
\widowpenalties 1 10000
\raggedbottom
\setbeamertemplate{part page}{
  \centering
  \begin{beamercolorbox}[sep=16pt,center]{part title}
    \usebeamerfont{part title}\insertpart\par
  \end{beamercolorbox}
}
\setbeamertemplate{section page}{
  \centering
  \begin{beamercolorbox}[sep=12pt,center]{part title}
    \usebeamerfont{section title}\insertsection\par
  \end{beamercolorbox}
}
\setbeamertemplate{subsection page}{
  \centering
  \begin{beamercolorbox}[sep=8pt,center]{part title}
    \usebeamerfont{subsection title}\insertsubsection\par
  \end{beamercolorbox}
}
\AtBeginPart{
  \frame{\partpage}
}
\AtBeginSection{
  \ifbibliography
  \else
    \frame{\sectionpage}
  \fi
}
\AtBeginSubsection{
  \frame{\subsectionpage}
}
\usepackage{amsmath,amssymb}
\usepackage{lmodern}
\usepackage{iftex}
\ifPDFTeX
  \usepackage[T1]{fontenc}
  \usepackage[utf8]{inputenc}
  \usepackage{textcomp} % provide euro and other symbols
\else % if luatex or xetex
  \usepackage{unicode-math}
  \defaultfontfeatures{Scale=MatchLowercase}
  \defaultfontfeatures[\rmfamily]{Ligatures=TeX,Scale=1}
\fi
\usefonttheme{professionalfonts}
% Use upquote if available, for straight quotes in verbatim environments
\IfFileExists{upquote.sty}{\usepackage{upquote}}{}
\IfFileExists{microtype.sty}{% use microtype if available
  \usepackage[]{microtype}
  \UseMicrotypeSet[protrusion]{basicmath} % disable protrusion for tt fonts
}{}
\makeatletter
\@ifundefined{KOMAClassName}{% if non-KOMA class
  \IfFileExists{parskip.sty}{%
    \usepackage{parskip}
  }{% else
    \setlength{\parindent}{0pt}
    \setlength{\parskip}{6pt plus 2pt minus 1pt}}
}{% if KOMA class
  \KOMAoptions{parskip=half}}
\makeatother
\usepackage{xcolor}
\IfFileExists{xurl.sty}{\usepackage{xurl}}{} % add URL line breaks if available
\IfFileExists{bookmark.sty}{\usepackage{bookmark}}{\usepackage{hyperref}}
\hypersetup{
  pdftitle={Getting Started with R},
  pdfauthor={Patrick Toche },
  hidelinks,
  pdfcreator={LaTeX via pandoc}}
\urlstyle{same} % disable monospaced font for URLs
\newif\ifbibliography
\usepackage{color}
\usepackage{fancyvrb}
\newcommand{\VerbBar}{|}
\newcommand{\VERB}{\Verb[commandchars=\\\{\}]}
\DefineVerbatimEnvironment{Highlighting}{Verbatim}{commandchars=\\\{\}}
% Add ',fontsize=\small' for more characters per line
\usepackage{framed}
\definecolor{shadecolor}{RGB}{248,248,248}
\newenvironment{Shaded}{\begin{snugshade}}{\end{snugshade}}
\newcommand{\AlertTok}[1]{\textcolor[rgb]{0.94,0.16,0.16}{#1}}
\newcommand{\AnnotationTok}[1]{\textcolor[rgb]{0.56,0.35,0.01}{\textbf{\textit{#1}}}}
\newcommand{\AttributeTok}[1]{\textcolor[rgb]{0.77,0.63,0.00}{#1}}
\newcommand{\BaseNTok}[1]{\textcolor[rgb]{0.00,0.00,0.81}{#1}}
\newcommand{\BuiltInTok}[1]{#1}
\newcommand{\CharTok}[1]{\textcolor[rgb]{0.31,0.60,0.02}{#1}}
\newcommand{\CommentTok}[1]{\textcolor[rgb]{0.56,0.35,0.01}{\textit{#1}}}
\newcommand{\CommentVarTok}[1]{\textcolor[rgb]{0.56,0.35,0.01}{\textbf{\textit{#1}}}}
\newcommand{\ConstantTok}[1]{\textcolor[rgb]{0.00,0.00,0.00}{#1}}
\newcommand{\ControlFlowTok}[1]{\textcolor[rgb]{0.13,0.29,0.53}{\textbf{#1}}}
\newcommand{\DataTypeTok}[1]{\textcolor[rgb]{0.13,0.29,0.53}{#1}}
\newcommand{\DecValTok}[1]{\textcolor[rgb]{0.00,0.00,0.81}{#1}}
\newcommand{\DocumentationTok}[1]{\textcolor[rgb]{0.56,0.35,0.01}{\textbf{\textit{#1}}}}
\newcommand{\ErrorTok}[1]{\textcolor[rgb]{0.64,0.00,0.00}{\textbf{#1}}}
\newcommand{\ExtensionTok}[1]{#1}
\newcommand{\FloatTok}[1]{\textcolor[rgb]{0.00,0.00,0.81}{#1}}
\newcommand{\FunctionTok}[1]{\textcolor[rgb]{0.00,0.00,0.00}{#1}}
\newcommand{\ImportTok}[1]{#1}
\newcommand{\InformationTok}[1]{\textcolor[rgb]{0.56,0.35,0.01}{\textbf{\textit{#1}}}}
\newcommand{\KeywordTok}[1]{\textcolor[rgb]{0.13,0.29,0.53}{\textbf{#1}}}
\newcommand{\NormalTok}[1]{#1}
\newcommand{\OperatorTok}[1]{\textcolor[rgb]{0.81,0.36,0.00}{\textbf{#1}}}
\newcommand{\OtherTok}[1]{\textcolor[rgb]{0.56,0.35,0.01}{#1}}
\newcommand{\PreprocessorTok}[1]{\textcolor[rgb]{0.56,0.35,0.01}{\textit{#1}}}
\newcommand{\RegionMarkerTok}[1]{#1}
\newcommand{\SpecialCharTok}[1]{\textcolor[rgb]{0.00,0.00,0.00}{#1}}
\newcommand{\SpecialStringTok}[1]{\textcolor[rgb]{0.31,0.60,0.02}{#1}}
\newcommand{\StringTok}[1]{\textcolor[rgb]{0.31,0.60,0.02}{#1}}
\newcommand{\VariableTok}[1]{\textcolor[rgb]{0.00,0.00,0.00}{#1}}
\newcommand{\VerbatimStringTok}[1]{\textcolor[rgb]{0.31,0.60,0.02}{#1}}
\newcommand{\WarningTok}[1]{\textcolor[rgb]{0.56,0.35,0.01}{\textbf{\textit{#1}}}}
\setlength{\emergencystretch}{3em} % prevent overfull lines
\providecommand{\tightlist}{%
  \setlength{\itemsep}{0pt}\setlength{\parskip}{0pt}}
\setcounter{secnumdepth}{-\maxdimen} % remove section numbering
\geometry{paperwidth=160mm,paperheight=106mm}
\usepackage{lmodern}% bug: converts \pounds to dollar sign
\usepackage[english]{babel}
\usepackage[T1]{fontenc}% font encoding, load before inputenc
\usepackage{fontspec}% font encoding
\frenchspacing
\defaultfontfeatures{
   Mapping=tex-text,
   Scale=MatchLowercase,
}
%   \setsansfont{Montserrat}
\setsansfont{Lato}
\setmonofont{Droid Sans Mono}
\newfontfamily\fontbold{Lato Bold}
\newfontfamily\fontitalic{Lato Italic}
\newfontfamily\fontbolditalic{Lato Bold Italic}
\newfontfamily\quotefont{Minotaur}

\usepackage{amsmath}
\usepackage{graphicx}
\usepackage{caption}
\usepackage{hyperref}

\colorlet{themetext}{black}
\colorlet{themefill}{RoyalBlue!50}
\colorlet{themecolor}{NavyBlue}
\usetheme{default}
\setbeamertemplate{navigation symbols}{}
\setbeamercovered{transparent}
\setbeamertemplate{title page}[empty]
\usecolortheme{seahorse}
\setbeamercovered{transparent=4}
\setbeamercolor{palette primary}{use=structure,fg=black,bg=themefill}
\setbeamercolor{title}{fg=black}
\setbeamercolor{frametitle}{fg=black}
\setbeamercolor{itemize item}{fg=themecolor}
\setbeamercolor{enumerate item}{fg=themecolor}
\setbeamercolor{itemize subitem}{fg=themecolor}
\setbeamercolor{enumerate subitem}{fg=themecolor}
%\setbeamertemplate{itemize item}[triangle]% default, triangle, circle, square, ball
\setbeamertemplate{itemize item}{\raisebox{-0.7ex}{\scalebox{1}[2.5]{\bfseries\textendash}}} 
\setbeamertemplate{itemize subitem}{\raisebox{-0.7ex}{\scalebox{1}[2.5]{\bfseries\textendash}}} 
\setbeamertemplate{enumerate subitem}{\alph{enumii}.}
\ifLuaTeX
  \usepackage{selnolig}  % disable illegal ligatures
\fi

\title{Getting Started with R}
\subtitle{Econ 440 - Introduction to Econometrics}
\author{Patrick Toche}
\date{01 March 2022}

\begin{document}
\frame{\titlepage}

\begin{frame}{Contents}
\protect\hypertarget{contents}{}
\begin{enumerate}
\tightlist
\item
  Know your data types and data containers
\item
  Convert strings and factors to numeric values
\item
  Select/Create data inside a dataframe
\item
  Import/Export data
\item
  Quick sums
\item
  Split/Apply
\item
  Create functions
\end{enumerate}
\end{frame}

\begin{frame}{Introduction}
\protect\hypertarget{introduction}{}
This document is work in progress. It will be updated as we proceed
through the course, in response to questions that arise for particular
tasks. It is designed to be permanently incomplete and permanently
revised. Also, please use with care: Always back up your data!
\end{frame}

\begin{frame}[fragile]{Know your data types}
\protect\hypertarget{know-your-data-types}{}
We are going to be dealing with the following data types: numbers,
strings, booleans, factors, and containers filled with these, including:
vectors/lists, named lists, matrices, dataframes, tibbles.

Start a new session, type the letter a into the console: you get an
error!

\footnotesize

\begin{Shaded}
\begin{Highlighting}[]
\NormalTok{a}
\end{Highlighting}
\end{Shaded}

\normalsize

Now type the letter c into the console: you get something!

\footnotesize

\begin{Shaded}
\begin{Highlighting}[]
\NormalTok{c}
\CommentTok{\#\textgreater{} function (...)  .Primitive("c")}
\end{Highlighting}
\end{Shaded}

\normalsize

That's because in \texttt{R} the letter \texttt{c} is the name of a
function \texttt{c()}. That's unfortunate, but \texttt{R} was designed
in the 1970s when very few people had a need of symbols beyond
\texttt{a} and \texttt{b}. Lol.
\end{frame}

\begin{frame}[fragile]{Know your data types}
\protect\hypertarget{know-your-data-types-1}{}
Other than \texttt{c()}, other names that are used by \texttt{R}, and
are therefore best avoided as names for other objects, include
\texttt{df} and \texttt{data}. But because they are names of functions,
these names are actually still available for other objects. For
instance,

\footnotesize

\begin{Shaded}
\begin{Highlighting}[]
\CommentTok{\# define c}
\NormalTok{c }\OtherTok{=} \DecValTok{0}

\CommentTok{\# check the value of c}
\NormalTok{c}
\CommentTok{\#\textgreater{} [1] 0}

\CommentTok{\# but c() is still available}
\FunctionTok{c}\NormalTok{(}\DecValTok{0}\NormalTok{,}\DecValTok{1}\NormalTok{)}
\CommentTok{\#\textgreater{} [1] 0 1}
\end{Highlighting}
\end{Shaded}

\normalsize It's not too smart, but we do it all the time. Better people
use \texttt{C} or \texttt{DF}, because \texttt{R} is case sensitive and
the upper-case symbols are available.
\end{frame}

\begin{frame}[fragile]{Know your data types}
\protect\hypertarget{know-your-data-types-2}{}
The following is definitely stupid:

\footnotesize

\begin{Shaded}
\begin{Highlighting}[]
\CommentTok{\# define c}
\NormalTok{c }\OtherTok{=} \ControlFlowTok{function}\NormalTok{(x) }\ConstantTok{NULL}

\CommentTok{\# check c(0,1)}
\FunctionTok{c}\NormalTok{()}
\CommentTok{\#\textgreater{} NULL}

\CommentTok{\# remove this before it ruins our life}
\FunctionTok{rm}\NormalTok{(c)}
\end{Highlighting}
\end{Shaded}

\normalsize
\end{frame}

\begin{frame}[fragile]{Know your data types}
\protect\hypertarget{know-your-data-types-3}{}
So let's investigate some data types:

\footnotesize

\begin{Shaded}
\begin{Highlighting}[]
\CommentTok{\# floating{-}point number aka num}
\NormalTok{a1 }\OtherTok{=} \DecValTok{0}
\FunctionTok{str}\NormalTok{(a1)}
\CommentTok{\#\textgreater{}  num 0}

\CommentTok{\# string aka character chr}
\NormalTok{a2 }\OtherTok{=} \StringTok{"0"}
\FunctionTok{str}\NormalTok{(a2)}
\CommentTok{\#\textgreater{}  chr "0"}

\CommentTok{\# boolean aka logical}
\NormalTok{a3 }\OtherTok{=} \ConstantTok{TRUE}
\FunctionTok{str}\NormalTok{(a3)}
\CommentTok{\#\textgreater{}  logi TRUE}
\end{Highlighting}
\end{Shaded}

\normalsize In \texttt{R} booleans are called \texttt{logical} vectors.
\end{frame}

\begin{frame}[fragile]{Know your data types}
\protect\hypertarget{know-your-data-types-4}{}
\footnotesize

\begin{Shaded}
\begin{Highlighting}[]
\CommentTok{\# integer aka int}
\NormalTok{a4 }\OtherTok{=}\NormalTok{ 0L}
\FunctionTok{str}\NormalTok{(a4)}
\CommentTok{\#\textgreater{}  int 0}

\CommentTok{\# factor}
\NormalTok{a5 }\OtherTok{=} \FunctionTok{factor}\NormalTok{(}\DecValTok{0}\NormalTok{, }\AttributeTok{level=}\DecValTok{1}\NormalTok{)}
\FunctionTok{str}\NormalTok{(a5)}
\CommentTok{\#\textgreater{}  Factor w/ 1 level "1": NA}

\CommentTok{\# vector of factors}
\NormalTok{a6 }\OtherTok{=} \FunctionTok{factor}\NormalTok{(}\FunctionTok{c}\NormalTok{(}\DecValTok{0}\NormalTok{,}\DecValTok{1}\NormalTok{), }\AttributeTok{levels=}\FunctionTok{c}\NormalTok{(}\DecValTok{1}\NormalTok{,}\DecValTok{2}\NormalTok{))}
\FunctionTok{str}\NormalTok{(a6)}
\CommentTok{\#\textgreater{}  Factor w/ 2 levels "1","2": NA 1}
\end{Highlighting}
\end{Shaded}

\normalsize
\end{frame}

\begin{frame}[fragile]{Know your data containers}
\protect\hypertarget{know-your-data-containers}{}
We will be dealing mostly with `vectors' and `dataframes' In \texttt{R}
lists are vectors. A vector is the most fundamental data container in
\texttt{R}. Even a single object is under the hood a vector with a
single value.

\footnotesize

\begin{Shaded}
\begin{Highlighting}[]
\CommentTok{\# a vector/list of numbers:}
\NormalTok{l1 }\OtherTok{=} \FunctionTok{c}\NormalTok{(}\DecValTok{0}\NormalTok{, }\DecValTok{1}\NormalTok{)}
\FunctionTok{str}\NormalTok{(l1)}
\CommentTok{\#\textgreater{}  num [1:2] 0 1}

\CommentTok{\# a vector/list of integers:}
\NormalTok{l2 }\OtherTok{=} \FunctionTok{c}\NormalTok{(}\DecValTok{0}\NormalTok{, 1L)}
\FunctionTok{str}\NormalTok{(l2)}
\CommentTok{\#\textgreater{}  num [1:2] 0 1}

\CommentTok{\# a vector/list of booleans:}
\NormalTok{l3 }\OtherTok{=} \FunctionTok{c}\NormalTok{(}\ConstantTok{FALSE}\NormalTok{, }\ConstantTok{TRUE}\NormalTok{)}
\FunctionTok{str}\NormalTok{(l3)}
\CommentTok{\#\textgreater{}  logi [1:2] FALSE TRUE}
\end{Highlighting}
\end{Shaded}

\normalsize
\end{frame}

\begin{frame}[fragile]{Know your data containers}
\protect\hypertarget{know-your-data-containers-1}{}
In \texttt{R} named lists have a purpose similar to `dictionaries' in
\texttt{Python}.

\footnotesize

\begin{Shaded}
\begin{Highlighting}[]
\CommentTok{\# a named list:}
\NormalTok{l4 }\OtherTok{=} \FunctionTok{c}\NormalTok{(}\StringTok{"name1"} \OtherTok{=} \DecValTok{1}\NormalTok{, }\StringTok{"name2"} \OtherTok{=} \ConstantTok{TRUE}\NormalTok{)}
\FunctionTok{str}\NormalTok{(l4)}
\CommentTok{\#\textgreater{}  Named num [1:2] 1 1}
\CommentTok{\#\textgreater{}  {-} attr(*, "names")= chr [1:2] "name1" "name2"}
\end{Highlighting}
\end{Shaded}

\normalsize

Every object stored in a named list has a name and a value:

\footnotesize

\begin{Shaded}
\begin{Highlighting}[]
\CommentTok{\# get the name and value with []}
\NormalTok{l4[}\StringTok{"name1"}\NormalTok{]}
\CommentTok{\#\textgreater{} name1 }
\CommentTok{\#\textgreater{}     1}

\CommentTok{\# get the value with [[]]}
\NormalTok{l4[[}\StringTok{"name1"}\NormalTok{]]}
\CommentTok{\#\textgreater{} [1] 1}
\end{Highlighting}
\end{Shaded}

\normalsize
\end{frame}

\begin{frame}[fragile]{Know your data containers}
\protect\hypertarget{know-your-data-containers-2}{}
The easiest way to create a dataframe is from several vectors.

\footnotesize

\begin{Shaded}
\begin{Highlighting}[]
\CommentTok{\# define band member, the date they joined/quit the band, and their net worth}
\CommentTok{\# data from a quick web search {-} not meant to be accurate!}
\NormalTok{member }\OtherTok{\textless{}{-}} \FunctionTok{c}\NormalTok{(}\StringTok{\textquotesingle{}Brian\textquotesingle{}}\NormalTok{, }\StringTok{\textquotesingle{}Mick\textquotesingle{}}\NormalTok{, }\StringTok{\textquotesingle{}Keith\textquotesingle{}}\NormalTok{, }\StringTok{\textquotesingle{}Stewart\textquotesingle{}}\NormalTok{, }\StringTok{\textquotesingle{}Bill\textquotesingle{}}\NormalTok{, }\StringTok{\textquotesingle{}Charlie\textquotesingle{}}\NormalTok{, }\StringTok{\textquotesingle{}Ronnie\textquotesingle{}}\NormalTok{)}
\NormalTok{start }\OtherTok{\textless{}{-}} \FunctionTok{as.Date}\NormalTok{(}\FunctionTok{c}\NormalTok{(}\StringTok{\textquotesingle{}1962{-}07{-}12\textquotesingle{}}\NormalTok{, }\StringTok{\textquotesingle{}1962{-}07{-}12\textquotesingle{}}\NormalTok{,}\StringTok{\textquotesingle{}1962{-}07{-}12\textquotesingle{}}\NormalTok{,}\StringTok{\textquotesingle{}1962{-}07{-}12\textquotesingle{}}\NormalTok{,}
                   \StringTok{\textquotesingle{}1962{-}12{-}07\textquotesingle{}}\NormalTok{,}\StringTok{\textquotesingle{}1963{-}02{-}02\textquotesingle{}}\NormalTok{,}\StringTok{\textquotesingle{}1976{-}04{-}23\textquotesingle{}}\NormalTok{))}
\NormalTok{today }\OtherTok{\textless{}{-}} \FunctionTok{as.character}\NormalTok{(}\FunctionTok{Sys.Date}\NormalTok{())}
\NormalTok{end }\OtherTok{\textless{}{-}} \FunctionTok{as.Date}\NormalTok{(}\FunctionTok{c}\NormalTok{(}\StringTok{\textquotesingle{}1969{-}06{-}09\textquotesingle{}}\NormalTok{, today, today, }
                 \StringTok{\textquotesingle{}1963{-}05{-}01\textquotesingle{}}\NormalTok{, }\StringTok{\textquotesingle{}1992{-}12{-}01\textquotesingle{}}\NormalTok{, }\StringTok{\textquotesingle{}2021{-}08{-}24\textquotesingle{}}\NormalTok{, today))}
\NormalTok{worth }\OtherTok{\textless{}{-}} \FunctionTok{c}\NormalTok{(}\DecValTok{10}\NormalTok{, }\DecValTok{500}\NormalTok{, }\DecValTok{500}\NormalTok{, }\DecValTok{1}\NormalTok{, }\DecValTok{80}\NormalTok{, }\DecValTok{250}\NormalTok{, }\DecValTok{200}\NormalTok{)}
\NormalTok{df }\OtherTok{\textless{}{-}}\NormalTok{ rolling.stones }\OtherTok{\textless{}{-}} \FunctionTok{data.frame}\NormalTok{(member, start, end, worth)}
\FunctionTok{str}\NormalTok{(rolling.stones)}
\CommentTok{\#\textgreater{} \textquotesingle{}data.frame\textquotesingle{}:    7 obs. of  4 variables:}
\CommentTok{\#\textgreater{}  $ member: chr  "Brian" "Mick" "Keith" "Stewart" ...}
\CommentTok{\#\textgreater{}  $ start : Date, format: "1962{-}07{-}12" "1962{-}07{-}12" ...}
\CommentTok{\#\textgreater{}  $ end   : Date, format: "1969{-}06{-}09" "2022{-}02{-}20" ...}
\CommentTok{\#\textgreater{}  $ worth : num  10 500 500 1 80 250 200}

\CommentTok{\# add information about the units:}
\NormalTok{rolling.stones}\SpecialCharTok{$}\NormalTok{currency }\OtherTok{\textless{}{-}} \StringTok{"$"}  \CommentTok{\# R is clever that way}
\end{Highlighting}
\end{Shaded}

\normalsize
\end{frame}

\begin{frame}[fragile]{Know your data containers}
\protect\hypertarget{know-your-data-containers-3}{}
We can output the core structure of a dataframe with the \texttt{dput()}
function.

\footnotesize

\begin{Shaded}
\begin{Highlighting}[]
\FunctionTok{dput}\NormalTok{(df)}
\CommentTok{\#\textgreater{} structure(list(member = c("Brian", "Mick", "Keith", "Stewart", }
\CommentTok{\#\textgreater{} "Bill", "Charlie", "Ronnie"), start = structure(c({-}2730, {-}2730, }
\CommentTok{\#\textgreater{} {-}2730, {-}2730, {-}2582, {-}2525, 2304), class = "Date"), end = structure(c({-}206, }
\CommentTok{\#\textgreater{} 19043, 19043, {-}2437, 8370, 18863, 19043), class = "Date"), worth = c(10, }
\CommentTok{\#\textgreater{} 500, 500, 1, 80, 250, 200)), class = "data.frame", row.names = c(NA, }
\CommentTok{\#\textgreater{} {-}7L))}
\end{Highlighting}
\end{Shaded}

\normalsize This is useful to share the dataframe without creating a
\texttt{csv} file or similar. It is the best way to share toy data on
websites like \href{https://stackoverflow.com/}{stackoverflow}.

You can convert the dataframe to a \texttt{tibble} using the
\texttt{as\_tibble()} function:

\footnotesize

\begin{Shaded}
\begin{Highlighting}[]
\FunctionTok{dput}\NormalTok{(}\FunctionTok{as\_tibble}\NormalTok{(df))}
\CommentTok{\#\textgreater{} structure(list(member = c("Brian", "Mick", "Keith", "Stewart", }
\CommentTok{\#\textgreater{} "Bill", "Charlie", "Ronnie"), start = structure(c({-}2730, {-}2730, }
\CommentTok{\#\textgreater{} {-}2730, {-}2730, {-}2582, {-}2525, 2304), class = "Date"), end = structure(c({-}206, }
\CommentTok{\#\textgreater{} 19043, 19043, {-}2437, 8370, 18863, 19043), class = "Date"), worth = c(10, }
\CommentTok{\#\textgreater{} 500, 500, 1, 80, 250, 200)), class = c("tbl\_df", "tbl", "data.frame"}
\CommentTok{\#\textgreater{} ), row.names = c(NA, {-}7L))}
\end{Highlighting}
\end{Shaded}

\normalsize which shows that the \texttt{tibble} does not contain more
information than the \texttt{dataframe}.
\end{frame}

\begin{frame}[fragile]{Convert strings and factors to numeric values}
\protect\hypertarget{convert-strings-and-factors-to-numeric-values}{}
Conversion across types must be done with care, obviously.

Functions such as \texttt{as.numeric()}, \texttt{as.character()} are
`vectorized', which means they can be applied to a vector/list and also
to a single column in a dataframe.

\footnotesize

\begin{Shaded}
\begin{Highlighting}[]
\CommentTok{\# convert a number ot a string:}
\FunctionTok{as.character}\NormalTok{(}\DecValTok{0}\NormalTok{)}
\CommentTok{\#\textgreater{} [1] "0"}

\CommentTok{\# convert a string to a number:}
\FunctionTok{as.numeric}\NormalTok{(}\StringTok{"0"}\NormalTok{)}
\CommentTok{\#\textgreater{} [1] 0}

\CommentTok{\# convert a string to an integer:}
\FunctionTok{as.integer}\NormalTok{(}\StringTok{"0"}\NormalTok{)}
\CommentTok{\#\textgreater{} [1] 0}
\FunctionTok{as.integer}\NormalTok{(}\StringTok{"0.5"}\NormalTok{)}
\CommentTok{\#\textgreater{} [1] 0}

\CommentTok{\# convert a vector of booleans:}
\FunctionTok{as.numeric}\NormalTok{(}\FunctionTok{c}\NormalTok{(}\ConstantTok{TRUE}\NormalTok{, }\ConstantTok{FALSE}\NormalTok{))}
\CommentTok{\#\textgreater{} [1] 1 0}
\end{Highlighting}
\end{Shaded}

\normalsize
\end{frame}

\begin{frame}[fragile]{Convert strings and factors to numeric values}
\protect\hypertarget{convert-strings-and-factors-to-numeric-values-1}{}
\footnotesize

\begin{Shaded}
\begin{Highlighting}[]
\CommentTok{\# convert a factor to number:}
\NormalTok{f }\OtherTok{=} \FunctionTok{factor}\NormalTok{(}\FunctionTok{c}\NormalTok{(}\DecValTok{0}\NormalTok{,}\DecValTok{1}\NormalTok{), }\AttributeTok{levels=}\FunctionTok{c}\NormalTok{(}\DecValTok{1}\NormalTok{,}\DecValTok{2}\NormalTok{))}
\FunctionTok{as.character}\NormalTok{(f)}
\CommentTok{\#\textgreater{} [1] NA  "1"}
\FunctionTok{as.numeric}\NormalTok{(}\FunctionTok{as.character}\NormalTok{(f))}
\CommentTok{\#\textgreater{} [1] NA  1}

\CommentTok{\# convert a column of a dataframe and reassign back to dataframe}
\NormalTok{df}\SpecialCharTok{$}\NormalTok{worth }\OtherTok{\textless{}{-}} \FunctionTok{as.integer}\NormalTok{(df}\SpecialCharTok{$}\NormalTok{worth)}
\FunctionTok{str}\NormalTok{(df)}
\CommentTok{\#\textgreater{} \textquotesingle{}data.frame\textquotesingle{}:    7 obs. of  4 variables:}
\CommentTok{\#\textgreater{}  $ member: chr  "Brian" "Mick" "Keith" "Stewart" ...}
\CommentTok{\#\textgreater{}  $ start : Date, format: "1962{-}07{-}12" "1962{-}07{-}12" ...}
\CommentTok{\#\textgreater{}  $ end   : Date, format: "1969{-}06{-}09" "2022{-}02{-}20" ...}
\CommentTok{\#\textgreater{}  $ worth : int  10 500 500 1 80 250 200}
\end{Highlighting}
\end{Shaded}

\normalsize
\end{frame}

\begin{frame}[fragile]{Select data inside a dataframe}
\protect\hypertarget{select-data-inside-a-dataframe}{}
\footnotesize

\begin{Shaded}
\begin{Highlighting}[]
\CommentTok{\# get a column}
\NormalTok{df}\SpecialCharTok{$}\NormalTok{member}
\CommentTok{\#\textgreater{} [1] "Brian"   "Mick"    "Keith"   "Stewart" "Bill"    "Charlie" "Ronnie"}

\CommentTok{\# get one row}
\NormalTok{df[}\DecValTok{2}\NormalTok{,]}
\CommentTok{\#\textgreater{}   member      start        end worth}
\CommentTok{\#\textgreater{} 2   Mick 1962{-}07{-}12 2022{-}02{-}20   500}

\CommentTok{\# get several rows}
\NormalTok{df[}\DecValTok{2}\SpecialCharTok{:}\DecValTok{3}\NormalTok{,]}
\CommentTok{\#\textgreater{}   member      start        end worth}
\CommentTok{\#\textgreater{} 2   Mick 1962{-}07{-}12 2022{-}02{-}20   500}
\CommentTok{\#\textgreater{} 3  Keith 1962{-}07{-}12 2022{-}02{-}20   500}

\CommentTok{\# get a row by criterion}
\NormalTok{df[df}\SpecialCharTok{$}\NormalTok{worth }\SpecialCharTok{\textgreater{}=} \DecValTok{300}\NormalTok{,]}
\CommentTok{\#\textgreater{}   member      start        end worth}
\CommentTok{\#\textgreater{} 2   Mick 1962{-}07{-}12 2022{-}02{-}20   500}
\CommentTok{\#\textgreater{} 3  Keith 1962{-}07{-}12 2022{-}02{-}20   500}
\end{Highlighting}
\end{Shaded}

\normalsize
\end{frame}

\begin{frame}[fragile]{Add data to a dataframe}
\protect\hypertarget{add-data-to-a-dataframe}{}
We can create categorical variables as follows:

\footnotesize

\begin{Shaded}
\begin{Highlighting}[]
\NormalTok{df}\SpecialCharTok{$}\NormalTok{poor }\OtherTok{\textless{}{-}}\NormalTok{ df}\SpecialCharTok{$}\NormalTok{worth }\SpecialCharTok{\textless{}} \DecValTok{100}
\NormalTok{df}\SpecialCharTok{$}\NormalTok{middle }\OtherTok{\textless{}{-}}\NormalTok{ df}\SpecialCharTok{$}\NormalTok{worth }\SpecialCharTok{\textgreater{}=} \DecValTok{100} \SpecialCharTok{\&}\NormalTok{ df}\SpecialCharTok{$}\NormalTok{worth }\SpecialCharTok{\textless{}} \DecValTok{300}
\NormalTok{df}\SpecialCharTok{$}\NormalTok{rich }\OtherTok{\textless{}{-}}\NormalTok{ df}\SpecialCharTok{$}\NormalTok{worth }\SpecialCharTok{\textgreater{}=} \DecValTok{300}
\end{Highlighting}
\end{Shaded}

\normalsize

We can create a duration variable as follows:

\footnotesize

\begin{Shaded}
\begin{Highlighting}[]
\NormalTok{df}\SpecialCharTok{$}\NormalTok{duration }\OtherTok{\textless{}{-}} \FunctionTok{round}\NormalTok{(}\DecValTok{100}\SpecialCharTok{*}\FunctionTok{as.integer}\NormalTok{((df}\SpecialCharTok{$}\NormalTok{end}\SpecialCharTok{{-}}\NormalTok{df}\SpecialCharTok{$}\NormalTok{start))}\SpecialCharTok{/}\FunctionTok{as.integer}\NormalTok{(}\FunctionTok{max}\NormalTok{(df}\SpecialCharTok{$}\NormalTok{end}\SpecialCharTok{{-}}\NormalTok{df}\SpecialCharTok{$}\NormalTok{start)))}
\end{Highlighting}
\end{Shaded}

\normalsize This calculates the interval between the start and end
dates, calculates the maximum time interval in the datasets, and for
each interval calculates the percentage it represents.

\footnotesize

\begin{Shaded}
\begin{Highlighting}[]
\NormalTok{df}
\CommentTok{\#\textgreater{}    member      start        end worth  poor middle  rich duration}
\CommentTok{\#\textgreater{} 1   Brian 1962{-}07{-}12 1969{-}06{-}09    10  TRUE  FALSE FALSE       12}
\CommentTok{\#\textgreater{} 2    Mick 1962{-}07{-}12 2022{-}02{-}20   500 FALSE  FALSE  TRUE      100}
\CommentTok{\#\textgreater{} 3   Keith 1962{-}07{-}12 2022{-}02{-}20   500 FALSE  FALSE  TRUE      100}
\CommentTok{\#\textgreater{} 4 Stewart 1962{-}07{-}12 1963{-}05{-}01     1  TRUE  FALSE FALSE        1}
\CommentTok{\#\textgreater{} 5    Bill 1962{-}12{-}07 1992{-}12{-}01    80  TRUE  FALSE FALSE       50}
\CommentTok{\#\textgreater{} 6 Charlie 1963{-}02{-}02 2021{-}08{-}24   250 FALSE   TRUE FALSE       98}
\CommentTok{\#\textgreater{} 7  Ronnie 1976{-}04{-}23 2022{-}02{-}20   200 FALSE   TRUE FALSE       77}
\end{Highlighting}
\end{Shaded}

\normalsize
\end{frame}

\begin{frame}[fragile]{Use data in a dataframe}
\protect\hypertarget{use-data-in-a-dataframe}{}
The correlation between net worth and the duration variable is positive,
as expected:

\footnotesize

\begin{Shaded}
\begin{Highlighting}[]
\FunctionTok{cor}\NormalTok{(df}\SpecialCharTok{$}\NormalTok{duration, df}\SpecialCharTok{$}\NormalTok{worth)}
\CommentTok{\#\textgreater{} [1] 0.883}
\end{Highlighting}
\end{Shaded}

\normalsize

And significant:

\footnotesize

\begin{Shaded}
\begin{Highlighting}[]
\FunctionTok{cor.test}\NormalTok{(df}\SpecialCharTok{$}\NormalTok{duration, df}\SpecialCharTok{$}\NormalTok{worth)}
\CommentTok{\#\textgreater{} }
\CommentTok{\#\textgreater{}  Pearson\textquotesingle{}s product{-}moment correlation}
\CommentTok{\#\textgreater{} }
\CommentTok{\#\textgreater{} data:  df$duration and df$worth}
\CommentTok{\#\textgreater{} t = 4, df = 5, p{-}value = 0.008}
\CommentTok{\#\textgreater{} alternative hypothesis: true correlation is not equal to 0}
\CommentTok{\#\textgreater{} 95 percent confidence interval:}
\CommentTok{\#\textgreater{}  0.388 0.983}
\CommentTok{\#\textgreater{} sample estimates:}
\CommentTok{\#\textgreater{}   cor }
\CommentTok{\#\textgreater{} 0.883}
\end{Highlighting}
\end{Shaded}

\normalsize
\end{frame}

\begin{frame}[fragile]{Import data}
\protect\hypertarget{import-data}{}
The most common data format is \texttt{csv} --- comma separated values.

The most commonly needed options are \texttt{stringsAsFactors=FALSE} to
prevent \texttt{R} from converting strings to factors and the option to
use or discard header information, with \texttt{header=FALSE} or
\texttt{header=TRUE}:

\footnotesize

\begin{Shaded}
\begin{Highlighting}[]
\CommentTok{\# basic import:}
\NormalTok{d }\OtherTok{\textless{}{-}} \FunctionTok{read.csv}\NormalTok{(}\StringTok{"Age\_HourlyEarnings.csv"}\NormalTok{, }\AttributeTok{stringsAsFactors=}\ConstantTok{FALSE}\NormalTok{)}

\CommentTok{\# import with more control:}
\NormalTok{d }\OtherTok{\textless{}{-}} \FunctionTok{read.csv}\NormalTok{(}\StringTok{"Age\_HourlyEarnings.csv"}\NormalTok{, }\AttributeTok{stringsAsFactors=}\ConstantTok{FALSE}\NormalTok{, }
              \AttributeTok{skip=}\DecValTok{1}\NormalTok{, }\AttributeTok{na.strings=}\StringTok{"NA"}\NormalTok{, }\AttributeTok{header=}\ConstantTok{TRUE}\NormalTok{, }\AttributeTok{strip.white=}\ConstantTok{TRUE}\NormalTok{)}
\end{Highlighting}
\end{Shaded}

\normalsize

There are many ways to import excel data. One option is to use the
\texttt{read\_xlsx()} function from the \texttt{readxl} package.

\footnotesize

\begin{Shaded}
\begin{Highlighting}[]
\FunctionTok{library}\NormalTok{(readxl)}
\NormalTok{d }\OtherTok{\textless{}{-}} \FunctionTok{read\_xlsx}\NormalTok{(}\StringTok{"Age\_HourlyEarnings.xlsx"}\NormalTok{, }\AttributeTok{col\_names=}\ConstantTok{TRUE}\NormalTok{, }\AttributeTok{skip=}\DecValTok{1}\NormalTok{, }\AttributeTok{trim\_ws=}\ConstantTok{TRUE}\NormalTok{)}
\end{Highlighting}
\end{Shaded}

\normalsize
\end{frame}

\begin{frame}[fragile]{Export data}
\protect\hypertarget{export-data}{}
To save a dataframe as a \texttt{csv} file:

\footnotesize

\begin{Shaded}
\begin{Highlighting}[]
\FunctionTok{write.csv}\NormalTok{(df, }\AttributeTok{file =} \StringTok{"filename.csv"}\NormalTok{, }\AttributeTok{row.names=}\ConstantTok{FALSE}\NormalTok{)}
\end{Highlighting}
\end{Shaded}

\normalsize where you typically do not want to save the row names.

To save a dataframe as a \texttt{xlsx} file, among many options, you can
use the \texttt{write\_xlsx()} function from the \texttt{writexl}
package:

\footnotesize

\begin{Shaded}
\begin{Highlighting}[]
\FunctionTok{library}\NormalTok{(writexl)}
\FunctionTok{write\_xlsx}\NormalTok{(df, }\AttributeTok{file =} \StringTok{"filename.csv"}\NormalTok{, }\AttributeTok{row.names=}\ConstantTok{FALSE}\NormalTok{)}
\end{Highlighting}
\end{Shaded}

\normalsize

You can also save the data in the native \texttt{RData} format. This
format will preserve information about data types, information that will
otherwise be lost with the \texttt{csv} and \texttt{xlsx} formats. The
only downside of the \texttt{RData} format is that it cannot be imported
by spreadsheets and may not be easily read by other software, whereas
the \texttt{csv} format is universal and the \texttt{xlsx} format nearly
universal.

\footnotesize

\begin{Shaded}
\begin{Highlighting}[]
\FunctionTok{save}\NormalTok{(df, }\AttributeTok{file =} \StringTok{"filename.RData"}\NormalTok{)}
\end{Highlighting}
\end{Shaded}

\normalsize
\end{frame}

\begin{frame}[fragile]{Export data}
\protect\hypertarget{export-data-1}{}
It is possible to import data from other formats, including
\texttt{Stata}, \texttt{SPSS}, \texttt{SAS}. One option, among several,
comes from the \texttt{Hmisc} and \texttt{foreign} packages. Examples:

\footnotesize

\begin{Shaded}
\begin{Highlighting}[]
\FunctionTok{library}\NormalTok{(foreign)}
\NormalTok{df.stata }\OtherTok{\textless{}{-}} \FunctionTok{read.dta}\NormalTok{(}\StringTok{"stata{-}format.dta"}\NormalTok{)}

\FunctionTok{library}\NormalTok{(Hmisc)}
\NormalTok{df.spss }\OtherTok{\textless{}{-}} \FunctionTok{spss.get}\NormalTok{(}\StringTok{"spss{-}format.por"}\NormalTok{, }\AttributeTok{use.value.labels=}\ConstantTok{TRUE}\NormalTok{)}

\FunctionTok{library}\NormalTok{(Hmisc)}
\NormalTok{df.sas }\OtherTok{\textless{}{-}} \FunctionTok{sasxport.get}\NormalTok{(}\StringTok{"sas{-}format.xpt"}\NormalTok{)}
\end{Highlighting}
\end{Shaded}

\normalsize
\end{frame}

\begin{frame}[fragile]{Quick sums}
\protect\hypertarget{quick-sums}{}
\texttt{R} knows how to sum the elements of a vector:

\footnotesize

\begin{Shaded}
\begin{Highlighting}[]
\NormalTok{x }\OtherTok{=} \FunctionTok{c}\NormalTok{(}\DecValTok{1}\NormalTok{,}\DecValTok{2}\NormalTok{,}\DecValTok{3}\NormalTok{,}\DecValTok{4}\NormalTok{,}\DecValTok{5}\NormalTok{)}
\FunctionTok{sum}\NormalTok{(x)}
\CommentTok{\#\textgreater{} [1] 15}

\NormalTok{x }\OtherTok{=} \FunctionTok{c}\NormalTok{(}\ConstantTok{TRUE}\NormalTok{, }\ConstantTok{TRUE}\NormalTok{)}
\FunctionTok{sum}\NormalTok{(x)}
\CommentTok{\#\textgreater{} [1] 2}
\end{Highlighting}
\end{Shaded}

\normalsize

Other operations are possible in the same way:

\footnotesize

\begin{Shaded}
\begin{Highlighting}[]
\NormalTok{x }\OtherTok{=} \FunctionTok{c}\NormalTok{(}\ConstantTok{TRUE}\NormalTok{, }\ConstantTok{FALSE}\NormalTok{)}
\FunctionTok{mean}\NormalTok{(x)}
\CommentTok{\#\textgreater{} [1] 0.5}
\end{Highlighting}
\end{Shaded}

\normalsize

To sum across the columns or rows of a dataframe or matrix, the
\texttt{colSums()} and \texttt{rowSums()} are optimized for speed.
Similar are \texttt{colMeans()} and \texttt{rowMeans()}.

\footnotesize

\begin{Shaded}
\begin{Highlighting}[]
\NormalTok{m }\OtherTok{=} \FunctionTok{matrix}\NormalTok{(}\DecValTok{1}\SpecialCharTok{:}\DecValTok{9}\NormalTok{, }\AttributeTok{nrow =} \DecValTok{3}\NormalTok{)}
\FunctionTok{colSums}\NormalTok{(m)}
\CommentTok{\#\textgreater{} [1]  6 15 24}
\FunctionTok{rowMeans}\NormalTok{(m)}
\CommentTok{\#\textgreater{} [1] 4 5 6}
\end{Highlighting}
\end{Shaded}

\normalsize
\end{frame}

\begin{frame}[fragile]{Split/Apply}
\protect\hypertarget{splitapply}{}
To apply a particular calculation to a subset of the data, the
\texttt{split()} function is convenient.

\footnotesize

\begin{Shaded}
\begin{Highlighting}[]
\CommentTok{\# split the data: poor vs the rest}
\FunctionTok{split}\NormalTok{(df, df}\SpecialCharTok{$}\NormalTok{poor)}
\CommentTok{\#\textgreater{} $\textasciigrave{}FALSE\textasciigrave{}}
\CommentTok{\#\textgreater{}    member      start        end worth  poor middle  rich duration}
\CommentTok{\#\textgreater{} 2    Mick 1962{-}07{-}12 2022{-}02{-}20   500 FALSE  FALSE  TRUE      100}
\CommentTok{\#\textgreater{} 3   Keith 1962{-}07{-}12 2022{-}02{-}20   500 FALSE  FALSE  TRUE      100}
\CommentTok{\#\textgreater{} 6 Charlie 1963{-}02{-}02 2021{-}08{-}24   250 FALSE   TRUE FALSE       98}
\CommentTok{\#\textgreater{} 7  Ronnie 1976{-}04{-}23 2022{-}02{-}20   200 FALSE   TRUE FALSE       77}
\CommentTok{\#\textgreater{} }
\CommentTok{\#\textgreater{} $\textasciigrave{}TRUE\textasciigrave{}}
\CommentTok{\#\textgreater{}    member      start        end worth poor middle  rich duration}
\CommentTok{\#\textgreater{} 1   Brian 1962{-}07{-}12 1969{-}06{-}09    10 TRUE  FALSE FALSE       12}
\CommentTok{\#\textgreater{} 4 Stewart 1962{-}07{-}12 1963{-}05{-}01     1 TRUE  FALSE FALSE        1}
\CommentTok{\#\textgreater{} 5    Bill 1962{-}12{-}07 1992{-}12{-}01    80 TRUE  FALSE FALSE       50}

\CommentTok{\# apply a calculation: compute the mean for each group}
\FunctionTok{sapply}\NormalTok{(}\FunctionTok{split}\NormalTok{(df}\SpecialCharTok{$}\NormalTok{worth, df}\SpecialCharTok{$}\NormalTok{poor), mean)}
\CommentTok{\#\textgreater{} FALSE  TRUE }
\CommentTok{\#\textgreater{} 362.5  30.3}
\end{Highlighting}
\end{Shaded}

\normalsize
\end{frame}

\begin{frame}[fragile]{Functions}
\protect\hypertarget{functions}{}
It is often convenient to create convenience functions. Ah yes.

A population standard deviation does not exist in base \texttt{R}. For
quick check, \texttt{?sd}: The Help page for \texttt{sd} states ``Like
\texttt{var} this uses denominator n-1.''

Let's roll our own:

\footnotesize

\begin{Shaded}
\begin{Highlighting}[]
\NormalTok{std }\OtherTok{\textless{}{-}} \ControlFlowTok{function}\NormalTok{(x) }\FunctionTok{sqrt}\NormalTok{(}\FunctionTok{sum}\NormalTok{((x}\SpecialCharTok{{-}}\FunctionTok{mean}\NormalTok{(x))}\SpecialCharTok{\^{}}\DecValTok{2}\NormalTok{)}\SpecialCharTok{/}\FunctionTok{length}\NormalTok{(x))}
\FunctionTok{std}\NormalTok{(df}\SpecialCharTok{$}\NormalTok{worth)}
\CommentTok{\#\textgreater{} [1] 196}
\end{Highlighting}
\end{Shaded}

\normalsize Note that we divide by \texttt{length(x)} and not
\texttt{length(x)-1}

Functions that stretch over several lines use an opening and closing
curly brace:

\footnotesize

\begin{Shaded}
\begin{Highlighting}[]
\NormalTok{weighted.std }\OtherTok{\textless{}{-}} \ControlFlowTok{function}\NormalTok{(x,w) \{}
    \FunctionTok{sqrt}\NormalTok{(}\FunctionTok{sum}\NormalTok{(w}\SpecialCharTok{*}\NormalTok{(x}\SpecialCharTok{{-}}\FunctionTok{weighted.mean}\NormalTok{(x,w))}\SpecialCharTok{\^{}}\DecValTok{2}\NormalTok{)}\SpecialCharTok{/}\FunctionTok{sum}\NormalTok{(w))}
\NormalTok{\}}
\FunctionTok{std}\NormalTok{(df}\SpecialCharTok{$}\NormalTok{worth)}
\CommentTok{\#\textgreater{} [1] 196}
\end{Highlighting}
\end{Shaded}

\normalsize
\end{frame}

\begin{frame}[fragile]{Dataframe V. List}
\protect\hypertarget{dataframe-v.-list}{}
There is a subtle difference between the two lines of code below.

\footnotesize

\begin{verbatim}
#> # A tibble: 3 x 1
#>   member 
#>   <chr>  
#> 1 Brian  
#> 2 Stewart
#> 3 Bill
#> [1] "Brian"   "Stewart" "Bill"
\end{verbatim}

\normalsize

They yield the same result here. Execute code snippets to see how
\texttt{R} works: So what is the difference? Try \texttt{str()} on the
resulting object to see.

\footnotesize

\begin{verbatim}
#> tibble [3 x 1] (S3: tbl_df/tbl/data.frame)
#>  $ member: chr [1:3] "Brian" "Stewart" "Bill"
#>  chr [1:3] "Brian" "Stewart" "Bill"
#>  chr [1:3] "Brian" "Stewart" "Bill"
#>  chr [1:3] "Brian" "Stewart" "Bill"
\end{verbatim}

\normalsize

Beware of tibbles, sometimes. See the difference?

\footnotesize

\begin{verbatim}
#> [1] 91
#> [1] 91
#> [1] NA
#> [1] 30.3
\end{verbatim}

\normalsize
\end{frame}

\begin{frame}{END OF DOCUMENT}
\protect\hypertarget{end-of-document}{}
This document ends here. Please check back later for updates.
\end{frame}

\end{document}
