% !TEX root = ../../MakeBeamer.tex
\title[Probability]{Review of Statistics: Bernoulli Distribution}
\date{}


\begin{document}


\inputfile{../Section-Cover-Problems}


\def\ask{Let $Y$ be a Bernoulli random variable with success probability $\Pr(Y=1)=p$,\ and let $Y_1,\ldots,Y_n$ be i.i.d. draws from this distribution. Let $\hat{p}$ be the fraction of successes ($1$s) in this sample.}


%%%%%%%%%%%%%%%%%%%%%%%%%%%%%%%%%%%%%%%%%%%%%%%%%%%%%%%%%
\begin{frame}
\frametitle{Problems and Applications: Bernoulli Distribution}
\ask
\begin{enumerate}\setcounter{enumi}{0}

\item Show that $\hat{p} = \mean{Y}$.

\begin{answer}
$\hat{p}$ is the fraction of successes in this sample, so
\begin{align*}
\hat{p} = \frac{1}{n} \cdot (Y_1 + \ldots + Y_n)
\end{align*}
The mean $\mean{Y}$ is the expected value of $Y$,
\begin{align*}
\mean{Y} 
  = \exp[Y]
  & = \frac{1}{n} \cdot (\exp[Y_1] + \ldots \exp[Y_n])
\end{align*}
And since $\hat{p}$ is a fixed sample value,
\begin{align*}
\hat{p}
    = \exp[\hat{p}]
    = \frac{1}{n} \cdot (\exp[Y_1] + \ldots \exp[Y_n])
    = \mean{Y}
\end{align*}
\end{answer}

\end{enumerate}
\end{frame}
%%%%%%%%%%%%%%%%%%%%%%%%%%%%%%%%%%%%%%%%%%%%%%%%%%%%%%%%%


%%%%%%%%%%%%%%%%%%%%%%%%%%%%%%%%%%%%%%%%%%%%%%%%%%%%%%%%%
\begin{frame}
\frametitle{Problems and Applications: Bernoulli Distribution}
\ask
\begin{enumerate}\setcounter{enumi}{1}

\item Show that $\hat{p}$ is an unbiased estimator of $p$.

\begin{answer}
The random variable $Y_i$ takes value $1$ with probability $p$ and $0$ with probability $1-p$,
\begin{align*}
\exp[Y_i] 
  = p \cdot 1 + (1-p) \cdot 0 = p
\end{align*}
By definition of the fraction of successes $\hat{p}$,
\begin{align*}
\hat{p}
  & = \frac{1}{n} \cdot (Y_1 + \ldots + Y_n)\\
\implies 
\exp[\hat{p}]
  & = \frac{1}{n} \cdot (\exp[Y_1] + \ldots + \exp[Y_n])\\
  & = \frac{1}{n} \cdot (p + \ldots + p)\\
  & = \frac{1}{n} \cdot n \cdot p
    = p
\implies
  \exp[\hat{p}-p] 
    = 0
\end{align*}
\end{answer}

\end{enumerate}
\end{frame}
%%%%%%%%%%%%%%%%%%%%%%%%%%%%%%%%%%%%%%%%%%%%%%%%%%%%%%%%%


%%%%%%%%%%%%%%%%%%%%%%%%%%%%%%%%%%%%%%%%%%%%%%%%%%%%%%%%%
\begin{frame}
\frametitle{Problems and Applications: Bernoulli Distribution}
\ask
\begin{enumerate}\setcounter{enumi}{2}

\item Show that $\var(\hat{p})=p(1-p)/n$.

\begin{answer}
\begin{align*}
\var(\hat{p}) 
    = \var(\mean{Y})
  & = \var\left(\frac{1}{n}(Y_1 + \ldots + Y_n)\right)\\
  & = \frac{1}{n^2} \cdot \var(Y_1 + \ldots + Y_n)
      \qquad \text{because $n$ is constant}\\
  & = \frac{1}{n^2} \cdot (\var(Y_1) + \ldots + \var(Y_n))
      \qquad \text{because the $Y_i$s are i.i.d.}\\
  & = \frac{1}{n^2} \cdot (\var(Y) + \ldots + \var(Y))
      \qquad \text{because the $Y_i$s are i.i.d.}\\
  & = \frac{1}{n^2} \cdot n \cdot \var(Y)\\
  & = \frac{1}{n} \cdot \var(Y)
\end{align*}
\end{answer}

\end{enumerate}
\end{frame}
%%%%%%%%%%%%%%%%%%%%%%%%%%%%%%%%%%%%%%%%%%%%%%%%%%%%%%%%%


%%%%%%%%%%%%%%%%%%%%%%%%%%%%%%%%%%%%%%%%%%%%%%%%%%%%%%%%%
\begin{frame}
\frametitle{Problems and Applications: Bernoulli Distribution}
\ask
\begin{enumerate}\setcounter{enumi}{3}

\item Show that $\var(\hat{p})=p(1-p)/n$.

\begin{answer}
Now compute $\var(Y)$
\begin{align*}
\exp[Y]
  & = p \cdot 1\phantom{^2} + (1-p) \cdot 0\phantom{^2} 
   = p\\
\exp[Y^2]
 & = p \cdot 1^2 + (1-p) \cdot 0^2 
   = p\\
\var(Y) 
 & = \exp[Y^2] - (\exp[Y])^2
   = p - p^2
   = p(1-p)\\
\implies 
\var(\hat{p}) 
 & = \var(\mean{Y}) 
   = \frac{\var(Y)}{n}
   = \frac{p(1-p)}{n}
\end{align*}
\end{answer}

\end{enumerate}
\end{frame}
%%%%%%%%%%%%%%%%%%%%%%%%%%%%%%%%%%%%%%%%%%%%%%%%%%%%%%%%%


\end{document}