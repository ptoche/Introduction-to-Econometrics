% !TEX root = ../../MakeBeamer.tex
\title[Probability]{Review of Statistics: California Test Scores}
\date{}


\begin{document}


\inputfile{../Section-Cover-Problems}


%%%%%%%%%%%%%%%%%%%%%%%%%%%%%%%%%%%%%%%%%%%%%%%%%%%%%%%%%
\begin{frame}
\frametitle{Problems and Applications: California Test Scores}
Data on fifth-grade test scores (reading and mathematics) for $420$ school districts in California yield average score $\mean{Y}=654.2$ and standard deviation $s_{Y}=19.1$.
\pause
\begin{enumerate}
\item Construct a $95\%$ confidence interval for the mean test score in the population.
\item When the districts were divided into those with small classes ($< 20$ students per teacher) and those with large classes ($\geq 20$ students per teacher), the following results were found:
\begin{center}
\begin{tabular}{lccc}
\toprule
Class Size & Average Score ($\mean{Y}$)      
                     & Standard Deviation ($s_{Y}$)
                               &   $n$\\
\midrule
Small      & $657.4$ &  $19.4$ & $238$\\
Large      & $650.0$ &  $17.9$ & $182$\\
\bottomrule
\end{tabular}
\end{center}
Is there statistically significant evidence that the districts with smaller classes have higher average test scores? Explain.
\end{enumerate}
\end{frame}
%%%%%%%%%%%%%%%%%%%%%%%%%%%%%%%%%%%%%%%%%%%%%%%%%%%%%%%%%


%%%%%%%%%%%%%%%%%%%%%%%%%%%%%%%%%%%%%%%%%%%%%%%%%%%%%%%%%
\begin{frame}
\frametitle{Problems and Applications: California Test Scores}

$n=420$, $\mean{Y}=654.2$, $s_{Y}=19.1$.

\begin{enumerate}\setcounter{enumi}{0}

\item Construct a $95\%$ confidence interval for the mean test score in the population.

\begin{answer}
As the population standard deviation is unknown, the asymptotic distribution of the test statistic follows a Student-$t$ distribution. 
The significance level is $1-0.95=0.05$.
The corresponding critical $t$-statistic for $420-1=419$ degrees of freedom is about $1.965642$. This compares with about $1.959964$ for the standard normal distribution. 
% alpha = 1-0.95
% qt(1-alpha/2,df=419)
% 1.965642
% qnorm(1-alpha/2)
% 1.959964

The standard error is:
\begin{align*}
\SE = \frac{s_{Y}}{\sqrt{n}}
    = \frac{19.1}{\sqrt{420}}
    \approx 0.93
% 19.1/sqrt(420)
% 0.9319846
\end{align*}

A two-sided confidence interval for mean test score in the population is:
\begin{align*}
\mean{Y} \pm t_{1-\alpha/2} \times \SE
   & = 654.2 \pm 1.96 \times 0.93\\
   & = (652.37, 656.03)
% alpha = 1-0.95
% 654.2 + qt(1-alpha/2,df=419) * 19.1/sqrt(420)
% 656.0319
% 654.2 - qt(1-alpha/2,df=419) * 19.1/sqrt(420)
% 652.3681
% qt(1-alpha/2,df=419) * 19.1/sqrt(420)
% 1.831948
\end{align*}
The margin of error is about $1.83$.
\end{answer}

\end{enumerate}
\end{frame}
%%%%%%%%%%%%%%%%%%%%%%%%%%%%%%%%%%%%%%%%%%%%%%%%%%%%%%%%%


%%%%%%%%%%%%%%%%%%%%%%%%%%%%%%%%%%%%%%%%%%%%%%%%%%%%%%%%%
\begin{frame}
\frametitle{Problems and Applications: California Test Scores}

\begin{enumerate}\setcounter{enumi}{1}

\item Is there statistically significant evidence that the districts with smaller classes have higher average test scores? Explain.

\begin{answer}
A two-sided test:
\begin{align*}
H_{0} \colon & \mu_{\text{large}} - \mu_{\text{small}} = 0\\
H_{1} \colon & \mu_{\text{large}} - \mu_{\text{small}} \ne 0
\end{align*}
The Student-$t$ statistic for this test:
\begin{align*}
t   = \frac{(657.4-650.0)-0}{\sqrt{\dfrac{(19.4)^{2}}{238}+\dfrac{(17.9)^{2}}{182}}}
    \approx \frac{7.4}{1.828}
    \approx 4.05
% den = sqrt((19.4)^(2)/238+(17.9)^(2)/182)
% 1.82807
% 657.4-650.0
% 7.4
% (657.4-650.0)/den
% 4.047986
\end{align*}
There are enough degrees of freedom that we can refer to the standard normal distribution for an accurate critical value.
Since $|4.05| > 1.96$, we reject the null hypothesis of no effect of class sizes.
\end{answer}

\end{enumerate}
\end{frame}
%%%%%%%%%%%%%%%%%%%%%%%%%%%%%%%%%%%%%%%%%%%%%%%%%%%%%%%%%

\end{document}
