  % !TEX root = ../../MakeBeamer.tex
\title[Weight]{Hypothesis Testing \& Confidence Intervals: California Test Scores}
\date{}


\begin{document}


\inputfile{../Section-Cover-Problems}


\def\ask{\begin{align*}
\verywidehat{TestScore} 
  = \muse{520.4}{20.4} - \muse{5.82}{2.21} \times CS,
  \quad R^2 = 0.08,
  \quad \vn{SER} = 11.5
\end{align*}}


%%%%%%%%%%%%%%%%%%%%%%%%%%%%%%%%%%%%%%%%%%%%%%%%%%%%%%%%%
\begin{frame}
\frametitle{Confidence Intervals}
A researcher, using data on class size (CS) and average test scores from $100$ third-grade classes, estimates the OLS regression:
\ask
\begin{enumerate}
\item Construct a $95\%$ confidence interval for $\beta_1$, the regression slope coefficient.
\item Calculate the p-value for the two-sided test of the null hypothesis
$H_0\colon \beta_1 = 0$. Do you reject the null hypothesis at the $5\%$ level? At the $1\%$ level?
\item Calculate the $p$-value for the two-sided test of the null hypothesis $H_{0}\colon\beta_{1}=-5.6$. Without doing any additional calculations, determine whether $-5.6$ is contained in the $95\%$ confidence interval for $\beta_{1}$.
\item Construct a $99\%$ confidence interval for $\beta_{0}$.
\end{enumerate}
\end{frame}
%%%%%%%%%%%%%%%%%%%%%%%%%%%%%%%%%%%%%%%%%%%%%%%%%%%%%%%%%


%%%%%%%%%%%%%%%%%%%%%%%%%%%%%%%%%%%%%%%%%%%%%%%%%%%%%%%%%
\begin{frame}
\frametitle{Confidence Intervals}
\ask
\begin{enumerate}\setcounter{enumi}{0}

\item Construct a $95\%$ confidence interval for $\beta_1$, the regression slope coefficient.

\begin{answer}
A two-tailed $\alpha\%$ confidence interval for $\beta_{1}$ is:
\begin{align*}
\hat{\beta}_{1} & \pm t_{\alpha/2} \cdot \SE(\hat{\beta}_{1}) \\
       = - 5.82 & \pm t_{\alpha/2} \cdot 2.21
\end{align*}
As the sample size is $n=100$, the Student-$t$ distribution with $n-2=98$ degrees of freedom is reasonably well approximated by the standard normal distribution. 
The critical $t$- or $z$-value may be read from a probability table or computed with, say, the R command \Rinline{qt(1-0.05/2,df=98)} or \Rinline{qnorm(1-0.05/2)}.
% qt(1-0.05/2,df=98)
% 1.984467
% qnorm(1-0.05/2)
% 1.959964

A two-tailed $95\%$ confidence interval for $\beta_{1}$ is:
\begin{align*}
-10.15 & < \beta_{1} < -1.49
:\quad\text{based on the standard normal distribution $z_{0.05/2}\approx1.96$}\\
% -5.82 - qnorm(1-0.05/2) * 2.21
% -10.15152
% -5.82 + qnorm(1-0.05/2) * 2.21
% -1.48848
-10.20 & < \beta_{1} < -1.44
:\quad\text{based on the Student-$t$ distribution $t_{0.05/2}(98)\approx1.98$}
% -5.82 - qt(1-0.05/2,df=98) * 2.21
% -10.20567
% -5.82 + qt(1-0.05/2,df=98) * 2.21
% -1.434327
\end{align*}
\end{answer}

\end{enumerate}
\end{frame}
%%%%%%%%%%%%%%%%%%%%%%%%%%%%%%%%%%%%%%%%%%%%%%%%%%%%%%%%%


%%%%%%%%%%%%%%%%%%%%%%%%%%%%%%%%%%%%%%%%%%%%%%%%%%%%%%%%%
\begin{frame}
\frametitle{Confidence Intervals}
\ask
\begin{enumerate}\setcounter{enumi}{1}

\item Calculate the p-value for the two-sided test of the null hypothesis
$H_0\colon \beta_1 = 0$. Do you reject the null hypothesis at the $5\%$ level? At the $1\%$ level?

\begin{answer}
The test statistic associated with $H_{0}$ is:
\begin{align*}
t_{0} 
    = \frac{\hat{\beta}_{1}-\beta_{1,0}}{\SE(\hat{\beta}_{1})}
    = \frac{-5.82-0}{2.21}
    \approx -2.63
% -5.82/2.21
% -2.633484
\end{align*} 
The two-sided $p$-value is therefore:
\begin{align*}
p\text{-value} 
    = 2 \Phi(t_{0})
    \approx 0.0085
% 2 * pnorm(-5.82/2.21)
% 0.008451378
\end{align*} 
The $p$-value may be computed with \Rinline{2*pnorm(-2.63)}.

Since the $p$-value is smaller than one percent, we reject the null hypothesis at the $5\%$ significance level and at the $1\%$ level.
\end{answer}

\end{enumerate}
\end{frame}
%%%%%%%%%%%%%%%%%%%%%%%%%%%%%%%%%%%%%%%%%%%%%%%%%%%%%%%%%


%%%%%%%%%%%%%%%%%%%%%%%%%%%%%%%%%%%%%%%%%%%%%%%%%%%%%%%%%
\begin{frame}
\frametitle{Confidence Intervals}
\ask

\begin{enumerate}\setcounter{enumi}{2}

\item Calculate the $p$-value for the two-sided test of the null hypothesis $H_{0}\colon\beta_{1}=-5.6$. Without doing any additional calculations, determine whether $-5.6$ is contained in the $95\%$ confidence interval for $\beta_{1}$.

\begin{answer}
The test statistic associated with $H_{0}$ is:
\begin{align*}
t_{0} 
    = \frac{\hat{\beta}_{1}-\beta_{1,0}}{\SE(\hat{\beta}_{1})}
    = \frac{-5.82+5.6}{2.21}
    \approx -0.10
% (-5.82+5.6)/2.21
% -0.09954751
\end{align*} 
The two-sided $p$-value is therefore:
\begin{align*}
p\text{-value} 
    = 2 \Phi(t_{0})
    \approx 0.92 
% 2*pnorm(-0.09954751)
% 0.9207036
\end{align*} 
The $p$-value may be computed with \Rinline{2*pnorm(-0.10)}.
The $p$-value is large and we cannot reject the null hypothesis at the usual significance levels. Since we cannot reject the null at the $5\%$ significance level, $-5.6$ is contained in the $95\%$ confidence interval for $\beta_{1}$.
\end{answer}

\end{enumerate}
\end{frame}
%%%%%%%%%%%%%%%%%%%%%%%%%%%%%%%%%%%%%%%%%%%%%%%%%%%%%%%%%


%%%%%%%%%%%%%%%%%%%%%%%%%%%%%%%%%%%%%%%%%%%%%%%%%%%%%%%%%
\begin{frame}
\frametitle{Confidence Intervals}
\ask

\begin{enumerate}\setcounter{enumi}{3}

\item Construct a $99\%$ confidence interval for $\beta_{0}$.

\begin{answer}
A two-tailed $\alpha\%$ confidence interval for $\beta_{0}$ is: 
\begin{align*}
\hat{\beta}_{0} & \pm t_{\alpha/2} \cdot \SE(\hat{\beta}_{0}) \\
        = 520.4 & \pm t_{\alpha/2} \cdot 20.4 
\end{align*}
As the sample size is $n=100$, the Student-$t$ distribution with $n-2=98$ degrees of freedom is reasonably well approximated by the standard normal distribution. 
The critical $t$-value, $t_{\alpha/2}\approx2.5758$, may be read from a probability table or computed with the R command \Rinline{qt(1-0.01/2,df=98)} or \Rinline{qnorm(1-0.01/2)}.
% qnorm(1-0.01/2)
% 2.575829
% qt(1-0.01/2,df=98)
% 2.626931

A two-tailed $99\%$ confidence interval for $\beta_{0}$ is:
\begin{align*}
467.85 & < \beta_{0} < 572.95
:\quad\text{based on the standard normal distribution $z_{0.01/2}\approx2.58$}\\
% 520.4 - qnorm(1-0.01/2) * 20.4
% 467.8531
% 520.4 + qnorm(1-0.01/2) * 20.4
% 572.9469
466.81 & < \beta_{0} < 573.99
:\quad\text{based on the Student-$t$ distribution $t_{0.01/2}(98)\approx2.63$}
% 520.4 - qt(1-0.01/2,df=98) * 20.4
% 466.8106
% 520.4 + qt(1-0.01/2,df=98) * 20.4
% 573.9894
\end{align*}
\end{answer}

\end{enumerate}
\end{frame}
%%%%%%%%%%%%%%%%%%%%%%%%%%%%%%%%%%%%%%%%%%%%%%%%%%%%%%%%%


\end{document}
