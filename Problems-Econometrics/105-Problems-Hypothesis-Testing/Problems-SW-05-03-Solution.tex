% !TEX root = ../../MakeBeamer.tex
\title[Weight]{Hypothesis Testing \& Confidence Intervals: Weight \& Height}
\date{}

\begin{document}


\inputfile{../Section-Cover-Problems}


%%%%%%%%%%%%%%%%%%%%%%%%%%%%%%%%%%%%%%%%%%%%%%%%%%%%%%%%%
\begin{frame}
\frametitle{Problems and Applications}
\exercise{Stock \& Watson, Introduction (4th), Chapter~5, Exercise~3.}
Suppose a random sample of $200$ $20$-year-old men is selected from a population and their heights and weights are recorded. A regression of weight on height yields:
\begin{align*}
\verywidehat{\vn{Weight}} 
  = \muse{-99.41}{2.15} + \muse{3.94}{0.31}\,\vn{Height},
  \quad R^2 = 0.81,
  \quad \vn{SER} = 10.2
\end{align*}
where $\vn{Weight}$ is measured in pounds and $\vn{Height}$ is measured in inches. Two of your classmates differ in height by $1.5$ inches. Construct a $99\%$ confidence interval for the difference in their weights.
\end{frame}
%%%%%%%%%%%%%%%%%%%%%%%%%%%%%%%%%%%%%%%%%%%%%%%%%%%%%%%%%


%%%%%%%%%%%%%%%%%%%%%%%%%%%%%%%%%%%%%%%%%%%%%%%%%%%%%%%%%
\begin{frame}
\frametitle{Confidence Interval for a Mean Difference}

In the regression of $Y$ on $X$, construct a confidence interval for the difference $\Delta Y$.

\begin{answer}
Let $u$ denote the error in the regression of $Y$ on $X$:
\begin{align*}
Y = \beta_{0} + \beta_{1} X + u
\end{align*}
Consider two observations $X_{i}$ and $X_{j}$ and let the difference be $\Delta X=X_{i}-X_{j}$. Likewise for $\Delta Y$. By construction, the data and parameters satisfy:
\begin{align*}
Y_{i} & = \beta_{0} + \beta_{1} X_{i} + u_{i}\\
Y_{j} & = \beta_{0} + \beta_{1} X_{j} + u_{j}\\
\implies 
\Delta Y & = \beta_{1}\,\Delta X + \Delta u
\end{align*}
It follows that a confidence interval for $\Delta Y$ is a confidence interval for $\beta_{1}\,\Delta X$ and since $\Delta X$ is a constant, the confidence interval is just:
\begin{align*}
\left[ \hat{\beta}_{1} \pm t_{\alpha/2} \cdot \SE(\hat{\beta}_{1}) \right] \cdot \Delta X 
\end{align*}
\end{answer}

\end{frame}
%%%%%%%%%%%%%%%%%%%%%%%%%%%%%%%%%%%%%%%%%%%%%%%%%%%%%%%%%


%%%%%%%%%%%%%%%%%%%%%%%%%%%%%%%%%%%%%%%%%%%%%%%%%%%%%%%%%
\begin{frame}
\frametitle{Test Scores}

Construct a $99\%$ confidence interval for the difference in their weights.

\begin{answer}
To construct a confidence interval for the difference in their weights, relate the expected difference in their weights to the observed difference in their height:
\begin{align*}
\Delta\verywidehat{\vn{Weight}} 
  = \Delta\,\vn{Height}\,\hat{\beta}_{1}
\end{align*}
The standard error for the expected difference is:
\begin{align*}
\SE(\Delta\verywidehat{\vn{Weight}})
  = \Delta\,\vn{Height}\, \SE(\hat{\beta}_{1})
\end{align*}
A confidence interval for the difference in their weights may be constructed in a manner analogous to a confidence interval for the slope coefficient $\beta_{1}$:
\begin{align*}
\hat{\beta}_{1} \cdot \Delta\,\vn{Height} 
    & \pm t_{\alpha/2} \cdot \SE(\hat{\beta}_{1}) \cdot \Delta\,\vn{Height} \\
    = \quad 3.94 \cdot 1.5 
    & \pm 2.58 \cdot 0.31 \cdot 1.5 
\end{align*}
The critical $t$-value, $t_{\alpha/2}\approx2.5758$, may be computed with \Rinline{qnorm(1-0.01/2)}.
A $99\%$ confidence interval for the difference in their weights is therefore:
\begin{align*}
4.71 < \beta_{1} < 7.11
% 1.5 * 3.94 - 1.5 * 2.5758 * 0.31
% 4.712253
% 1.5 * 3.94 + 1.5 * 2.5758 * 0.31
% 7.107747
\end{align*}
\end{answer}

\end{frame}
%%%%%%%%%%%%%%%%%%%%%%%%%%%%%%%%%%%%%%%%%%%%%%%%%%%%%%%%%


\end{document}
