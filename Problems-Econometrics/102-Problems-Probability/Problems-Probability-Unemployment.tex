% !TEX root = ../../MakeBeamer.tex
\title[Probability]{Review of Probability: Tabulation}
\date{}


\begin{document}


\inputfile{../Section-Cover-Problems}


%%%%%%%%%%%%%%%%%%%%%%%%%%%%%%%%%%%%%%%%%%%%%%%%%%%%%%%%%
\begin{frame}
\frametitle{Probability \& Unemployment}
The following table gives the joint probability distribution between employment status and college graduation among those either employed or looking for work (unemployed) in the working-age U.S. population for September 2017.
\smallskip
\begin{enumerate}
\item Compute $\exp(Y)$.
\item The unemployment rate is the fraction of the labor force that is unemployed. Show that the unemployment rate is given by $1-\exp(Y)$.
\item Calculate $\exp(Y|X=1)$ and $\exp(Y|X=0)$.
\item Calculate the unemployment rate for (i) college graduates and (ii) non-college graduates.
\item A randomly selected member of this population reports being unemployed. What is the probability that this worker is a college graduate? A non–college graduate?
\item Are educational achievement and employment status independent? Explain.
\end{enumerate}
\end{frame}
%%%%%%%%%%%%%%%%%%%%%%%%%%%%%%%%%%%%%%%%%%%%%%%%%%%%%%%%%


%%%%%%%%%%%%%%%%%%%%%%%%%%%%%%%%%%%%%%%%%%%%%%%%%%%%%%%%%
\begin{frame}
\frametitle{Probability \& Unemployment}
\begin{center}
\newcolumntype{Z}{>{\centering\arraybackslash}wc{0.16\linewidth}}% 
\begin{tabular*}{\linewidth}{p{0.4\linewidth}@{\extracolsep{\fill}}ZZZ} 
\multicolumn{4}{l}{\textbf{Employment \& College Graduation} (Population aged 25 and above, September 2017)}\\
\toprule 
    & \text{\makecell[c]{Unemployed \\ $Y=0$}} 
             & \text{\makecell[c]{Employed \\ $Y=1$}} 
                     & \text{\makecell[c]{\\ Total}} \\
\cmidrule{2-4}
\text{Non-College Graduates ($X=0$)}
    &  0.026 & 0.576 & 0.602 \\
\text{College Graduates ($X=1$)}
    &  0.009 & 0.389 & 0.398 \\
\text{Total}
    &  0.035 & 0.965 & 1.000 \\
\bottomrule
\end{tabular*}
\end{center}
\end{frame}
%%%%%%%%%%%%%%%%%%%%%%%%%%%%%%%%%%%%%%%%%%%%%%%%%%%%%%%%%


%%%%%%%%%%%%%%%%%%%%%%%%%%%%%%%%%%%%%%%%%%%%%%%%%%%%%%%%%
\begin{frame}
\frametitle{Probability \& Unemployment}
\begin{enumerate}\setcounter{enumi}{0}

\item Compute $\exp(Y)$.

\begin{answer}
\begin{align*}
\exp[Y] 
  & = 0 \times \Pr(Y=0) + 1 \times \Pr(Y=1) \\
  & = 0 \times 0.035 \hspace{25pt} + 1 \times 0.965 \\
  & = 0.965
\end{align*}
\end{answer}

\end{enumerate}
\end{frame}
%%%%%%%%%%%%%%%%%%%%%%%%%%%%%%%%%%%%%%%%%%%%%%%%%%%%%%%%%


%%%%%%%%%%%%%%%%%%%%%%%%%%%%%%%%%%%%%%%%%%%%%%%%%%%%%%%%%
\begin{frame}
\frametitle{Probability \& Unemployment}
\begin{enumerate}\setcounter{enumi}{1}

\item Show that the unemployment rate is given by $1-\exp(Y)$.

\begin{answer}
The probability of unemployment is also the unemployment rate $u=\Pr(Y=0)$.
\begin{align*}
\Pr(Y=0) 
  & = 0.035 \\ 
1 - \exp[Y] = 1 - 0.965 
  & = 0.035
\end{align*}
\end{answer}

\end{enumerate}
\end{frame}
%%%%%%%%%%%%%%%%%%%%%%%%%%%%%%%%%%%%%%%%%%%%%%%%%%%%%%%%%


%%%%%%%%%%%%%%%%%%%%%%%%%%%%%%%%%%%%%%%%%%%%%%%%%%%%%%%%%
\begin{frame}
\frametitle{Probability \& Unemployment}
\begin{enumerate}\setcounter{enumi}{2}

\item Calculate $\exp(Y|X=1)$ and $\exp(Y|X=0)$.

\begin{answer}
\begin{align*}
% 0.576/0.602
\exp(Y|X=1) & = 0 \times \Pr(Y=0|X=1) + 1 \times \Pr(Y=1|X=1) \\
  & = \Pr(Y=1|X=1) \\
  & = \frac{\Pr(Y=1,X=1)}{\Pr(X=1)}
    = \frac{0.389}{0.398}
    = 0.977 \\[1ex]
\exp(Y|X=0) & = 0 \times \Pr(Y=0|X=0) + 1 \times \Pr(Y=1|X=0) \\
  & = \Pr(Y=1|X=0) \\
  & = \frac{\Pr(Y=1,X=0)}{\Pr(X=0)}
    = \frac{0.576}{0.602}
    = 0.957 \\
\end{align*}
\end{answer}

\end{enumerate}
\end{frame}
%%%%%%%%%%%%%%%%%%%%%%%%%%%%%%%%%%%%%%%%%%%%%%%%%%%%%%%%%


%%%%%%%%%%%%%%%%%%%%%%%%%%%%%%%%%%%%%%%%%%%%%%%%%%%%%%%%%
\begin{frame}
\frametitle{Probability \& Unemployment}
\begin{enumerate}\setcounter{enumi}{3}

\item Calculate the unemployment rate for (i) college graduates and (ii) non-college graduates.

\begin{answer}
(i)~Unemployment rate for college graduates:
\begin{align*}
1 - \exp[Y|X=1] 
  = 1 - 0.977
  & = 0.023
\end{align*}

(ii)~Unemployment rate for non-college graduates:
\begin{align*}
1 - \exp[Y|X=0] 
  = 1 - 0.957
  & = 0.043
\end{align*}
\end{answer}

\end{enumerate}
\end{frame}
%%%%%%%%%%%%%%%%%%%%%%%%%%%%%%%%%%%%%%%%%%%%%%%%%%%%%%%%%\\


%%%%%%%%%%%%%%%%%%%%%%%%%%%%%%%%%%%%%%%%%%%%%%%%%%%%%%%%%
\begin{frame}
\frametitle{Probability \& Unemployment}
\begin{enumerate}\setcounter{enumi}{4}

\item What is the probability that this unemployed worker is a college graduate? A non–college graduate?

\begin{answer}
Probability this unemployed person is a college graduate:
\begin{align*}
\Pr[X=1|Y=0] 
  & = \frac{\Pr[X=1,Y=0]}{\Pr[Y=0]}
    = \frac{0.009}{0.035}
    = 0.257
\end{align*}

Probability this unemployed person is not a college graduate:
\begin{align*}
\Pr[X=0|Y=0] 
  & = \frac{\Pr[X=0,Y=0]}{\Pr[Y=0]}
    = \frac{0.026}{0.035}
    = 0.743
\end{align*}
\end{answer}

\end{enumerate}
\end{frame}
%%%%%%%%%%%%%%%%%%%%%%%%%%%%%%%%%%%%%%%%%%%%%%%%%%%%%%%%%


%%%%%%%%%%%%%%%%%%%%%%%%%%%%%%%%%%%%%%%%%%%%%%%%%%%%%%%%%
\begin{frame}
\frametitle{Probability \& Unemployment}
\begin{enumerate}\setcounter{enumi}{5}

\item Are educational achievement and employment status independent? Explain.

\begin{answer}
Let's check the plausibility of independence for non-college graduates:
\begin{align*}
\Pr[X=0,Y=0] 
  & = 0.026 \\
\Pr[X=0] \times \Pr[Y=0] 
  & = 0.602 \times 0.035
    = 0.022
\end{align*}
Since $0.022 \approx 0.026$, the hypothesis is still plausible. 

Let's check it for college graduates:
\begin{align*}
\Pr[X=1,Y=0] 
  & = 0.009 \\
\Pr[X=1] \times \Pr[Y=0] 
  & = 0.398 \times 0.035
    = 0.014
\end{align*}
Since $0.009 \ll 0.014$, the independence hypothesis is shaky.

Even more convincing evidence against independence:
\begin{align*}
\Pr[X=0|Y=0] = 0.743 \ne 0.602 = \Pr[X=0] \\
\Pr[X=1|Y=0] = 0.257 \ne 0.398 = \Pr[X=1] \\
\end{align*}
\end{answer}

\end{enumerate}
\end{frame}
%%%%%%%%%%%%%%%%%%%%%%%%%%%%%%%%%%%%%%%%%%%%%%%%%%%%%%%%%

\end{document}