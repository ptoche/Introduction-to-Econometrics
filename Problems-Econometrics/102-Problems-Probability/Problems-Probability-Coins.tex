% !TEX root = ../../MakeBeamer.tex
\title[Probability]{Review of Probability: Distribution Function}
\date{}


\begin{document}


\inputfile{../Section-Cover-Problems}


\def\ask{Let $Y$ denote the number of ``heads'' that occur when two coins are tossed.}

%%%%%%%%%%%%%%%%%%%%%%%%%%%%%%%%%%%%%%%%%%%%%%%%%%%%%%%%%
\begin{frame}
\frametitle{Probability Distribution}
\ask
\begin{enumerate}\setcounter{enumi}{0}
\item Derive the probability distribution of $Y$.
\end{enumerate}

\begin{answer}
Let the sample space be $\Omega=\{(H,H),(H,T),(T,H),(T,T)\}$, with associated probabilities
\begin{align*}
\Pr[(H,H)] 
  = \Pr[(H,T)] 
  = \Pr[(T,H)]
  = \Pr[(H,H)] 
  = \frac{1}{4}
\end{align*}
We have $Y=0$ if $(T,T)$; $Y=1$ if $(H,T)$ or $(T,H)$; and $Y=2$ if $(H,H)$.
The sample space is $\Omega_Y=\{0,1,2\}$ with probability distribution:
\begin{align*}
\Pr[Y=0] & = \frac{1}{4} \\
\Pr[Y=1] & = \frac{1}{4} + \frac{1}{4} = \frac{1}{2}\\
\Pr[Y=2] & = \frac{1}{4}
\end{align*}
Importantly, check that $\Pr[Y=0]+\Pr[Y=1]+Pr[Y=2]=1$. 
\end{answer}

\end{frame}
%%%%%%%%%%%%%%%%%%%%%%%%%%%%%%%%%%%%%%%%%%%%%%%%%%%%%%%%%


%%%%%%%%%%%%%%%%%%%%%%%%%%%%%%%%%%%%%%%%%%%%%%%%%%%%%%%%%
\begin{frame}
\frametitle{Probability Distribution}
\ask
\begin{enumerate}\setcounter{enumi}{1}
\item Derive the cumulative probability distribution of $Y$.
\end{enumerate}

\begin{answer}
The cumulative distribution adds up the probabilities for each outcome from $Y=0$ to $Y=2$:
\begin{align*}
\Pr[Y=0] & = \frac{1}{4} \\
\Pr[Y=1~\text{or}~Y=0] & = \frac{1}{4} + \frac{1}{2} = \frac{3}{4}\\
\Pr[Y=2~\text{or}~Y=1~\text{or}~Y=0] & = \frac{1}{4} + \frac{3}{4} = 1
\end{align*}
By an abuse of notation, we sometimes write $\Pr[Y<0]=0$ and $\Pr[Y<\infty]=1$.
\end{answer}

\end{frame}
%%%%%%%%%%%%%%%%%%%%%%%%%%%%%%%%%%%%%%%%%%%%%%%%%%%%%%%%%


%%%%%%%%%%%%%%%%%%%%%%%%%%%%%%%%%%%%%%%%%%%%%%%%%%%%%%%%%
\begin{frame}
\frametitle{Probability Distribution}
\ask
\begin{enumerate}\setcounter{enumi}{2}
\item Derive the mean and variance of $Y$.
\end{enumerate}

\begin{answer}
Intuitively (but beware of intuitions!) we guess $\exp[Y]=1$, which is easy to check:
\begin{align*}
\exp[Y] 
  = \frac{1}{4} \cdot 0
  + \frac{1}{2} \cdot 1
  + \frac{1}{4} \cdot 2
  = 1 
\end{align*}
The variance may be calculated directly:
\begin{align*}
\var[Y] 
& = \frac{1}{4} \cdot \left(0-1\right)^2
  + \frac{1}{2} \cdot \left(1-1\right)^2
  + \frac{1}{4} \cdot \left(2-1\right)^2
  = \frac{1}{2}
\end{align*}

Or indirectly:
\begin{align*}
\var[Y] 
& = \exp[Y^2] - \left(\exp[Y]\right)^2\\
& = \frac{1}{4} \cdot 0^2
  + \frac{1}{2} \cdot 1^2
  + \frac{1}{4} \cdot 2^2
  - 1^2
  = \frac{3}{2} - 1 
  = \frac{1}{2}
    \tick
\end{align*}
\end{answer}

\end{frame}
%%%%%%%%%%%%%%%%%%%%%%%%%%%%%%%%%%%%%%%%%%%%%%%%%%%%%%%%%


\end{document}
