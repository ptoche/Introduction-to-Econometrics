

%%%%%%%%%%%%%%%%%%%%%%%%%%%%%%%%%%%%%%%%%%%%%%%%%%%%%%%%%
\begin{frame}
\frametitle{Homoskedasticity and Heteroskedasticity}
\begin{itemize}
\item The error term $u_{i}$ is homoskedastic if the variance of the conditional distribution of $u_{i}$ given $X_{i}$ is constant for $i=1,\ldots,n$ and in particular does not depend on $X_{i}$. Otherwise, the error term is heteroskedastic.
\item Whether the errors are homoskedastic or heteroskedastic, the OLS estimator is unbiased, consistent, and asymptotically normal.
\item Economic theory rarely gives any reason to believe that the errors are homoskedastic --- It is prudent to assume that the errors might be heteroskedastic. Many software programs report homoskedasticity- only standard errors as their default setting.
\item If the error term is homoskedastic, the formulas for the variances of $\hat{\beta}_{0}$ and $\hat{\beta}_{1}$ simplify:
\begin{align*}
\sigma_{\hat{\beta}_{0}}^{2}
  & = \frac{\tfrac{1}{n}\cdot\sigma_{u}^{2} \cdot \tfrac{1}{n}\sum_{i=1}^{n}X_{i}^{2}}{\tfrac{1}{n}\sum_{i=1}^{n}(X_{i}-\mean{X})^{2}}\\
\sigma_{\hat{\beta}_{1}}^{2}
  & = \frac{\tfrac{1}{n}\sigma_{u}^{2}}{\tfrac{1}{n}\sum_{i=1}^{n}(X_{i}-\mean{X})^{2}}
\end{align*}
\end{itemize}
\end{frame}
%%%%%%%%%%%%%%%%%%%%%%%%%%%%%%%%%%%%%%%%%%%%%%%%%%%%%%%%%


%%%%%%%%%%%%%%%%%%%%%%%%%%%%%%%%%%%%%%%%%%%%%%%%%%%%%%%%%
\begin{frame}
\frametitle{Homoskedasticity and Heteroskedasticity}
\begin{minipage}[t]{0.48\linewidth}
\begin{figure}
\centering
\includegraphics[width=\linewidth,height=0.4\textheight,keepaspectratio]%
{StockWatson4e-05-fig-02a-Zoom}
\only<1->{\caption{\textbf{The spread of these distributions does not depend on $x$.}}}
\end{figure}
\end{minipage}%
\begin{minipage}[t]{0.48\linewidth}
\begin{figure}
\centering
\includegraphics[width=\linewidth,height=0.4\textheight,keepaspectratio]%
{StockWatson4e-05-fig-02b-Zoom}
\only<1->{\caption{\textbf{These become more spread out for larger class sizes.}}}
\end{figure}
\end{minipage}
\end{frame}
%%%%%%%%%%%%%%%%%%%%%%%%%%%%%%%%%%%%%%%%%%%%%%%%%%%%%%%%%


%%%%%%%%%%%%%%%%%%%%%%%%%%%%%%%%%%%%%%%%%%%%%%%%%%%%%%%%%
\begin{frame}
\frametitle{Heteroskedasticity: Application}
\begin{itemize}
\item Workers with more education have higher earnings than workers with less education. On average, hourly earnings increase by $\$2.37$ for each additional year of education.
\item The spread of the distribution of earnings increases with the years of education. While some workers with many years of education have low-paying jobs, very few workers with low levels of education have high-paying jobs. For workers with $10$ years of education, the standard deviation of the residuals is $\$6.31$; for workers with a high school diploma, it is $\$8.54$; and for workers with a college degree, $\$13.55$.
\item Not all college graduates will be earning $\$75$ per hour by age $29$, but some will, but workers with only $10$ years of education have no shot at those jobs.
\begin{align*}
\verywidehat{\vn{Earnings}} 
  = -\muse{12.12}{1.36} + \muse{2.37}{0.10} \times Education,
  \quad R^2 = 0.185,
  \quad \vn{SER} = 11.24
\end{align*}
\end{itemize}
\end{frame}
%%%%%%%%%%%%%%%%%%%%%%%%%%%%%%%%%%%%%%%%%%%%%%%%%%%%%%%%%


%%%%%%%%%%%%%%%%%%%%%%%%%%%%%%%%%%%%%%%%%%%%%%%%%%%%%%%%%
\begin{frame}
\frametitle{Heteroskedasticity: Application}
\begin{figure}
\centering
\includegraphics[width=\linewidth,height=0.75\textheight,keepaspectratio]%
{StockWatson4e-05-fig-03-Zoom}
\only<1->{\caption{\textbf{Hourly Earnings and Years of Education for $2731$ full-time $29$- to $30$-year-old workers in the United States, 2015.}}}
\end{figure}
\end{frame}
%%%%%%%%%%%%%%%%%%%%%%%%%%%%%%%%%%%%%%%%%%%%%%%%%%%%%%%%%