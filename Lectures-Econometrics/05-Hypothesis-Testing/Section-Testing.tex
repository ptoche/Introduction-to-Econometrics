

%%%%%%%%%%%%%%%%%%%%%%%%%%%%%%%%%%%%%%%%%%%%%%%%%%%%%%%%%
\begin{frame}
\frametitle{Recall: Hypothesis Test About Sample Mean}
\begin{itemize}
\item \emph{Two-Sided Test About $\mu$}
\begin{align*}
H_{0} \colon \exp[Y] & = \mu_{0}\\
H_{1} \colon \exp[Y] & \ne \mu_{0}
\end{align*}
\item \emph{Step 1:}
Compute $\SE(\mean{Y})$
\item \emph{Step 2:} 
Compute $t$-statistic
\begin{align*}
t_{0} = \frac{\mean{Y}-\mu_{0}}{\SE(\mean{Y})}
\end{align*}
\item \emph{Step 3 | $\alpha$ Variant:} 
Set significance level $\alpha$ and compute the critical $t$ value for that level:
\begin{align*}
|t_{0}| > t_{\alpha/2}
\implies 
\text{Reject $H_{0}$}
\end{align*}
\item \emph{Step 3 | $p$ Variant:} 
Compute $p$-value for a two-sided test
\begin{align*}
\text{$p$-value} = 2 \Phi(-|t_{0}|)
\to \text{small}
\implies 
\text{Reject $H_{0}$}
\end{align*}
The challenge is deciding whether values like $p\text{-value} \approx 5\%$ are ``small'' for your purpose.
\end{itemize}
\end{frame}
%%%%%%%%%%%%%%%%%%%%%%%%%%%%%%%%%%%%%%%%%%%%%%%%%%%%%%%%%


%%%%%%%%%%%%%%%%%%%%%%%%%%%%%%%%%%%%%%%%%%%%%%%%%%%%%%%%%
\begin{frame}[shrink=10]
\frametitle{Testing Hypotheses About Regression Coefficients}
\begin{itemize}
\item \emph{Two-Sided Test About $\beta_{1}$}
\begin{align*}
H_{0} \colon \beta_{1} & = \beta_{1,0}\\
H_{1} \colon \beta_{1} & \ne \beta_{1,0}
\end{align*}
\item \emph{Step 1:}
Compute the standard error 
\begin{emphalign*}
\SE(\hat{\beta}_{1})
\end{emphalign*}
\item \emph{Step 2:} 
Compute the $t$-statistic
\begin{emphalign*}
t_{0} = \frac{\hat{\beta}_{1}-\beta_{1,0}}{\SE(\hat{\beta}_{1})}
\end{emphalign*}
\item \emph{Step 3:} 
Compute the $p$-value
\begin{emphalign*}
\text{$p$-value} = 2 \Phi(-|t_{0}|)
\to \text{Is $p$-value small?}
\end{emphalign*}
$p$-value: Probability of sampling a value $\hat{\beta}_{1}$ at least as far from $\beta_{1,0}$ as our sample $\hat{\beta}_{1}$ actually is.
\end{itemize}
\end{frame}
%%%%%%%%%%%%%%%%%%%%%%%%%%%%%%%%%%%%%%%%%%%%%%%%%%%%%%%%%


%%%%%%%%%%%%%%%%%%%%%%%%%%%%%%%%%%%%%%%%%%%%%%%%%%%%%%%%%
\begin{frame}
\frametitle{Regression Equations Reporting}
\begin{itemize}
\item Regression results report the standard errors associated with each coefficient estimate. They are usually reported  in parentheses below each coefficient.
\begin{align*}
\verywidehat{\vn{TestScore}} 
  = \muse{698.9}{10.4} - \muse{2.28}{0.52} \times \vn{STR},
  \quad R^2 = 0.051,
  \quad \vn{SER} = 18.6
\end{align*}
The above report is equivalent to:
\begin{align*}
\verywidehat{\vn{TestScore}} & = \beta_{0} + \beta_{1} \times \vn{STR}\\
\hat{\beta}_{0} & = 698.9\\
\SE(\hat{\beta}_{0}) & = 10.4\\
\hat{\beta}_{1} & = -2.28\\
\SE(\hat{\beta}_{1}) & = 0.52
\end{align*}
\end{itemize}
\end{frame}
%%%%%%%%%%%%%%%%%%%%%%%%%%%%%%%%%%%%%%%%%%%%%%%%%%%%%%%%%


%%%%%%%%%%%%%%%%%%%%%%%%%%%%%%%%%%%%%%%%%%%%%%%%%%%%%%%%%
\begin{frame}
\frametitle{Testing Hypotheses}
\begin{itemize}
\item A very common desire is to test the significance of the regression coefficients:
\begin{align*}
H_{0} \colon \beta_{1} & = 0\\
H_{1} \colon \beta_{1} & \ne 0
\end{align*}
\item \emph{Step 1:} Read the standard error from the regression output.
\item \emph{Step 2:} Compute the $t$-statistic under the null:
\begin{align*}
t_{0} 
  = \frac{-2.28}{0.52}
  = -4.38
\end{align*}
\item \emph{Step 3 | $\alpha$ Variant:} 
Let $\alpha=0.05$ --- a good criterion for the social sciences, not so much for medical research! Compute the critical value or read it from a probability table. Since the sample size is large, we can approximate the Student-$t$ distribution with the standard normal distribution:
\begin{align*}
t_{\alpha/2} \approx 1.96
\end{align*}
\item Because $|t_{0} > t_{\alpha/2}|$, we reject the null hypothesis in favor of the two-sided alternative at the $\alpha=5\%$ significance level.
\end{itemize}
\end{frame}
%%%%%%%%%%%%%%%%%%%%%%%%%%%%%%%%%%%%%%%%%%%%%%%%%%%%%%%%%


%%%%%%%%%%%%%%%%%%%%%%%%%%%%%%%%%%%%%%%%%%%%%%%%%%%%%%%%
\begin{frame}[fragile,shrink=10]% lstlisting requires the fragile option
\frametitle{Critical Values: Standard Normal Distribution\flushbold{R code}}
\begin{itemize}
\item Compute the critical $z$-value for a two-sided test:
\begin{Rcode}
alpha = 0.05
qnorm(1-alpha/2)
## 1.959964
\end{Rcode}
\item Compute the critical $z$-value for a one-sided test:
\begin{Rcode}
alpha = 0.05
qnorm(1-alpha)
## 1.644854
\end{Rcode}
\item What do these compute?
\begin{Rcode}
qnorm(0.05)
## -1.644854

qnorm(0.05/2)
## -1.959964
\end{Rcode}
\end{itemize}
\end{frame}
%%%%%%%%%%%%%%%%%%%%%%%%%%%%%%%%%%%%%%%%%%%%%%%%%%%%%%%%


%%%%%%%%%%%%%%%%%%%%%%%%%%%%%%%%%%%%%%%%%%%%%%%%%%%%%%%%
\begin{frame}[fragile,shrink=10]% lstlisting requires the fragile option
\frametitle{Critical Values: Student-$t$ Distribution\flushbold{R code}}
\begin{itemize}
\item Compute the critical $t$-value for a two-sided test, with $10$ degrees of freedom:
\begin{Rcode}
alpha = 0.05
qt(1-alpha/2, df=10)
## 2.228139
\end{Rcode}
\item Compute the critical $t$-value for a one-sided test, with $10$ degrees of freedom:
\begin{Rcode}
alpha = 0.05
qt(1-alpha, df=10)
## 1.812461
\end{Rcode}
\item Note how the critical $t$-value is larger than the critical $z$-value.
\end{itemize}
\end{frame}
%%%%%%%%%%%%%%%%%%%%%%%%%%%%%%%%%%%%%%%%%%%%%%%%%%%%%%%%


%%%%%%%%%%%%%%%%%%%%%%%%%%%%%%%%%%%%%%%%%%%%%%%%%%%%%%%%
\begin{frame}[fragile,shrink=10]% lstlisting requires the fragile option
\frametitle{$p$-Values: Standard Normal Distribution\flushbold{R code}}
\begin{itemize}
\item Compute the $p$-value for a two-sided test:
\begin{Rcode}
2 * pnorm(-4.38)
## 1.186793e-05
\end{Rcode}
\item Compute the $p$-value for a one-sided test:
\begin{Rcode}
pnorm(-4.38)
## 5.933965e-06
\end{Rcode}
\item A $p$-value smaller than (say) $0.05$ suggests that if the null hypothesis is true, then our sample estimate is at least two standard deviations away from the hypothesized mean. 
\item A $p$-value smaller than (say) $0.05$ provides evidence against the null hypothesis.
\item A $p$-value smaller than (say) $0.0001$ provides even stronger evidence against the null hypothesis. 
\item Beware: This inference is valid if the estimated model satisfies all the conditions needed for inference. If one of these assumptions is violated, the $p$-value may no longer provide adequate guidance.
\item Note how the $p$-value for the one-sided test is smaller than for the two-sided test.
\end{itemize}
\end{frame}
%%%%%%%%%%%%%%%%%%%%%%%%%%%%%%%%%%%%%%%%%%%%%%%%%%%%%%%%


%%%%%%%%%%%%%%%%%%%%%%%%%%%%%%%%%%%%%%%%%%%%%%%%%%%%%%%%
\begin{frame}[fragile,shrink=10]% lstlisting requires the fragile option
\frametitle{$p$-Values: Student-$t$ Distribution\flushbold{R code}}
\begin{itemize}
\item Compute the $p$-value for a two-sided test, with $10$ degrees of freedom:
\begin{Rcode}
2 * pt(-4.38, df=10)
## 0.0013774
\end{Rcode}
\item Compute the $p$-value for a one-sided test, with $10$ degrees of freedom:
\begin{Rcode}
pt(-4.38, df=10)
## 0.0006887001
\end{Rcode}
\item Note how the $p$-value for the Student-$t$ distribution is larger than for the standard Normal distribution.
\end{itemize}
\end{frame}
%%%%%%%%%%%%%%%%%%%%%%%%%%%%%%%%%%%%%%%%%%%%%%%%%%%%%%%%


%%%%%%%%%%%%%%%%%%%%%%%%%%%%%%%%%%%%%%%%%%%%%%%%%%%%%%%%%
\begin{frame}
\frametitle{Understand the $p$-Value}
\begin{figure}
\centering
\includegraphics[width=\linewidth,height=0.8\textheight,keepaspectratio]%
{StockWatson4e-05-fig-01-Zoom}
\only<1->{\caption{\textbf{The $p$-value of a two-sided test for $t_{0}=-4.38$ is about $0.00001$.}}}
\end{figure}
\end{frame}
%%%%%%%%%%%%%%%%%%%%%%%%%%%%%%%%%%%%%%%%%%%%%%%%%%%%%%%%%

