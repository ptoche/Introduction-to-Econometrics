

%%%%%%%%%%%%%%%%%%%%%%%%%%%%%%%%%%%%%%%%%%%%%%%%%%%%%%%%%
\begin{frame}
\frametitle{Regression when $X$ is a Binary Variable}
\begin{itemize}
\item \emph{Binary Variable:}\\
A discrete variable that can take on only two possible values, e.g. $0$ and $1$. 
\item Examples: Male Vs Female. Boom Vs Recession. Employed Vs Unemployed. Democrat Vs Republican. 
\item Also called an indicator variable and/or a dummy variable.
\item \emph{Categorical Variable:}\\
A generization to several states. Example: African, American, Asian, European. Blood types: A, B, AB, O. Vaccination Status: Non-vaccinated, One dose, Two doses, Three doses.
\item Also called a dichotomous variable. 
\item In regression analysis, the presence of categorical variables changes the interpretation of the regression results, but does not change the computation of regression coefficients. 
\end{itemize}
\end{frame}
%%%%%%%%%%%%%%%%%%%%%%%%%%%%%%%%%%%%%%%%%%%%%%%%%%%%%%%%%


%%%%%%%%%%%%%%%%%%%%%%%%%%%%%%%%%%%%%%%%%%%%%%%%%%%%%%%%%
\begin{frame}
\frametitle{Interpreting Regression Coefficients}
\begin{itemize}
\item Let $\vn{STR}_{i}$ denote the student-teacher ratio in district $i$. 
Let $D_{i} \in \{0,1\}$ according to:
\begin{align*}
D_{i} = 
  \begin{cases}
  1 ~\text{if}~  \vn{STR}_{i} & < 20 \\
  0 ~\text{if}~  \vn{STR}_{i} & \geq 20 
  \end{cases}
\end{align*}
\item The population regression with $D_{i}$ as the regressor is:
\begin{align*}
Y_{i} = \beta_{0} + \beta_{1} D_{i} + u_{i}
\end{align*}
and is equivalent to:
\begin{align*}
Y_{i} & = \beta_{0} + u_{i} ~\text{if}~ D_{i} = 0\\
Y_{i} & = \beta_{0} + \beta_{1} + u_{i} ~\text{if}~ D_{i} = 1
\end{align*}
which implies $\exp[Y_{i}|D_{i}=1]=\beta_{0} + \beta_{1}$.
\item Because $\beta_{1}$ is the difference in the population means, the OLS estimator $\hat{\beta}_{1}$ is the difference between the sample averages of $Y_{i}$ in the two groups. 
\end{itemize}
\end{frame}
%%%%%%%%%%%%%%%%%%%%%%%%%%%%%%%%%%%%%%%%%%%%%%%%%%%%%%%%%


%%%%%%%%%%%%%%%%%%%%%%%%%%%%%%%%%%%%%%%%%%%%%%%%%%%%%%%%%
\begin{frame}
\frametitle{Hypothesis Tests \& Confidence Intervals}
\begin{itemize}
\item The null hypothesis that the two population means are the same can be tested against the alternative hypothesis that they differ by testing the null hypothesis $\beta_{1}=0$ against the alternative $\beta_{1}\ne0$. 
\item Example:In the regression of the test score against the student-teacher ratio binary variable $D_{i}$.
\begin{align*}
\verywidehat{\vn{TestScore}} 
  = \muse{650.0}{1.3} + \muse{7.4}{1.8} \times D,
  \quad R^2 = 0.037,
  \quad \vn{SER} = 18.7
\end{align*}
\item The average test score for the sub-sample with student-teacher ratios greater than or equal to $20$ ($D=0$) is $650.0$, and the average test score for the other sub-sample ($D=1$) is $650.0 + 7.4 = 657.4$. 
\item The difference between the sample average test scores for the two groups is $7.4$. 
\item Is the difference in the population mean test scores in the two groups statistically significantly different from $0$ at the $5\%$ level?
\begin{align*}
t = 7.4/1.8 = 4.04 > 1.96
\end{align*}
The null can be rejected at the $5\%$ level. 
\end{itemize}
\end{frame}
%%%%%%%%%%%%%%%%%%%%%%%%%%%%%%%%%%%%%%%%%%%%%%%%%%%%%%%%%

