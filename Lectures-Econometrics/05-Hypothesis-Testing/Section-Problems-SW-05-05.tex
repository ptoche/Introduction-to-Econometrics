

%%%%%%%%%%%%%%%%%%%%%%%%%%%%%%%%%%%%%%%%%%%%%%%%%%%%%%%%%
\begin{frame}
\frametitle{Problems and Applications}
\exercise{Stock \& Watson, Introduction (4th), Chapter~5, Exercise~5.}
In the 1980s, Tennessee conducted an experiment in which kindergarten students were randomly assigned to ``regular'' and ``small'' classes and given standardized tests at the end of the year. (Regular classes contained approximately $24$ students, and small classes contained approximately $15$ students.) Suppose, in the population, the standardized tests have a mean score of $925$ points and a standard deviation of $75$ points. Let $SmallClass$ denote a binary variable equal to $1$ if the student is assigned to a small class and equal to $0$ otherwise. A regression of $\vn{TestScore}$ on $SmallClass$ yields
\begin{align*}
\verywidehat{\vn{TestScore}} 
  = -\muse{918.0}{1.6} + \muse{13.9}{2.5} \times SmallClass,
  \quad R^2 = 0.01,
  \quad \vn{SER} = 74.6
\end{align*}
\vspace*{-2ex}
\begin{enumerate}
\item Do small classes improve test scores? By how much? Is the effect large? Explain.
\item Is the estimated effect of class size on test scores statistically significant? Carry out a test at the $5\%$ level.
\item Construct a $99\%$ confidence interval for the effect of $SmallClass$ on $\vn{TestScore}$.
\item Does least squares assumption $1$ plausibly hold for this regression? Explain.
\end{enumerate}
\end{frame}
%%%%%%%%%%%%%%%%%%%%%%%%%%%%%%%%%%%%%%%%%%%%%%%%%%%%%%%%%

