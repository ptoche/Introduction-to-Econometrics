

%%%%%%%%%%%%%%%%%%%%%%%%%%%%%%%%%%%%%%%%%%%%%%%%%%%%%%%%%
\begin{frame}
\frametitle{Population Regression Line}
\emph{Population Regression:}
\begin{itemize}
\item The population regression function in the multiple regression model is a model of the conditional expectation of $Y_{i}$:
\begin{align*}
\exp[Y_{i}|X_{1i}=x_{1}, X_{2i}=x_{2}] 
    = \beta_{0} + \beta_{1} x_{1} + \beta_{2} x_{2}
\end{align*}
\item \emph{Example:} Effect of class size in California districts: 
Multiple regression allows us to isolate the effect on test scores ($Y_{i}$) of the student-teacher ratio ($X_{1i}$), while holding other regressors constant, in particular the percentage of students in the district who are English learners ($X_{2i}$). 
\end{itemize}
\end{frame}
%%%%%%%%%%%%%%%%%%%%%%%%%%%%%%%%%%%%%%%%%%%%%%%%%%%%%%%%%


%%%%%%%%%%%%%%%%%%%%%%%%%%%%%%%%%%%%%%%%%%%%%%%%%%%%%%%%%
\begin{frame}
\frametitle{Population Regression Line}
\emph{Interpreting the slope coefficient:}
\begin{itemize}
\item The slope coefficient $\beta_{1}$ is the predicted difference in $Y$ between two observations with a unit difference in $X_{1}$, holding $X_{2}$ constant.
\item Consider a change in $X_{1i}$ holding $X_{2i}$ constant:
\begin{align*}
Y_{i} + \Delta\,Y_{i}
    & = \beta_{0} + \beta_{1} (X_{1i} + \Delta X_{1i}) + \beta_{2} X_{2i}\\
Y_{i} - \beta_{0} - \beta_{1} X_{1i} - \beta_{2} X_{2i}
    & = u_{i} = \beta_{1} \Delta X_{1i} - \Delta\,Y_{i}
\end{align*}
\item Calculate the expected value of $Y_{i}$ evaluated at $X_{1i}=x_{1}$ and $X_{2i}=x_{2}$:
\begin{align*}
\exp[Y_{i}|X_{1i} = x_{1}, X_{2i} = x_{2}] - \beta_{0} - \beta_{1} x_{1} - \beta_{2} x_{2} 
                & = \exp[u_{i}|X_{1i} = x_{1}, X_{2i} = x_{2}] = 0\\
    & = \beta_{1} \Delta X_{1} - \Delta\,Y \,\bigg|_{X_{1} = x_{1}, X_{2} = x_{2}}\\
\implies 
\beta_{1}
    & = \frac{\Delta Y}{\Delta\,X_{1}}\bigg|_{X_{1} = x_{1}, X_{2} = x_{2}}
\end{align*}
\end{itemize}
\end{frame}
%%%%%%%%%%%%%%%%%%%%%%%%%%%%%%%%%%%%%%%%%%%%%%%%%%%%%%%%%

