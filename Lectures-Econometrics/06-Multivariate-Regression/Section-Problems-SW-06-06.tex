

%%%%%%%%%%%%%%%%%%%%%%%%%%%%%%%%%%%%%%%%%%%%%%%%%%%%%%%%%
\begin{frame}
\frametitle{Problems and Applications}
\exercise{Stock \& Watson, Introduction (4th), Chapter~6, Exercise~6.}
A researcher plans to study the causal effect of police on crime, using data from a random sample of U.S. counties. He plans to regress the county's crime rate on the (per capita) size of the county's police force.
\begin{enumerate}
\item Explain why this regression is likely to suffer from omitted variable bias. Which variables would you add to the regression to control for important omitted variables?
\item Use the previous answer to determine whether the regression will likely over- or underestimate the effect of police on the crime rate. That is, do you think that $\hat{\beta}_{1}>\beta_{1}$ or $\hat{\beta}_{1}<\beta_{1}$?
\end{enumerate}
\end{frame}
%%%%%%%%%%%%%%%%%%%%%%%%%%%%%%%%%%%%%%%%%%%%%%%%%%%%%%%%%

