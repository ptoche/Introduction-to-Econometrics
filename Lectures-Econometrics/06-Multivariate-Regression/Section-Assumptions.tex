

%%%%%%%%%%%%%%%%%%%%%%%%%%%%%%%%%%%%%%%%%%%%%%%%%%%%%%%%%
\begin{frame}
\frametitle{Least Squares Assumptions}
\emph{Assumptions for causal inference:}
\begin{itemize}
\item Causal inference is possible only under a strict set of assumptions.
\begin{enumerate}
\item Errors have zero mean: $\exp[\hat{u}_{i}| X_{1i}=x_{1},\ldots,X_{ki}=x_{k}]=0$. \,
The conditional distribution of $u_{i}$ given $X_{1i},\ldots,X_{ki}$ has zero mean.
\item Regressors are i.i.d.: $X_{1i},\ldots,X_{ki} \sim \text{i.i.d.}$. \, Random sampling implies the regressors are independently and identically distributed.
\item Large outliers are unlikely: The regressand and regressors have finite kurtosis, $\exp[Y^{4}]<\infty, \exp[X_{j}^{4}]<\infty$.
\item No perfect multicollinearity among the regressors. 
\end{enumerate}
\item Under these assumptions, the least-squares estimators $\hat{\beta}_{0},\ldots,\hat{\beta}_{k}$ are jointly normally distributed with
\begin{align*}
\hat{\beta}_{j} \sim \N(\beta_{j}, \sigma_{\beta_{j}}^{2}), j=0,\ldots,k.
\end{align*}
\end{itemize}
\end{frame}
%%%%%%%%%%%%%%%%%%%%%%%%%%%%%%%%%%%%%%%%%%%%%%%%%%%%%%%%%
