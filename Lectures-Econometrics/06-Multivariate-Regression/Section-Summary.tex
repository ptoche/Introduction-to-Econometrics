

%%%%%%%%%%%%%%%%%%%%%%%%%%%%%%%%%%%%%%%%%%%%%%%%%%%%%%%%%
\begin{frame}
\frametitle{Summary}
\begin{itemize}
\item Omitted variable bias occurs when an omitted variable (a) is correlated with an included regressor and (b) is a determinant of $Y$.
\item The multiple regression model is a linear regression model that includes multiple regressors, $X_1,X_2,\ldots,X_k$. Associated with each regressor is a regression coefficient, $\beta_1,\beta_2,\ldots,\beta_k$. The coefficient $\beta_1$ is the expected difference in $Y$ associated with a one-unit difference in $X_1$, holding the other regressors constant.
\item The coefficients in multiple regression can be estimated by OLS. When the Gauss-Markov assumptions hold, the OLS estimators of the causal effect are unbiased, consistent, and normally distributed in large samples.
\item The role of control variables is to hold constant omitted factors so that the variable of interest is no longer correlated with the error term.
\item Perfect multicollinearity occurs when one regressor is an exact linear function of the other regressors.
\item The standard error of the regression, the $R^2$, and the $\bar{R}^2$ are measures of fit for the multiple regression model.
\end{itemize}
\end{frame}
%%%%%%%%%%%%%%%%%%%%%%%%%%%%%%%%%%%%%%%%%%%%%%%%%%%%%%%%%


