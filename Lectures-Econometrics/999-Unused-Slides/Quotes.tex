Goodhart's law is an adage often stated as "When a measure becomes a target, it ceases to be a good measure"

Campbell's law:
The more any quantitative social indicator is used for social decision-making, the more subject it will be to corruption pressures and the more apt it will be to distort and corrupt the social processes it is intended to monitor.

The term cobra effect was coined by economist Horst Siebert based on an anecdote of a (possibly ahistorical) occurrence in India during British rule.[2][3][4] The British government, concerned about the number of venomous cobras in Delhi, offered a bounty for every dead cobra. Initially, this was a successful strategy; large numbers of snakes were killed for the reward. Eventually, however, enterprising people began to breed cobras for the income. When the government became aware of this, the reward program was scrapped. When cobra breeders set their now-worthless snakes free, the wild cobra population further increased.