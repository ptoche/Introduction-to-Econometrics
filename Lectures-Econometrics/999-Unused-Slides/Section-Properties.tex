

\section{Properties}


%%%%%%%%%%%%%%%%%%%%%%%%%%%%%%%%%%%%%%%%%%%%%%%%%%%%%%%%%%
\begin{frame}
\frametitle{Properties of probability}
\begin{itemize}
\item $\Pr(\emptyset) = 0$ 
Derivation: $S$ and $\emptyset$ are disjoint,
so $\Pr(S \cup \emptyset) = \Pr(S) + \Pr(\emptyset) = 1 +
\Pr(\emptyset)$. Also, $S \cup \emptyset = S$, so $\Pr(S \cup \emptyset) = \Pr(S) = 1$. Combining, $1 = 1 + \Pr(\emptyset)$, so $\Pr(\emptyset) = 0$
\item $\Pr(A^c) = 1 - \Pr(A)$
\item $A \subseteq B$ implies $\Pr(A) \leq \Pr(B)$
$A \subseteq B$ implies $B = (B \setminus A) \cup A$. $B
\setminus A)$ and $A$ are disjoint.
\item $\Pr(A \cup B) = \Pr(A) + \Pr(B) - \Pr(A \cap B)$ 
$A \cap B$ is read ``the intersection of $A$ and $B$'' or ``$A$ and $B$.'' It is the event that both $A$ and $B$ occur. 
\end{itemize}
\end{frame}
%%%%%%%%%%%%%%%%%%%%%%%%%%%%%%%%%%%%%%%%%%%%%%%%%%%%%%%%%%  

