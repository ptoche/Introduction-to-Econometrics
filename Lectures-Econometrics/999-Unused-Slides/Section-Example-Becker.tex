

\subsection{Becker Crime Model}


%%%%%%%%%%%%%%%%%%%%%%%%%%%%%%%%%%%%%%%%%%%%%%%%%%%%%%%%%
\begin{frame}
  \frametitle{Example: Becker's (1968) economic model of crime}
  \begin{itemize}
  \item Economic Model: we have the relation,
     \begin{align*}
      Y &= f(X_1, X_2, X_3, X_4, ...)
       \end{align*}
    \begin{itemize}
    \item $Y$ : hours spent in criminal activities (crime)
    \item $X_1$: wage for an hour spent in criminal activities
    \item $X_2$:  wage in legal employment (wagem)
    \item $X_3$: income other than from crime or employment (othinc)
    \item $X_4$: probability of getting caught (freqarr)
    \end{itemize}
  \item An standard  econometric specification is:
     \begin{align*} 
     \mathrm{crime} &= \beta_0 + \beta_1 \mathrm{wagem} + \beta_2
    \mathrm{othinc} + \beta_3 \mathrm{freqarr} + U
    \end{align*}
    Where the term $U$ captures unobserved factors such as:
    \begin{itemize}
    \item the reward for criminal activity, 
    \item  family background, 
    \item measurement error
    \end{itemize}
  \end{itemize}
\end{frame}
%%%%%%%%%%%%%%%%%%%%%%%%%%%%%%%%%%%%%%%%%%%%%%%%%%%%%%%%%

