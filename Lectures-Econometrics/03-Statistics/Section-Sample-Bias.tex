

%%%%%%%%%%%%%%%%%%%%%%%%%%%%%%%%%%%%%%%%%%%%%%%%%%%%%%%%%
\begin{frame}
\frametitle{Alf Landon Wins!}
\begin{minipage}{0.38\linewidth}
\begin{figure}
\centering
\includegraphics[width=\linewidth,height=0.8\textheight,keepaspectratio]%
{Photo-Alf-Landon-1936}
\end{figure}
\end{minipage}\hfill%
\begin{minipage}{0.6\linewidth}
The Literary Digest had correctly predicted the outcomes of the 1916, 1920, 1924, 1928, and 1932 elections by conducting polls. These polls were popular. In one of the largest surveys ever, the Literary Digest surveyed more than $2,000,000$ readers --- a response rate of about $25\%$ --- and predicted Republican presidential candidate Alfred Landon would win $57$ percent of the popular vote and $370$ electoral votes. Instead, he lost the popular vote by more than $10$ million votes and carried only two states, for a total of eight electoral votes to Roosevelt's $523$. The problem was that the mailing had targeted people more likely to vote Republican. By contrast, the American Institute of Public Opinion, founded by George Gallup, set quotas for the numbers of individuals needed for each type of respondent and by polling ``only'' $50,000$ people made an accurate prediction. Gallup became a household name. 
\end{minipage}
\end{frame}
%%%%%%%%%%%%%%%%%%%%%%%%%%%%%%%%%%%%%%%%%%%%%%%%%%%%%%%%%


