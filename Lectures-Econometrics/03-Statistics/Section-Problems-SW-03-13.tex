

%%%%%%%%%%%%%%%%%%%%%%%%%%%%%%%%%%%%%%%%%%%%%%%%%%%%%%%%%
\begin{frame}
\frametitle{Problems and Applications}
\exercise{Stock \& Watson, Introduction (4th), Chapter~3, Exercise~13.}
Data on fifth-grade test scores (reading and mathematics) for $420$ school districts in California yield average score $\mean{Y}=654.2$ and standard deviation $s_{Y}=19.1$.
\begin{enumerate}
\item Construct a $95\%$ confidence interval for the mean test score in the population.
\item When the districts were divided into those with small classes ($< 20$ students per teacher) and those with large classes ($\geq 20$ students per teacher), the following results were found:
\begin{center}
\begin{tabular}{lccc}
\toprule
Class Size & Average Score ($\mean{Y}$)      
                     & Standard Deviation ($s_{Y}$)
                               &   $n$\\
\midrule
Small      & $657.4$ &  $19.4$ & $238$\\
Large      & $650.0$ &  $17.9$ & $182$\\
\bottomrule
\end{tabular}
\end{center}
Is there statistically significant evidence that the districts with smaller classes have higher average test scores? Explain.
\end{enumerate}
\end{frame}
%%%%%%%%%%%%%%%%%%%%%%%%%%%%%%%%%%%%%%%%%%%%%%%%%%%%%%%%%
