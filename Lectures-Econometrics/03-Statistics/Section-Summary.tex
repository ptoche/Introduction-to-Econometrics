

%%%%%%%%%%%%%%%%%%%%%%%%%%%%%%%%%%%%%%%%%%%%%%%%%%%%%%%%%
\begin{frame}
\frametitle{Summary}
\begin{enumerate}
\item The sample mean $\mean{Y}$ is an estimator of the population mean $\mu_{Y}$. If the sample observations $Y_1,\ldots,Y_n$ are i.i.d.,
\begin{enumerate}
\item The sampling distribution of $\mean{Y}$ has mean $\mu_{Y}$ and variance $\sigma^2_{\mean{Y}}/n$.
\item The sample mean $\mean{Y}$ is unbiased.
\item \textit{Law of Large Numbers (LLN):} As the sample size $n$ is increased, the sample $\mean{Y}$ tends to the population mean $\mu_{Y}$ --- The sample mean is a consistent estimator of the population mean.
\item \textit{Central Limit Theorem (CLT):} For large sample sizes $n$, the sample mean $\mean{Y}$ has an approximately normal sampling distribution.
\end{enumerate}
\item The $t$-statistic is used to test the null hypothesis that the population mean takes on a particular value. If $n$ is large, the $t$-statistic has a standard normal sampling distribution when the null hypothesis is true.
\end{enumerate}
\end{frame}
%%%%%%%%%%%%%%%%%%%%%%%%%%%%%%%%%%%%%%%%%%%%%%%%%%%%%%%%%


%%%%%%%%%%%%%%%%%%%%%%%%%%%%%%%%%%%%%%%%%%%%%%%%%%%%%%%%%
\begin{frame}
\frametitle{Summary}
\begin{enumerate}\setcounter{enumi}{2}
\item The $t$-statistic can be used to calculate the $p$-value associated with the null hypothesis. The $p$-value is the probability of drawing a statistic at least as adverse to the null hypothesis as the one you actually computed in your sample, assuming the null hypothesis is correct. A small $p$-value is evidence that the null hypothesis is false.
\item A $95\%$ confidence interval for $\mean{Y}$ is an interval constructed so that it contains the true value of $\mean{Y}$ in $95\%$ of all possible samples.
\item Hypothesis tests and confidence intervals for the difference in the means of two populations are conceptually similar to tests and intervals for the mean of a single population.
\item The sample correlation coefficient is an estimator of the population correlation coefficient and measures the linear relationship between two variables -- that is, how well their scatterplot is approximated by a straight line.
\end{enumerate}
\end{frame}
%%%%%%%%%%%%%%%%%%%%%%%%%%%%%%%%%%%%%%%%%%%%%%%%%%%%%%%%%
