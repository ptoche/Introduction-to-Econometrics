

%%%%%%%%%%%%%%%%%%%%%%%%%%%%%%%%%%%%%%%%%%%%%%%%%%%%%%%%%
\begin{frame}
\frametitle{Terminology}
\begin{itemize}
\item \emph{Type-I Errors:} 
Incorrectly reject the null hypothesis even though it is true.
\item \emph{Type-II Errors:} 
Incorrectly fail to reject the null hypothesis even though it is false.
\item \emph{Significance level:} If you set a threshold probability for your tolerance to Type-I error (rejecting a correct null), say a threshold of $5\%$, then you will reject the null hypothesis if, and only if, the $p$-value is less than the significance level $0.05$. Against a two-sided alternative, the simple rule follows:
\begin{align*}
\text{reject}~ H_0 ~\text{if}~ |t^{\text{act}}| > 1.96
\end{align*}
where $1.96$ is the \emph{critical value} for a \emph{two-sided test}. 
\item \emph{Rejection region:} The set of values of the test statistic for which the test rejects the null hypothesis.
\item \emph{Acceptance region:} The set of values of the test statistic for which it does not reject the null hypothesis. 
\item \emph{Size of the test:} The probability that the test actually incorrectly rejects the null hypothesis when it is true.
\item \emph{Power of the test:} The probability that the test correctly rejects the null hypothesis when the alternative is true.
\end{itemize}
\end{frame}
%%%%%%%%%%%%%%%%%%%%%%%%%%%%%%%%%%%%%%%%%%%%%%%%%%%%%%%%%
