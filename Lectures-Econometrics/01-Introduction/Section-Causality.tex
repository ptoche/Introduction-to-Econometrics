

%%%%%%%%%%%%%%%%%%%%%%%%%%%%%%%%%%%%%%%%%%%%%%%%%%%%%%%%%
\begin{frame}
\frametitle{Data and causal effects}
\begin{itemize}
\item Economists are interested in causal relations.
\item Statistics establishes correlations.
\item And correlation is not causation.
\item In order to establish that one variable has a causal effect on another, other factors affecting the outcome must be held fixed.
\item In natural sciences, scientists can use controlled experiments.
\item Experiment are often impossible in economics (too costly and/or for ethical reasons)
\item In chemistry,  aerodynamics, computer science, experiences are the norm. In economics --- however tempted some may be to do it --- you cannot remove teachers from schools to test how that would affect students' test scores. 
\item Economists must rely on observational data --- data that originates from the real world, with all its constraints and statistical ``noise''. 
\end{itemize}
\end{frame}
%%%%%%%%%%%%%%%%%%%%%%%%%%%%%%%%%%%%%%%%%%%%%%%%%%%%%%%%%

