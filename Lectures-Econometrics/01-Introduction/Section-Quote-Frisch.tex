

%%%%%%%%%%%%%%%%%%%%%%%%%%%%%%%%%%%%%%%%%%%%%%%%%%%%%%%%%
\begin{frame}<beamer>
\framedquote{birth of econometrics}{%
  \begin{quote}
  \vspace{-2ex}
  \justify
  Econometrics is by no means the same as economic statistics. Nor is it identical with what we call general economic theory, although a considerable portion of this theory has a definitely quantitative character. Nor should econometrics be taken as synonymous with the application of mathematics to economics. Experience has shown that each of these three view-points, that of statistics, economic theory, and mathematics, is a necessary, but not by itself a sufficient, condition for a real understanding of the quantitative relations in modern economic life. It is the unification of all three that is powerful. And it is this unification that constitutes econometrics.
  \smallskip\newline 
  \textbf{Ragnar Frisch}, in the first issue of \textit{Econometrica}, 1933.
  \end{quote}
}%
\end{frame}
%%%%%%%%%%%%%%%%%%%%%%%%%%%%%%%%%%%%%%%%%%%%%%%%%%%%%%%%%