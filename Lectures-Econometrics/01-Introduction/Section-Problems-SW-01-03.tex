

%%%%%%%%%%%%%%%%%%%%%%%%%%%%%%%%%%%%%%%%%%%%%%%%%%%%%%%%%
\begin{frame}[shrink=5]
\frametitle{Problems and Applications}
\exercise{Stock \& Watson, Introduction (4th), Chapter~1, Exercise~3.}
You are asked to study the causal effect of hours spent on employee training (measured in hours per worker per week) in a manufacturing plant on the productivity of its workers (output per worker per hour). Describe:
\begin{itemize}
\item an ideal randomized controlled experiment to measure this causal effect;
\item an observational cross-sectional data set with which you could study this
effect;
\item an observational time series data set for studying this effect; and
\item an observational panel data set for studying this effect.
\end{itemize}
\pause
\exercise{Get Data, Compute, Plot.}
\pause
Get quarterly data on the GDP for the United States, going back to 1960 and up to the latest available observation. 
\begin{enumerate}
\item Compute the growth rate for each quarter. 
\item Compute the annualized growth rate for each quarter. 
\item Plot the two series on the same axes in blue and red.
\item Write a proper citation for the data source.
\item Describe the data you have selected.
\item What software did you use? 
\item What computation did you make to obtain an annualized growth rate?
\item What value do you get for 1960:Q3? 
\item What was the biggest challenge?
\end{enumerate}
\end{frame}
%%%%%%%%%%%%%%%%%%%%%%%%%%%%%%%%%%%%%%%%%%%%%%%%%%%%%%%%%

