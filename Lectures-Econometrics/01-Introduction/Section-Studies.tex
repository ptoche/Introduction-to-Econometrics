

%%%%%%%%%%%%%%%%%%%%%%%%%%%%%%%%%%%%%%%%%%%%%%%%%%%%%%%%%
\begin{frame}
\frametitle{Famous Empirical Studies}
\begin{itemize}
\item \emph{Return to education:} 
\begin{align*}
\log(\text{Wage}) 
  = \alpha + \beta \cdot \text{Years of Schooling} + U
\end{align*}
The main challenge with this class of regression model is the ``\emph{omitted variable bias}'': Ignoring the effect of systematic factors such as ability can cause the regression model to overestimate the effect of education on wages. This model is often called a ``Mincer regression'', named after Jacob Mincer. 
\item \emph{Effect of minimum wage and unemployment:} 
\begin{align*}
\text{Unemployment} 
  = \alpha + \beta \cdot \text{Minium Wage} + U 
\end{align*}
The main challenge with this class of regression model is ``\emph{reverse causality}``: High employment may lead to political pressure to raise the minimum wage. In other words, there is a two-way causality. This model is associated with the work of David Card and Alan Krueger. 
\item In these regression models, $U$ is a random catch-all term. The more $U$ is distributed like a mean-zero normal variate, the easier it is to make inferences.
\end{itemize}
\end{frame}
%%%%%%%%%%%%%%%%%%%%%%%%%%%%%%%%%%%%%%%%%%%%%%%%%%%%%%%%%


%%%%%%%%%%%%%%%%%%%%%%%%%%%%%%%%%%%%%%%%%%%%%%%%%%%%%%%%%
\begin{frame}
\frametitle{Famous Empirical Studies}
\begin{itemize}
\item \emph{Effect of policing on crime:} 
\begin{align*}
\text{Number of Crimes} 
  = \alpha + \beta \cdot \text{Size of Police Force} + U
\end{align*}
The challenge with this class of regression model is ``\emph{spurious correlation}'':
Cities with a lot of criminal activity have a bigger police force. The correlation can spuriously indicate that the size of the police force has a positive effect on the crime rate. 
\item \emph{Impact of MTV show ``16 and Pregnant'' on teen pregnancy:}
\begin{align*}
\log(\text{Teen Pregnancy})
  = \alpha + \beta \cdot \text{Show's Rating Among Teens} + U
\end{align*}
The challenge with this class of regression model is ``\emph{self selection}'': 
Teens who would be adverse to getting pregnant could be more likely to watch the show. See the 2015 article ``Media Influences on Social Outcomes: The Impact of MTV's 16 and Pregnant on Teen Childbearing'' by Melissa S. Kearney and Phillip B. Levine.
\end{itemize}
\end{frame}
%%%%%%%%%%%%%%%%%%%%%%%%%%%%%%%%%%%%%%%%%%%%%%%%%%%%%%%%%

