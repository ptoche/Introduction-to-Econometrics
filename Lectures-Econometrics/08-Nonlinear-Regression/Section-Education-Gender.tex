

%%%%%%%%%%%%%%%%%%%%%%%%%%%%%%%%%%%%%%%%%%%%%%%%%%%%%%%%%
\begin{frame}
\frametitle{Return to Education and Gender Gap}
\emph{The Return to Education revisited:}
\begin{itemize}
\item A multiple regression analysis that controls for determinants of earnings that, if omitted, could cause omitted variable bias, and that uses a nonlinear functional form. 
\item The next Table summarizes regressions estimated using data on $47,233$ full-time workers, ages $30$ through $64$, from the Current Population Survey (CPS).
\item The dependent variable is the logarithm of hourly earnings, so an additional year of education is associated with a constant percentage increase in earnings --- not a dollar increase.
\item The estimated economic return to education in regression (4) is $11.14\%$ for each year of education for men and 
$0.1114+0.0082=11.96\%$ for women. 
\item Because the regression functions for men and women have different slopes, the gender gap depends on the years of education. 
\begin{itemize}
\item For $12$ years of education, the gender gap is $27.0\%$ ~ ($0.0082\times12-0.368$). 
\item for $16$ years of education, the gender gap is $23.7\%$ ~ ($0.0082\times16-0.368$).
% 0.0082 * 12-0.368
% -0.2696
% 0.0082 * 16-0.368
% -0.2368
\end{itemize}
\item Labor economists estimate the return to education between $8\%$ and $11\%$, depending on the quality of the education received. 
\end{itemize}
\end{frame}
%%%%%%%%%%%%%%%%%%%%%%%%%%%%%%%%%%%%%%%%%%%%%%%%%%%%%%%%%


%%%%%%%%%%%%%%%%%%%%%%%%%%%%%%%%%%%%%%%%%%%%%%%%%%%%%%%%%
\begin{frame}
\frametitle{Return to Education and Gender Gap: United States, 2015}
\begin{figure}
\centering
\includegraphics[width=\linewidth,height=0.8\textheight,keepaspectratio]%
{StockWatson4e-08-tbl-01-Zoom-Cropped}
\end{figure}
\end{frame}
%%%%%%%%%%%%%%%%%%%%%%%%%%%%%%%%%%%%%%%%%%%%%%%%%%%%%%%%%


%%%%%%%%%%%%%%%%%%%%%%%%%%%%%%%%%%%%%%%%%%%%%%%%%%%%%%%%%
\begin{frame}
\frametitle{Return to Education and Gender Gap}
\emph{The Gender Gap revisited:}
\begin{enumerate}
\item The omission of sex in regression (1) does not result in substantial omitted variable bias: Even though sex enters regression (2) significantly and with a large coefficient, sex and years of education are nearly uncorrelated: On average, men and women have nearly the same levels of education. 
\item The returns to education are economically and statistically significantly different for men and women: In regression (3), the $t$-statistic testing the hypothesis that they are the same is $3.42$. The confidence interval is tight: the return to education is precisely estimated both for men and for women.
\item Regression (4) controls for the region of the country in which the individual lives, to address potential omitted variable bias that might arise if years of education differ systematically by region. Controlling for region makes a small difference to the estimated coefficients on the education terms relative to those reported in regression (3). 
\item Regression (4) controls for the potential experience of the worker, as measured by years since completion of schooling. The estimated coefficients imply a declining marginal value for each year of potential experience.
\end{enumerate}
\end{frame}
%%%%%%%%%%%%%%%%%%%%%%%%%%%%%%%%%%%%%%%%%%%%%%%%%%%%%%%%%

