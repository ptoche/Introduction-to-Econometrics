

%%%%%%%%%%%%%%%%%%%%%%%%%%%%%%%%%%%%%%%%%%%%%%%%%%%%%%%%%
\begin{frame}
\frametitle{The Normal Distribution} 
\begin{emphbox}{Normal Distribution}\justifying
A random variable $X$ follows a normal distribution with mean $\mu$ and variance $\sigma^2$ if it is continuously distributed with probability density function (pdf) given by:
\begin{align*}
f_X(x) = \frac{1}{\sqrt{2\pi\sigma^2}} e^{-\dfrac{(x-\mu)^2}{2\sigma^2}}
\end{align*} 
It is denoted $X\sim N(\mu,\sigma^2)$. 
\end{emphbox}
\begin{itemize}
\item \emph{Standard normal distribution:} 
$Z\sim N(0,1)$. That is, $\mu=0$ and $\sigma^2=1$.
\end{itemize}
\end{frame}
%%%%%%%%%%%%%%%%%%%%%%%%%%%%%%%%%%%%%%%%%%%%%%%%%%%%%%%%%


%%%%%%%%%%%%%%%%%%%%%%%%%%%%%%%%%%%%%%%%%%%%%%%%%%%%%%%%%
\begin{frame}
\frametitle{The Normal Distribution} 
\begin{figure}
\centering
\includegraphics[width=\linewidth,height=0.8\textheight,keepaspectratio]%
{distribution-normal}	
\end{figure}
\end{frame}
%%%%%%%%%%%%%%%%%%%%%%%%%%%%%%%%%%%%%%%%%%%%%%%%%%%%%%%%%


%%%%%%%%%%%%%%%%%%%%%%%%%%%%%%%%%%%%%%%%%%%%%%%%%%%%%%%%%
\begin{frame}
\frametitle{Properties of Normal Distribution} 
\emph{Property 1:}
\begin{itemize}
\item If $X\sim N(\mu,\sigma^2)$ then $(X-\mu)/\sigma\sim N(0,1)$.
\item Thus, we can express probabilities for any normal random variable in terms of $Z\sim N(0,1)$.
\item Exercise: Show that if $X\sim N(2,100)$ then $\Pr(X\geq 22) = \Pr(Z\geq 2)$.
\begin{align*}
\Pr(X\geq 22) 
    & = \Pr\left(\frac{X-2}{10} \geq \frac{22 -2}{10} \right)\\ 
    & = \Pr\left(Z \geq 2\right) 
\end{align*} 
Here we use the fact that $\mu=2$ and $\sigma = \sqrt{100} = 10$.
\item We calculate $\Pr(Z\geq 2)$ by looking up a table or using dedicated software.
\end{itemize}
\end{frame}
%%%%%%%%%%%%%%%%%%%%%%%%%%%%%%%%%%%%%%%%%%%%%%%%%%%%%%%%%


%%%%%%%%%%%%%%%%%%%%%%%%%%%%%%%%%%%%%%%%%%%%%%%%%%%%%%%%%
\begin{frame}
\frametitle{Properties of Normal Distribution} 
\emph{Property 2:}
\begin{itemize}
\item If $X\sim N(\mu_X,\sigma_X^2)$ and $Y\sim N(\mu_Y,\sigma_Y^2)$ are jointly normal, then $W=aX+bY$ is also normally distributed for any $a,b\in\Real$.
\item Calculate $\exp[W]$:
\begin{itemize}
\item By linearity of expectation we have that $\exp[aX+bY] = a\exp[X]+ b\exp[Y]$
\item So, $\mu_W = \exp[W] = a\mu_X + b\mu_Y$.
\end{itemize}
\item Calculate $\var(W)$:
\begin{align*}
\var(aX+bY) = a^2\var(X) + b^2\var(Y) + 2ab\cov(X,Y)\\
\implies 
\var(W) = \var(aX+bY) = a^2\sigma_X^2 + b^2\sigma_Y^2 + 2ab\sigma_{XY}
\end{align*}

\item Putting it together:
\begin{align*}
W\sim N\left(a\mu_X + b\mu_Y, a^2\sigma_X^2 + b^2\sigma_Y^2 + 2ab\sigma_{XY}\right)
\end{align*} 
\end{itemize}
\end{frame}
%%%%%%%%%%%%%%%%%%%%%%%%%%%%%%%%%%%%%%%%%%%%%%%%%%%%%%%%%


%%%%%%%%%%%%%%%%%%%%%%%%%%%%%%%%%%%%%%%%%%%%%%%%%%%%%%%%%
\begin{frame}
\frametitle{Properties of Normal Distribution} 
\emph{Property 3:} 
\begin{itemize}
\item The distribution of $X \sim N(0,\sigma^2)$ is symmetric around zero: 
\begin{align*}
\Pr(X \geq x) = \Pr(X \leq -x)
\end{align*} 
\item Thus, we can compute:
\begin{align*}
\Pr(|X|\geq c) = \Pr(X\geq c) + \Pr(X \leq -c) = 2\Pr(X\geq c)
\end{align*} 
\begin{figure}
\centering
\includegraphics<2->[width=\linewidth,height=0.45\textheight,keepaspectratio]%
{dist-normal-m-0-s-i-combined}
\end{figure}
\end{itemize}
\end{frame}
%%%%%%%%%%%%%%%%%%%%%%%%%%%%%%%%%%%%%%%%%%%%%%%%%%%%%%%%%

