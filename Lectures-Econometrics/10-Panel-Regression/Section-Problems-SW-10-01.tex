

%%%%%%%%%%%%%%%%%%%%%%%%%%%%%%%%%%%%%%%%%%%%%%%%%%%%%%%%%
\begin{frame}
\frametitle{Problems and Applications}
\exercise{Stock \& Watson, Introduction (4th), Chapter~10, Exercise~1.}
This exercise refers to the drunk driving panel data regressions summarized in Table 10.1.
\begin{enumerate}
\item New Jersey has a population of $8.1$ million people. Suppose New Jersey increased the tax on a case of beer by $\$1$ (in 1988 dollars). Use the results in column (4) to predict the number of lives that would be saved over the next year. Construct a $95\%$ confidence interval for your answer.
\item The drinking age in New Jersey is $21$. Suppose New Jersey lowered its drinking age to $18$. Use the results in column (4) to predict the change in the number of traffic fatalities in the next year. Construct a $95\%$ confidence interval for your answer.
\item Should time effects be included in the regression? Why or why not?
\item A researcher conjectures that the unemployment rate has a different effect on traffic fatalities in the western states than in the other states. How would you test this hypothesis? (Be specific about the specification of the regression and the statistical test you would use.)
\end{enumerate}
\end{frame}
%%%%%%%%%%%%%%%%%%%%%%%%%%%%%%%%%%%%%%%%%%%%%%%%%%%%%%%%%

