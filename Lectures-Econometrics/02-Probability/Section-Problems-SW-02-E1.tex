

%%%%%%%%%%%%%%%%%%%%%%%%%%%%%%%%%%%%%%%%%%%%%%%%%%%%%%%%%
\begin{frame}
\frametitle{Problems and Applications}
\exercise{Stock \& Watson, Introduction (4th), Chapter~2, Empirical Exercise~1.}
The spreadsheet \textbf{Age\_HourlyEarnings} contains the joint distribution of age (Age) and average hourly earnings ($\vn{AHE}$) for 25- to 34-year-old full-time workers in 2015 with an education level that exceeds a high school diploma. Use this joint distribution to carry out the following exercises.
\begin{enumerate}
\item Compute the marginal distribution of Age.
\item Compute the mean of $\vn{AHE}$ for each value of Age; that is, compute,
E(AHE|Age = 25), and so forth.
\item Compute and plot the mean of $\vn{AHE}$ versus Age. Are average hourly earnings and age related? Explain.
\item Use the law of iterated expectations to compute the mean of AHE; that is, compute E(AHE).
\item Compute the variance of $\vn{AHE}$.
\item Compute the covariance between $\vn{AHE}$ and Age.
\item Compute the correlation between $\vn{AHE}$ and Age.
\item Relate your answers in (f) and (g) to the plot you constructed in (c).
\end{enumerate}
\end{frame}
%%%%%%%%%%%%%%%%%%%%%%%%%%%%%%%%%%%%%%%%%%%%%%%%%%%%%%%%%
