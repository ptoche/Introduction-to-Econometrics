

%%%%%%%%%%%%%%%%%%%%%%%%%%%%%%%%%%%%%%%%%%%%%%%%%%%%%%%%%
\begin{frame}[shrink=7]
\frametitle{Problems and Applications}
\exercise{Stock \& Watson, Introduction (4th), Chapter~2, Exercise~6.}
The following table gives the joint probability distribution between employment status and college graduation among those either employed or looking for work (unemployed) in the working-age U.S. population for September 2017.
\smallskip
\begin{footnotesize}
% {StockWatson4e-02-tbl-A1}
\begin{center}
%\newcolumntype{C}{>{$}c<{$}}% @{\extracolsep{\fill}}
\newcolumntype{C}{>{\centering\arraybackslash}wc{0.16\linewidth}}% 
\begin{tabular*}{\linewidth}{p{0.4\linewidth}@{\extracolsep{\fill}}CCC} 
\multicolumn{4}{l}{\textbf{Employment \& College Graduation} (Population aged 25 and above, September 2017)}\\
\toprule 
    & \text{\makecell[c]{Unemployed \\ $Y=0$}} 
             & \text{\makecell[c]{Employed \\ $Y=1$}} 
                     & \text{\makecell[c]{\\ Total}} \\
\cmidrule{2-4}
\text{Non-College Graduates ($X=0$)}
    &  0.026 & 0.576 & 0.602 \\
\text{College Graduates ($X=1$)}
    &  0.009 & 0.389 & 0.398 \\
\text{Total}
    &  0.035 & 0.965 & 1.000 \\
\bottomrule
\end{tabular*}
\end{center}\medskip
\end{footnotesize}
\begin{enumerate}
\item Compute $E(Y)$.
\item The unemployment rate is the fraction of the labor force that is unemployed. Show that the unemployment rate is given by $1-E(Y)$.
\item Calculate $E(Y|X=1)$ and $E(Y|X=0)$.
\item Calculate the unemployment rate for (i) college graduates and (ii) non-college graduates.
\item A randomly selected member of this population reports being unemployed. What is the probability that this worker is a college graduate? A non-college graduate?
\item Are educational achievement and employment status independent? Explain.
\end{enumerate}
\end{frame}
%%%%%%%%%%%%%%%%%%%%%%%%%%%%%%%%%%%%%%%%%%%%%%%%%%%%%%%%%

