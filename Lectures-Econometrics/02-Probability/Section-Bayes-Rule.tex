

%%%%%%%%%%%%%%%%%%%%%%%%%%%%%%%%%%%%%%%%%%%%%%%%%%%%%%%%%%
\begin{frame}
\frametitle{Bayes' Rule}
\begin{itemize}
\item \emph{Bayes' Rule:}
\begin{emphalign*}
\Pr(A | B ) = \frac{\Pr (B | A) \cdot \Pr(A)}{\Pr(B)}
\end{emphalign*}
\item \emph{Example:} Interpreting the results of screening tests. The test is not perfect --- false positives and false negatives randomly occur. 
\item A positive test is therefore only a presumption of sickness, not an absolute certainty.
\item Let $\Pr(\text{sick})$ be the unconditional probability of being sick. Bayes' rule gives the probability of being sick conditional on testing positive, $\Pr(\text{sick}|\text{positive})$:
\begin{align*}
\Pr(\text{sick}|\text{positive}) 
  & = \frac{\Pr(\text{positive}|\text{sick}) \cdot \Pr(\text{sick})} {\Pr(\text{positive})}
\end{align*}
\item $\Pr(\text{sick}|\text{positive})$ is larger than $\Pr(\text{sick})$
\item $\Pr(\text{sick}|\text{positive})$ and $\Pr(\text{positive}|\text{sick})$ are commonly confused, but they can be very different. 
\end{itemize}
\end{frame}
%%%%%%%%%%%%%%%%%%%%%%%%%%%%%%%%%%%%%%%%%%%%%%%%%%%%%%%%%%  

