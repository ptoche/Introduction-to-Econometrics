

%%%%%%%%%%%%%%%%%%%%%%%%%%%%%%%%%%%%%%%%%%%%%%%%%%%%%%%%%%
\begin{frame}
\frametitle{Conditional Variance}
\begin{itemize}
\item \emph{Conditional Variance:} 
\begin{align*}
\var(Y|X) = \exp\left[(Y-\exp[Y|X])^2 | X\right]
\end{align*}
\item \emph{Law of Total Variance:}
\begin{align*}
\var(Y) = \var\left(\exp[Y | X]\right) +  \exp\left[\var(Y|X)\right]
\end{align*}
where the first term $\var\left(\exp[Y|X]\right)$ captures the explained part of the variance and the second term $\exp\left[\var(Y|X)\right]$ the unexplained part.
\item The Analysis of Variance (ANOVA) is based on the Law of Total Variance. In ANOVA, the observed variance in a particular variable is partitioned into components attributable to different sources of variation.
\item The law of total variance is also known as the law of iterated variances.
\end{itemize}
\end{frame}
%%%%%%%%%%%%%%%%%%%%%%%%%%%%%%%%%%%%%%%%%%%%%%%%%%%%%%%%%%  

