

%%%%%%%%%%%%%%%%%%%%%%%%%%%%%%%%%%%%%%%%%%%%%%%%%%%%%%%%%%
\begin{frame}
\frametitle{Data Types}
\begin{itemize}
\item \emph{Experimental Data:} Obtained from experiments designed to assess
the causal effect of a treatment on an outcome.
\begin{itemize}
\item Randomized controlled trials: Ideal experimental data for program evaluation. 
\item Example: Tennessee STAR project (Student-Teacher Achievement Ratio). A four-year longitudinal study: Over 7,000 students in 79 schools randomly assigned into small/medium/large classes.
\end{itemize}
\item \emph{Observational Data:} A type of data where researchers have no control on how the treatment is allocated.
\begin{itemize}
\item Obtained from surveys or administrative records, e.g. National Education Longitudinal Study. 
\end{itemize}
\item \emph{Cross-Sectional Data:} A type of data collected by observing many subjects (such as individuals, firms, countries, or regions) at one point in time.
\begin{itemize}
\item Example: California Test Score data: each district is one observation.
\end{itemize}
\item \emph{Time-Series Data:} A type of data collected at successive points in time. 
\begin{itemize}
\item Example: US inflation and unemployment rate data. Successive points in time are usually equally spaced (daily, monthly, quarterly, annual), but may not be.
\end{itemize}
\end{itemize}
\end{frame}
%%%%%%%%%%%%%%%%%%%%%%%%%%%%%%%%%%%%%%%%%%%%%%%%%%%%%%%%%%


%%%%%%%%%%%%%%%%%%%%%%%%%%%%%%%%%%%%%%%%%%%%%%%%%%%%%%%%%%
\begin{frame}
\frametitle{Data Types}
\begin{itemize}
\item \emph{Longitudinal Data:} A research design that involves repeated observations of the same variables over periods of time.
\begin{itemize}
\item Example: Panel Study of Income Dynamics (PSID). It may or may not be randomized. Often used to monitor individuals across several periods of their lives.
\end{itemize}
\item \emph{Cohort Studies:} One type of longitudinal study which sample a cohort --- a group of people who share a defining characteristic, who experienced a common event, such as birth or graduation.
\begin{itemize}
\item Example: Millenium Cohort Study (evaluate the long-term health effects of military service).
\end{itemize}
\item \emph{Panel Data:} A subset of longitudinal data where observations are for the same subjects each time. A balanced panel is a dataset in which each panel member is observed every year.
\begin{itemize}
\item Example: National Longitudinal Survey of Youth (NLSY).
\end{itemize}
\end{itemize}
\end{frame}
%%%%%%%%%%%%%%%%%%%%%%%%%%%%%%%%%%%%%%%%%%%%%%%%%%%%%%%%%%

