

%%%%%%%%%%%%%%%%%%%%%%%%%%%%%%%%%%%%%%%%%%%%%%%%%%%%%%%%%%
\begin{frame}
\frametitle{Basic Definitions}
\begin{itemize}
\item \emph{Random Experiment}: 
A repeatable procedure that has a well-defined set of outcomes.
\item \emph{Outcomes}: The mutually exclusive potential results of a random process.
\item \emph{Sample Space}: The set $\mathcal{S}$ of the possible outcomes of an experiment.
\item \emph{Event}: A subset of the sample space, $\mathcal{E} \subseteq \mathcal{S}$.
\item \emph{Random variable}: function from set $\mathcal{S}$ to a real number.
\item \emph{Probability}: A mapping from all subsets of the sample space $S$ to $[0,1]$ with these properties:%
\begin{itemize}
\item \label{P1} $\Pr(\mathcal{S}) = 1$
\item \label{P2} $0 \leq \Pr(\mathcal{E}) \leq 1$ for all $\mathcal{E} \subseteq \mathcal{S}$
\item \label{P3} If $\mathcal{E}_1, \mathcal{E}_2,\ldots$ are disjoint events, then 
$\Pr(\mathcal{E}_1 \cup \mathcal{E}_2 \cup ...) = \Pr (\mathcal{E}_1) + \Pr (\mathcal{E}_2) + \ldots$
\end{itemize}
\item The probability of an outcome is the long-run frequency that the outcome occurs. 
\end{itemize}
\end{frame}
%%%%%%%%%%%%%%%%%%%%%%%%%%%%%%%%%%%%%%%%%%%%%%%%%%%%%%%%%%  


%%%%%%%%%%%%%%%%%%%%%%%%%%%%%%%%%%%%%%%%%%%%%%%%%%%%%%%%%%
\begin{frame}
\frametitle{Examples}
\begin{itemize}
\item Example of a random experiment: Flipping a coin.
\item Sample space: $\mathcal{S}=\{H,T\}$. 
\item Equivalent representation: 
\begin{align*}
X = 
\begin{cases} 
1 & \text{if} \quad H\\
0 & \text{if} \quad T
\end{cases}
\end{align*}
\item The assignment is arbitrary, so another equivalent representation:
\begin{align*}
X = 
\begin{cases} 
1 & \text{if} \quad T\\
0 & \text{if} \quad H
\end{cases}
\end{align*}
\item Other examples of random experiments: rolling a die and observing the face; rolling two dice and observing the sums; drawing colored balls from an urn; car registration numbers from passing cars. 
\end{itemize}
\end{frame}
%%%%%%%%%%%%%%%%%%%%%%%%%%%%%%%%%%%%%%%%%%%%%%%%%%%%%%%%%%


%%%%%%%%%%%%%%%%%%%%%%%%%%%%%%%%%%%%%%%%%%%%%%%%%%%%%%%%%%
\begin{frame}
\frametitle{Set Notation}
\begin{itemize}
\item \emph{Set:} A collection of objects. These objects are called \emph{elements} of the set. 
\item A sample space is a set whose elements are the possible outcomes of an experiment. 
\item The \emph{union} of set $A$ and set $B$, written $A \cup B$, is the set of every element in either $A$ or $B$. 
\item The \emph{intersection} of set $A$ and $B$, written $A \cap B$, is the set of every element that belongs to both $A$ and $B$.  
\item The set with no elements, denoted $\emptyset$, is called the \emph{empty set}.
\item Two sets are \emph{disjoint} if their intersection is empty.
\item $A$ is a \emph{subset} of $B$, denoted $A \subseteq B$, if every element of $A$ is also an element of $B$. 
\item If $A$ is a \emph{strict subset} of $B$, it is denoted $A \subset B$.
\end{itemize}
\end{frame}
%%%%%%%%%%%%%%%%%%%%%%%%%%%%%%%%%%%%%%%%%%%%%%%%%%%%%%%%%%  

