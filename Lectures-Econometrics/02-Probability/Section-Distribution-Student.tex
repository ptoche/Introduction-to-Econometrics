

%%%%%%%%%%%%%%%%%%%%%%%%%%%%%%%%%%%%%%%%%%%%%%%%%%%%%%%%%%
\begin{frame}
\frametitle{Student-t Distribution}
\begin{itemize}
\item \emph{Student-t Distribution:} The distribution of the ratio of a standard normal random variable to the square root of an independently distributed chi-squared random variable!
\item Named after Statistician William Sealy Gosset, who used the pseudonym ``Student.''
\item The Student-t distribution is the theoretical distribution of a standardized normal distribution when the standard deviation used in the standardization procedure is estimated from the sample data. The distribution is parameterized by the ``degrees of freedom''. 
\item Let $Z$ be a standard normal random variable, let $W$ be a chi-squared random variable with $m$ degrees of freedom, with $Z$ and $W$ independently distributed, then
\begin{align*}
Z \sim N(0, 1), 
\quad
W \sim \chi^2 (m), 
\quad
Z \indep W
\implies
\frac{Z}{\sqrt{W/m}} \sim t(m)
\end{align*}
\item The Student-t distribution has a bell shape similar to that of the normal distribution, but with ``fatter'' tails. As the degrees of freedom are increased, the Student-t distribution tends to the standard normal distribution. For values of $m$ larger than $20$ there is little practical difference between the two distributions. 
\end{itemize}
\end{frame}
%%%%%%%%%%%%%%%%%%%%%%%%%%%%%%%%%%%%%%%%%%%%%%%%%%%%%%%%%%  

% f <- function(x) pt(2, df=x)/pnorm(2)
% f(1:20) 
% 0.8722604 0.9293921 0.9519950 0.9638699 0.9711234 0.9759908 0.9794735 0.9820844 0.9841123 0.9857315 0.9870536 0.9881531 0.9890816 0.9898758 0.9905629 0.9911631 0.9916918 0.9921611 0.9925803 0.9929571