

%%%%%%%%%%%%%%%%%%%%%%%%%%%%%%%%%%%%%%%%%%%%%%%%%%%%%%%%%%
\begin{frame}
\frametitle{Measures of Central Tendency}
\begin{itemize}
\item \emph{Population mean:} \quad $\mu_{Y}$
\item A population mean may or may not exist. 
\item \emph{Sample mean:} \quad $\mean{Y}$
\item The sample mean is constructed from observed realizations of the random variable $Y$, so for a sample of size $n$,
\begin{align*}
\mean{Y} = \frac{y_1 + y_2 + \ldots + y_n}{n}
\end{align*}
where $y_i$ denotes the measured sample values. The bar is a short-hand for ``sample mean.''
\item The central tendency is also referred to as ``location.''
\item The expected value of $Y$ is also called the mean of $Y$.
\begin{align*}
\exp[Y] = \sum_{i=1}^n y_i \Pr(y_i) = \frac{1}{n}\sum_{i=1}^n y_i
\end{align*}
\item Other measures of central tendency: \emph{Median}, \emph{Mode}, \emph{Truncated Mean}, \emph{Geometric Mean}, \emph{Harmonic Mean}. 
The geometric mean is particularly useful for time series, e.g. to compute the compounded annual growth rate.
\end{itemize}
\end{frame}
%%%%%%%%%%%%%%%%%%%%%%%%%%%%%%%%%%%%%%%%%%%%%%%%%%%%%%%%%%

