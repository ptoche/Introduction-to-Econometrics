

%%%%%%%%%%%%%%%%%%%%%%%%%%%%%%%%%%%%%%%%%%%%%%%%%%%%%%%%%%
\begin{frame}
\frametitle{Normal Distribution}
\begin{itemize}
\item \emph{Normal Distribution:} A continuous distribution with a symmetric bell-shaped probability density function. The normal density with mean $\mu$ and variance $\sigma^2$ is symmetric around $\mu$ and has $95\%$ of its probability between $\mu-1.96\sigma$ and $\mu+1.96\sigma$. A normally distributed random variable is uniquely defined by its mean and variance and is denoted
\begin{align*}
Y \sim N(\mu, \sigma^2)
\end{align*}
The two parameters $\mu$ and $\sigma^2$ are sufficient to completely describe any normal distribution.
\item \emph{Standard Normal Distribution:} The special case with mean zero and unit variance, $N(0,1)$. The standard normal cumulative distribution is often denoted $\Phi$, that is for a fixed value $z$,
\begin{align*}
\Pr(Z \leq z) = \Phi(z)
\end{align*}
\item All normal distributions can be converted to the standard normal by standardization,
\begin{align*}
Y \sim N(\mu,\sigma^2)
\implies 
Z = \frac{Y-\mu}{\sigma} \sim N(0,1)
\end{align*}
\end{itemize}
\end{frame}
%%%%%%%%%%%%%%%%%%%%%%%%%%%%%%%%%%%%%%%%%%%%%%%%%%%%%%%%%%  


%%%%%%%%%%%%%%%%%%%%%%%%%%%%%%%%%%%%%%%%%%%%%%%%%%%%%%%%%
\begin{frame}
\frametitle{Normal Distribution: Thin Tails}
\begin{figure}
\centering
\includegraphics[width=\linewidth,height=0.8\textheight,keepaspectratio]%
{StockWatson4e-02-fig-05-Zoom}
\caption{The Normal distribution has ``thin tails'': very little probability lies in the tails, and so few outliers are expected to occur.}
\end{figure}
\end{frame}
%%%%%%%%%%%%%%%%%%%%%%%%%%%%%%%%%%%%%%%%%%%%%%%%%%%%%%%%%
