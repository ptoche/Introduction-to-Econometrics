

\section{Summary}


%%%%%%%%%%%%%%%%%%%%%%%%%%%%%%%%%%%%%%%%%%%%%%%%%%%%%%%%%
\begin{frame}
\frametitle{Summary}
\begin{itemize}
\item Regression models used for forecasting need not have a causal interpretation.
\item A time series variable is typically serially correlated.
\item The accuracy of a forecast is measured by its mean squared forecast error.
\item An autoregression of order $p$ is a linear multiple regression model in which the regressors are the first $p$ lags of the dependent variable. The coefficients of an AR($p$) can be estimated by OLS, and the estimated regression function can be used for forecasting. The lag order $p$ can be estimated using an information criterion such as the BIC or the AIC.
\item Adding other variables and their lags to an autoregression can improve forecasting performance.
\item Forecast intervals quantify forecast uncertainty. If the errors are normally distributed, an approximate $68\%$ forecast interval can be constructed as the forecast plus or minus an estimate of the root mean squared forecast error.
\item A series that contains a stochastic trend is non-stationary. A random walk stochastic trend can be detected using the ADF statistic and can be eliminated by using the first difference of the series.
\item If the population regression function changes over time, then OLS estimates produce unreliable forecasts. The QLR statistic can be used to test for a break.
\item Pseudo out-of-sample forecasts can be used to estimate the root mean squared forecast error, to compare different forecasting models, and to assess model stability toward the end of the sample.
\end{itemize}
\end{frame}
%%%%%%%%%%%%%%%%%%%%%%%%%%%%%%%%%%%%%%%%%%%%%%%%%%%%%%%%%


