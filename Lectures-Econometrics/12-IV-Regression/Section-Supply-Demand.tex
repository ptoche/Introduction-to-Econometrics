

%%%%%%%%%%%%%%%%%%%%%%%%%%%%%%%%%%%%%%%%%%%%%%%%%%%%%%%%%
\begin{frame}
\frametitle{Identification in a Supply-Demand Diagram}
\emph{Philip Wright's Problem} 
\begin{itemize}
\item Philip Wright was concerned with an important economic problem of his day: how to set an import tariff --- a tax on imported goods. The key to understanding the economic effect of a tariff was having quantitative estimates of the demand and supply curves of the goods. 
\item Consider the problem of estimating the elasticity of demand for butter from the demand equation
\begin{align*}
\ln(Q_{i}^{\text{butter}}) = \beta_{0} + \beta_{1} \ln(P_{i}^{\text{butter}}) + u_{i}
\end{align*}
where $Q_{i}^{\text{butter}}$ is the $i$th observation on the quantity of butter consumed, $P_{i}^{\text{butter}}$ is its price, and $u_{i}$ represents other factors that affect demand, such as income and consumer tastes.
\item The coefficient $\beta_{1}$ has the interpretation of the elasticity of $Y$ with respect to $X$.
\item Because of the interactions between supply and demand, the regressor $\ln(P_{i}^{\text{butter}})$ is likely correlated with the error term $u_{i}$. 
\end{itemize}
\end{frame}
%%%%%%%%%%%%%%%%%%%%%%%%%%%%%%%%%%%%%%%%%%%%%%%%%%%%%%%%%


%%%%%%%%%%%%%%%%%%%%%%%%%%%%%%%%%%%%%%%%%%%%%%%%%%%%%%%%%
\begin{frame}
\frametitle{Identification in a Supply-Demand Diagram}
\emph{$\ln(P_{i}^{\text{butter}})$ is correlated with $u_{i}$.} 
\begin{itemize}
\item In year $1$, the demand and supply curves for the first period are denoted $D_{1}$ and $S_{1}$.
\item The first period's equilibrium price and quantity are determined by their intersection. 
\item In year $2$, demand increases from $D_{1}$ to $D_{2}$ and supply decreases from $S_{1}$ to $S_{2}$.
\item The second period's equilibrium price and quantity are determined by the new intersection. 
\item In year $3$, demand increases again to $D_{3}$, supply increases to $S_{3}$.
\item A new equilibrium price and quantity are determined.
\item Because the points have been determined by changes in both demand and supply, they cannot be used to identify the demand or supply curve.
\item Consider a scatterplot of the equilibrium price/quantity points. Fitting a line to these points will estimate neither a demand curve nor a supply curve! 
\item To identify the demand curve, we would need to fix the supply curve. And vice versa: To identify the supply curve, we would need to fix the demand curve. 
\end{itemize}
\end{frame}
%%%%%%%%%%%%%%%%%%%%%%%%%%%%%%%%%%%%%%%%%%%%%%%%%%%%%%%%%


%%%%%%%%%%%%%%%%%%%%%%%%%%%%%%%%%%%%%%%%%%%%%%%%%%%%%%%%%
\begin{frame}
\frametitle{Identification in a Supply-Demand Diagram}
\begin{figure}
\centering
\includegraphics[width=\linewidth,height=0.7\textheight,keepaspectratio]%
{StockWatson4e-12-fig-01a}
\caption{Price and quantity are determined by the intersection of the supply and demand curves. The equilibrium in the first period is determined by the intersection of the demand curve $D_{1}$ and the supply curve $S_{1}$. Equilibrium in the second period is the intersection of $D_{2}$ and $S_{2}$, and equilibrium in the third period is the intersection of $D_{3}$ and $S_{3}$.}
\end{figure}
\end{frame}
%%%%%%%%%%%%%%%%%%%%%%%%%%%%%%%%%%%%%%%%%%%%%%%%%%%%%%%%%


%%%%%%%%%%%%%%%%%%%%%%%%%%%%%%%%%%%%%%%%%%%%%%%%%%%%%%%%%
\begin{frame}
\frametitle{Identification in a Supply-Demand Diagram}
\begin{figure}
\centering
\includegraphics[width=\linewidth,height=0.7\textheight,keepaspectratio]%
{StockWatson4e-12-fig-01b}
\caption{This scatterplot shows equilibrium price and quantity in $11$ different time periods. The demand and supply curves are hidden. Can you determine the demand and supply curves from the points on the scatterplot?}
\end{figure}
\end{frame}
%%%%%%%%%%%%%%%%%%%%%%%%%%%%%%%%%%%%%%%%%%%%%%%%%%%%%%%%%


%%%%%%%%%%%%%%%%%%%%%%%%%%%%%%%%%%%%%%%%%%%%%%%%%%%%%%%%%
\begin{frame}
\frametitle{Identification in a Supply-Demand Diagram}
\begin{figure}
\centering
\includegraphics[width=\linewidth,height=0.7\textheight,keepaspectratio]%
{StockWatson4e-12-fig-01c}
\caption{When the supply curve shifts from $S_{1}$ to $S_{2}$ to $S_{3}$ but the demand curve remains at $D_{1}$, the equilibrium prices and quantities trace out the demand curve.}
\end{figure}
\end{frame}
%%%%%%%%%%%%%%%%%%%%%%%%%%%%%%%%%%%%%%%%%%%%%%%%%%%%%%%%%


%%%%%%%%%%%%%%%%%%%%%%%%%%%%%%%%%%%%%%%%%%%%%%%%%%%%%%%%%
\begin{frame}
\frametitle{Identification in a Supply-Demand Diagram}
\emph{Philip Wright's Problem} 
\begin{itemize}
\item Wright discovered that we need some third variable --- the instrumental variable --- that shifts supply but does not shift demand.
\item The instrument must be correlated with the price. 
\begin{itemize}
\item The instrument shifts the supply curve, which leads to a change in the price. 
\end{itemize}
\item The instrument must be uncorrelated with the error term. 
\begin{itemize}
\item The instrument does not shift the demand curve.
\end{itemize}
\item Wright considered weather as an instrument. 
\begin{itemize}
\item below-average rainfall in a dairy region could impair grazing and thus reduce butter production at a given price --- it would shift the supply curve downwards (i.e. to the left) and increase the equilibrium price.
\end{itemize}
\item \emph{Instrument relevance:}
\newlinequad
dairy-region rainfall has a direct influence on the supply of butter and therefore on price. 
\item \emph{Instrument exogeneity:} 
\newlinequad
dairy-region rainfall has no direct influence on the demand for butter. 
\end{itemize}
\end{frame}
%%%%%%%%%%%%%%%%%%%%%%%%%%%%%%%%%%%%%%%%%%%%%%%%%%%%%%%%%
