

%%%%%%%%%%%%%%%%%%%%%%%%%%%%%%%%%%%%%%%%%%%%%%%%%%%%%%%%%
\begin{frame}
\frametitle{Sampling Distribution of the TSLS Estimator}
\emph{Large Sample Distribution} 
\begin{itemize}
\item For a single regressor $X$ and a single instrument $Z$, the TSLS estimator has a simple formula.
\item The TSLS estimator of $\beta_{1}$ is the ratio of the sample covariance between $Z$ and $Y$ to the sample covariance between $Z$ and $X$:
\begin{align*}
\hat{\beta}_{1}^{\text{TSLS}} 
  = \frac{s_{ZY}}{s_{ZX}}
\end{align*}
\item In large samples, $\hat{\beta}_{1}^{\text{TSLS}}$ is consistent and normally distributed. 
\begin{align*}
\hat{\beta}_{1}^{\text{TSLS}}
    & \xrightarrow{p} \beta_{1}\\
\hat{\beta}_{1}^{\text{TSLS}} 
    & \sim \N(\beta_{1}, \sigma^{2}_{\hat{\beta}_{1}^{\text{TSLS}}})\\
\sigma^{2}_{\hat{\beta}_{1}^{\text{TSLS}}} 
    & = \frac{1}{n} \frac{\var[(Z_{i}-\mu_{Z})u_{i}]}{[\cov(Z_{i}, X_{i})]^{2}}
\end{align*}
\item Because $\hat{\beta}_{1}^{\text{TSLS}}$ is normally distributed in large samples, hypothesis
tests about $\beta_{1}$ can be performed by computing the $t$-statistic.
\end{itemize}
\end{frame}
%%%%%%%%%%%%%%%%%%%%%%%%%%%%%%%%%%%%%%%%%%%%%%%%%%%%%%%%%
