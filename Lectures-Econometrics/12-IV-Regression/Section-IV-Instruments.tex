

%%%%%%%%%%%%%%%%%%%%%%%%%%%%%%%%%%%%%%%%%%%%%%%%%%%%%%%%%
\begin{frame}
\frametitle{Dealing with Weak Instruments}
\emph{Checking for Weak Instruments}
\begin{itemize}
\item \emph{Weak instruments Rule of Thumb:} for a single endogenous regressor, $F_{\text{first-stage}}<10$.
\item The first-stage $F$-statistic tests the hypothesis that the coefficients on the instruments are all equal to zero in the first stage of two stage least squares. 
\item If the instruments are weak, the TSLS estimator is biased even in large samples. 
\newlinequad
$t$-statistics and confidence intervals are unreliable.
\item \emph{What to do?}
\begin{itemize}
\item If you have many instruments, discard the weakest instruments.
\newlinequad
With weak instruments, the standard errors are invalid, so if these standard errors become smaller after dropping instruments, that is meaningless.
\item $J$-test for overidentifying restrictions
\item find stronger instruments!
\item Go beyond TSLS.
\end{itemize}
\end{itemize}
\end{frame}
%%%%%%%%%%%%%%%%%%%%%%%%%%%%%%%%%%%%%%%%%%%%%%%%%%%%%%%%%


%%%%%%%%%%%%%%%%%%%%%%%%%%%%%%%%%%%%%%%%%%%%%%%%%%%%%%%%%
\begin{frame}
\frametitle{Dealing with Weak Instruments}
\emph{Overidentifying Restrictions Test: $J$-Statistics}
\begin{itemize}
\item Use OLS to estimate a regression of the residuals from TSLS estimation ($\hat{u}^{\text{TSLS}}_{i}$) on the instruments ($Z$) and exogenous regressors ($W$):
\begin{align*}
\hat{u}^{\text{TSLS}}_{i} 
= \delta_{0} + \delta_{1}Z_{1i} + \ldots + \delta_{m}Z_{mi} + \delta_{m+1}W_{1i} + \ldots + \delta_{m+r}W_{ri} + e_{i}
\end{align*}
where $e_{i}$ is the regression error term. 
\item Use the estimated regression coefficients $\hat{\delta}_{0},\ldots,\hat{\delta}_{m}$ to test the hypothesis:
\begin{align*}
\delta_{1} = \ldots = \delta_{m} = 0
\end{align*}
\item Let $F$ denote the homoskedasticity-only $F$-statistic. The overidentifying restrictions test statistic is $J=mF$. 
\item Under the null hypothesis that all the instruments are exogenous, if $e_{i}$ is homoskedastic, in large samples $J$ is distributed $\chi^{2}_{m-k}$, where $m-k$ is the degree of overidentification.
\item The degree of overidentification, $m-k$, is the number of instruments ($Z$) minus the number of endogenous regressors ($W$).
\end{itemize}
\end{frame}
%%%%%%%%%%%%%%%%%%%%%%%%%%%%%%%%%%%%%%%%%%%%%%%%%%%%%%%%%