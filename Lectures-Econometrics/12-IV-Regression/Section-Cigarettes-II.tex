

%%%%%%%%%%%%%%%%%%%%%%%%%%%%%%%%%%%%%%%%%%%%%%%%%%%%%%%%%
\begin{frame}
\frametitle{The Demand for Cigarettes}
\begin{itemize}
\item The IV regression suffers from omitted variable bias: Include income as a control variable.
\item One instrument, $\vn{SalesTax}$:
\begin{align*}
\verywidehat{\ln(Q^{\text{cigarettes}})} 
  = \muse{9.43}{1.26} 
   -\muse{1.14}{0.37}\,\ln(P^{\text{cigarettes}})
   +\muse{0.21}{0.31}\,\ln(\vn{Inc})
\end{align*}
\item Two instruments, $\vn{SalesTax}$ and $\vn{CigTax}$:
\begin{align*}
\verywidehat{\ln(Q^{\text{cigarettes}})} 
  = \muse{9.89}{0.96} 
   -\muse{1.28}{0.25}\,\ln(P^{\text{cigarettes}})
   +\muse{0.28}{0.25}\,\ln(\vn{Inc})
\end{align*}
\item The standard errors on the estimated demand elasticity is smaller with two instruments:
\newlinequad
The regression explains more of the variation in cigarette prices.
\end{itemize}
\end{frame}
%%%%%%%%%%%%%%%%%%%%%%%%%%%%%%%%%%%%%%%%%%%%%%%%%%%%%%%%%


%%%%%%%%%%%%%%%%%%%%%%%%%%%%%%%%%%%%%%%%%%%%%%%%%%%%%%%%%
\begin{frame}
\frametitle{The Demand for Cigarettes: Long-Run Elasticity}
\begin{itemize}
\item To estimate the long-run price elasticity, consider quantity and price changes that occur over $10$-year periods and two instruments, $\vn{SalesTax}$ and $\vn{CigTax}$. Three models are estimated and compared. The first model uses a single instrument, $\vn{SalesTax}$. 
\item The first stage regression is:
\begin{align*}
\verywidehat{\ln(P^{\text{cigarettes}}_{1995})-\ln(P^{\text{cigarettes}}_{1985})} 
  & = \muse{0.53}{0.03} 
   -\muse{0.22}{0.22}\,\left[\ln(\vn{Inc}_{1995})-\ln(\vn{Inc}_{1985})\right]\\[1ex]
  & \quad +\muse{0.0255}{0.0044}\,\left[\ln(\vn{SalesTax}_{1995})-\ln(\vn{SalesTax}_{1985})\right]
\end{align*}
\item The first-stage $F$-statistic for the null hypothesis
\begin{align*}
& H_{0}\colon \vn{SalesTax}_{1995} - \vn{SalesTax}_{1985} = 0 \\
& F = t^{2} = (0.0255/0.0044)^2 = 33.7.
\end{align*}
\item Since $F>10$, the instrument $\vn{SalesTax}$ is not weak.
\end{itemize}
\end{frame}
%%%%%%%%%%%%%%%%%%%%%%%%%%%%%%%%%%%%%%%%%%%%%%%%%%%%%%%%%


%%%%%%%%%%%%%%%%%%%%%%%%%%%%%%%%%%%%%%%%%%%%%%%%%%%%%%%%%
\begin{frame}
\frametitle{The Demand for Cigarettes: Long-Run Elasticity}
\begin{figure}
\centering
\includegraphics[width=\linewidth,height=0.85\textheight,keepaspectratio]%
{StockWatson4e-12-tbl-01-Zoom}
\caption{Two-Stage Least Squares Estimates of the Demand for Cigarettes Using Panel Data for $48$ U.S. States.}
\end{figure}
\end{frame}
%%%%%%%%%%%%%%%%%%%%%%%%%%%%%%%%%%%%%%%%%%%%%%%%%%%%%%%%%


%%%%%%%%%%%%%%%%%%%%%%%%%%%%%%%%%%%%%%%%%%%%%%%%%%%%%%%%%
\begin{frame}
\frametitle{The Demand for Cigarettes: Long-Run Elasticity}
\begin{itemize}
\item \emph{The instruments are not weak:}
\begin{itemize}
\item The first-stage $F$-statistics are $33.7$, $107.2$, and $88.6$.
\item In all three cases the first-stage $F$-statistics exceed $10$.
\end{itemize}
\item The regressions in columns (1) and (2) are exactly identified --- We cannot use the $J$-test. 
\newlinequad
(each have a single instrument and a single endogenous regressor)
\item The regression in column (3) is overidentified 
\newlinequad
(there is one overidentifying restriction, $m-k=2-1=1$). 
\item \emph{The null hypothesis that both the instruments are exogenous is rejected:}
\begin{itemize}
\item The $J$-statistic for column (3) is $4.93$. 
\item The critical value from the $\chi^{2}_{1}$ distribution with significance level $5\%$ is $3.84$.
\end{itemize}
\end{itemize}
\end{frame}
%%%%%%%%%%%%%%%%%%%%%%%%%%%%%%%%%%%%%%%%%%%%%%%%%%%%%%%%%
