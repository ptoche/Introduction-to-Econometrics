

%%%%%%%%%%%%%%%%%%%%%%%%%%%%%%%%%%%%%%%%%%%%%%%%%%%%%%%%%
\begin{frame}
\frametitle{Problems and Applications}
\exercise{Stock \& Watson, Introduction (4th), Chapter~9, Exercise~1.}
Suppose you have just read a careful statistical study of the effect of advertising on the demand for cigarettes. Using data from New York during the 1970s, the study concluded that advertising on buses and subways was more effective than print advertising. Use the concept of external validity to determine if these results are likely to apply to Boston in the 1970s, Los Angeles in the 1970s, and New York in 2018.
\exercise{Stock \& Watson, Introduction (4th), Chapter~9, Exercise~3.}
Labor economists studying the determinants of women's earnings discovered a puzzling empirical result. Using randomly selected employed women, they regressed earnings on the women's number of children and a set of control variables (age, education, occupation, and so forth). They found that women with more children had higher wages, controlling for these other factors. Explain how sample selection might be the cause of this result. 
% Hint: Notice that women who do not work outside the home are missing from the sample.
\end{frame}
%%%%%%%%%%%%%%%%%%%%%%%%%%%%%%%%%%%%%%%%%%%%%%%%%%%%%%%%%

