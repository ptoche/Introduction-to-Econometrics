

%%%%%%%%%%%%%%%%%%%%%%%%%%%%%%%%%%%%%%%%%%%%%%%%%%%%%%%%%
\begin{frame}
\frametitle{Summary}
\begin{enumerate}
\item A study is internally valid if the statistical inferences about causal effects are valid for the population being studied. 
\item A study is externally valid if its inferences and conclusions can be generalized from the population and setting studied to other populations and settings.
\item There are two types of threats to internal validity: 1. OLS estimators are biased and inconsistent if the regressors and error terms are correlated. 2. Confidence intervals and hypothesis tests are not valid when the standard errors are incorrect.
\item Regressors and error terms may be correlated when there are omitted variables, an incorrect functional form is used, one or more of the regressors are measured with error, the sample is chosen non-randomly from the population, or there is simultaneous causality between the regressors and dependent variables.
\item Standard errors are incorrect when the error term is correlated across different observations. Stardard errors could be incorrect if the errors are heteroskedastic and the computer software does not use robust estimates.
\item When regression models are used solely for prediction, it is not necessary for the regression coefficients to be unbiased estimates of causal effects, but the regression model be externally valid.
\end{enumerate}
\end{frame}
%%%%%%%%%%%%%%%%%%%%%%%%%%%%%%%%%%%%%%%%%%%%%%%%%%%%%%%%%
