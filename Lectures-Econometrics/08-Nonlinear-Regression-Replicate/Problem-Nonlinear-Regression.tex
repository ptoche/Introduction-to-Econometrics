% Options for packages loaded elsewhere
\PassOptionsToPackage{unicode}{hyperref}
\PassOptionsToPackage{hyphens}{url}
%
\documentclass[
  10pt,
  ignorenonframetext,
  t, svgnames, handout]{beamer}
\usepackage{pgfpages}
\setbeamertemplate{caption}[numbered]
\setbeamertemplate{caption label separator}{: }
\setbeamercolor{caption name}{fg=normal text.fg}
\beamertemplatenavigationsymbolsempty
% Prevent slide breaks in the middle of a paragraph
\widowpenalties 1 10000
\raggedbottom
\setbeamertemplate{part page}{
  \centering
  \begin{beamercolorbox}[sep=16pt,center]{part title}
    \usebeamerfont{part title}\insertpart\par
  \end{beamercolorbox}
}
\setbeamertemplate{section page}{
  \centering
  \begin{beamercolorbox}[sep=12pt,center]{part title}
    \usebeamerfont{section title}\insertsection\par
  \end{beamercolorbox}
}
\setbeamertemplate{subsection page}{
  \centering
  \begin{beamercolorbox}[sep=8pt,center]{part title}
    \usebeamerfont{subsection title}\insertsubsection\par
  \end{beamercolorbox}
}
\AtBeginPart{
  \frame{\partpage}
}
\AtBeginSection{
  \ifbibliography
  \else
    \frame{\sectionpage}
  \fi
}
\AtBeginSubsection{
  \frame{\subsectionpage}
}
\usepackage{amsmath,amssymb}
\usepackage{lmodern}
\usepackage{iftex}
\ifPDFTeX
  \usepackage[T1]{fontenc}
  \usepackage[utf8]{inputenc}
  \usepackage{textcomp} % provide euro and other symbols
\else % if luatex or xetex
  \usepackage{unicode-math}
  \defaultfontfeatures{Scale=MatchLowercase}
  \defaultfontfeatures[\rmfamily]{Ligatures=TeX,Scale=1}
\fi
% Use upquote if available, for straight quotes in verbatim environments
\IfFileExists{upquote.sty}{\usepackage{upquote}}{}
\IfFileExists{microtype.sty}{% use microtype if available
  \usepackage[]{microtype}
  \UseMicrotypeSet[protrusion]{basicmath} % disable protrusion for tt fonts
}{}
\makeatletter
\@ifundefined{KOMAClassName}{% if non-KOMA class
  \IfFileExists{parskip.sty}{%
    \usepackage{parskip}
  }{% else
    \setlength{\parindent}{0pt}
    \setlength{\parskip}{6pt plus 2pt minus 1pt}}
}{% if KOMA class
  \KOMAoptions{parskip=half}}
\makeatother
\usepackage{xcolor}
\IfFileExists{xurl.sty}{\usepackage{xurl}}{} % add URL line breaks if available
\IfFileExists{bookmark.sty}{\usepackage{bookmark}}{\usepackage{hyperref}}
\hypersetup{
  pdftitle={Stock and Watson Chapter 8: Replication},
  pdfauthor={Patrick Toche, ptoche@fullerton.edu},
  hidelinks,
  pdfcreator={LaTeX via pandoc}}
\urlstyle{same} % disable monospaced font for URLs
\newif\ifbibliography
\usepackage{color}
\usepackage{fancyvrb}
\newcommand{\VerbBar}{|}
\newcommand{\VERB}{\Verb[commandchars=\\\{\}]}
\DefineVerbatimEnvironment{Highlighting}{Verbatim}{commandchars=\\\{\}}
% Add ',fontsize=\small' for more characters per line
\usepackage{framed}
\definecolor{shadecolor}{RGB}{248,248,248}
\newenvironment{Shaded}{\begin{snugshade}}{\end{snugshade}}
\newcommand{\AlertTok}[1]{\textcolor[rgb]{0.94,0.16,0.16}{#1}}
\newcommand{\AnnotationTok}[1]{\textcolor[rgb]{0.56,0.35,0.01}{\textbf{\textit{#1}}}}
\newcommand{\AttributeTok}[1]{\textcolor[rgb]{0.77,0.63,0.00}{#1}}
\newcommand{\BaseNTok}[1]{\textcolor[rgb]{0.00,0.00,0.81}{#1}}
\newcommand{\BuiltInTok}[1]{#1}
\newcommand{\CharTok}[1]{\textcolor[rgb]{0.31,0.60,0.02}{#1}}
\newcommand{\CommentTok}[1]{\textcolor[rgb]{0.56,0.35,0.01}{\textit{#1}}}
\newcommand{\CommentVarTok}[1]{\textcolor[rgb]{0.56,0.35,0.01}{\textbf{\textit{#1}}}}
\newcommand{\ConstantTok}[1]{\textcolor[rgb]{0.00,0.00,0.00}{#1}}
\newcommand{\ControlFlowTok}[1]{\textcolor[rgb]{0.13,0.29,0.53}{\textbf{#1}}}
\newcommand{\DataTypeTok}[1]{\textcolor[rgb]{0.13,0.29,0.53}{#1}}
\newcommand{\DecValTok}[1]{\textcolor[rgb]{0.00,0.00,0.81}{#1}}
\newcommand{\DocumentationTok}[1]{\textcolor[rgb]{0.56,0.35,0.01}{\textbf{\textit{#1}}}}
\newcommand{\ErrorTok}[1]{\textcolor[rgb]{0.64,0.00,0.00}{\textbf{#1}}}
\newcommand{\ExtensionTok}[1]{#1}
\newcommand{\FloatTok}[1]{\textcolor[rgb]{0.00,0.00,0.81}{#1}}
\newcommand{\FunctionTok}[1]{\textcolor[rgb]{0.00,0.00,0.00}{#1}}
\newcommand{\ImportTok}[1]{#1}
\newcommand{\InformationTok}[1]{\textcolor[rgb]{0.56,0.35,0.01}{\textbf{\textit{#1}}}}
\newcommand{\KeywordTok}[1]{\textcolor[rgb]{0.13,0.29,0.53}{\textbf{#1}}}
\newcommand{\NormalTok}[1]{#1}
\newcommand{\OperatorTok}[1]{\textcolor[rgb]{0.81,0.36,0.00}{\textbf{#1}}}
\newcommand{\OtherTok}[1]{\textcolor[rgb]{0.56,0.35,0.01}{#1}}
\newcommand{\PreprocessorTok}[1]{\textcolor[rgb]{0.56,0.35,0.01}{\textit{#1}}}
\newcommand{\RegionMarkerTok}[1]{#1}
\newcommand{\SpecialCharTok}[1]{\textcolor[rgb]{0.00,0.00,0.00}{#1}}
\newcommand{\SpecialStringTok}[1]{\textcolor[rgb]{0.31,0.60,0.02}{#1}}
\newcommand{\StringTok}[1]{\textcolor[rgb]{0.31,0.60,0.02}{#1}}
\newcommand{\VariableTok}[1]{\textcolor[rgb]{0.00,0.00,0.00}{#1}}
\newcommand{\VerbatimStringTok}[1]{\textcolor[rgb]{0.31,0.60,0.02}{#1}}
\newcommand{\WarningTok}[1]{\textcolor[rgb]{0.56,0.35,0.01}{\textbf{\textit{#1}}}}
\setlength{\emergencystretch}{3em} % prevent overfull lines
\providecommand{\tightlist}{%
  \setlength{\itemsep}{0pt}\setlength{\parskip}{0pt}}
\setcounter{secnumdepth}{-\maxdimen} % remove section numbering
\usepackage{lmodern}% bug: converts \pounds to dollar sign
\usepackage[english]{babel}
\usepackage[T1]{fontenc}% font encoding, load before inputenc
\usepackage{fontspec}% font encoding
\frenchspacing
\defaultfontfeatures{
   Mapping=tex-text,
   Scale=MatchLowercase,
}
%   \setsansfont{Montserrat}
\setsansfont{Lato}
\setmonofont{Droid Sans Mono}
\newfontfamily\fontbold{Lato Bold}
\newfontfamily\fontitalic{Lato Italic}
\newfontfamily\fontbolditalic{Lato Bold Italic}
\newfontfamily\quotefont{Minotaur}

\usepackage{amsmath}
\usepackage{graphicx}
\usepackage{caption}
\usepackage{hyperref}

\colorlet{themetext}{black}
\colorlet{themefill}{RoyalBlue!50}
\colorlet{themecolor}{NavyBlue}
\usetheme{default}
\setbeamertemplate{navigation symbols}{}
\setbeamercovered{transparent}
\setbeamertemplate{title page}[empty]
\usecolortheme{seahorse}
\setbeamercovered{transparent=4}
\setbeamercolor{palette primary}{use=structure,fg=black,bg=themefill}
\setbeamercolor{title}{fg=black}
\setbeamercolor{frametitle}{fg=black}
\setbeamercolor{itemize item}{fg=themecolor}
\setbeamercolor{enumerate item}{fg=themecolor}
\setbeamercolor{itemize subitem}{fg=themecolor}
\setbeamercolor{enumerate subitem}{fg=themecolor}
\setbeamertemplate{itemize item}[triangle] 
\setbeamertemplate{itemize subitem}{\raisebox{-0.7ex}{\scalebox{1}[2.5]{\bfseries\textendash}}} 
\setbeamertemplate{enumerate subitem}{\alph{enumii}.}
\ifLuaTeX
  \usepackage{selnolig}  % disable illegal ligatures
\fi

\title{Stock and Watson Chapter 8: Replication}
\subtitle{Econ 440 - Introduction to Econometrics}
\author{Patrick Toche,
\href{mailto:ptoche@fullerton.edu}{\nolinkurl{ptoche@fullerton.edu}}}
\date{17 April 2022}

\begin{document}
\frame{\titlepage}

\begin{frame}{Replication}
\protect\hypertarget{replication}{}
Replicate regression results from James H. Stock and Mark W. Watson,
Introduction to econometrics, Pearson, 4th Edition, Chapter 8. The data
used is available in the Stata format \(caschool.dta\) and in the Excel
format \(caschool.xlsx\).
\end{frame}

\begin{frame}[fragile]{Load dataset}
\protect\hypertarget{load-dataset}{}
Let's load the Stata dataset using the \texttt{haven} library.

\scriptsize

\begin{Shaded}
\begin{Highlighting}[]
\FunctionTok{library}\NormalTok{(haven)}
\NormalTok{df }\OtherTok{\textless{}{-}} \FunctionTok{read\_dta}\NormalTok{(}\StringTok{"caschool.dta"}\NormalTok{)}
\FunctionTok{head}\NormalTok{(df)}
\end{Highlighting}
\end{Shaded}

\begin{verbatim}
## # A tibble: 6 x 18
##   observat dist_cod county  district gr_span enrl_tot teachers calw_pct meal_pct
##      <dbl>    <dbl> <chr>   <chr>    <chr>      <dbl>    <dbl>    <dbl>    <dbl>
## 1        1    75119 Alameda Sunol G~ KK-08        195    10.9     0.510     2.04
## 2        2    61499 Butte   Manzani~ KK-08        240    11.1    15.4      47.9 
## 3        3    61549 Butte   Thermal~ KK-08       1550    82.9    55.0      76.3 
## 4        4    61457 Butte   Golden ~ KK-08        243    14      36.5      77.0 
## 5        5    61523 Butte   Palermo~ KK-08       1335    71.5    33.1      78.4 
## 6        6    62042 Fresno  Burrel ~ KK-08        137     6.40   12.3      87.0 
## # ... with 9 more variables: computer <dbl>, testscr <dbl>, comp_stu <dbl>,
## #   expn_stu <dbl>, str <dbl>, avginc <dbl>, el_pct <dbl>, read_scr <dbl>,
## #   math_scr <dbl>
\end{verbatim}

\normalsize You can also use the Excel file of course

\scriptsize

\begin{Shaded}
\begin{Highlighting}[]
\FunctionTok{library}\NormalTok{(readxl)}
\NormalTok{df }\OtherTok{\textless{}{-}} \FunctionTok{read\_xlsx}\NormalTok{(}\StringTok{"caschool.xlsx"}\NormalTok{, }\AttributeTok{trim\_ws=}\ConstantTok{TRUE}\NormalTok{)}
\end{Highlighting}
\end{Shaded}

\normalsize
\end{frame}

\begin{frame}[fragile]{Data cleaning}
\protect\hypertarget{data-cleaning}{}
The variable names are inconsistent with the textbook and other versions
of the dataset used in other exercises, so let's rename them:

\scriptsize

\begin{Shaded}
\begin{Highlighting}[]
\FunctionTok{names}\NormalTok{(df)[}\FunctionTok{names}\NormalTok{(df) }\SpecialCharTok{==} \StringTok{"avginc"}\NormalTok{] }\OtherTok{\textless{}{-}} \StringTok{"income"}
\FunctionTok{names}\NormalTok{(df)[}\FunctionTok{names}\NormalTok{(df) }\SpecialCharTok{==} \StringTok{"testscr"}\NormalTok{] }\OtherTok{\textless{}{-}} \StringTok{"testscore"}
\end{Highlighting}
\end{Shaded}

\normalsize
\end{frame}

\begin{frame}[fragile]{Linear Regression of Test Score on District
Income}
\protect\hypertarget{linear-regression-of-test-score-on-district-income}{}
\begin{block}{The simple linear regression.}
\protect\hypertarget{the-simple-linear-regression.}{}
\scriptsize

\begin{Shaded}
\begin{Highlighting}[]
\NormalTok{m1 }\OtherTok{\textless{}{-}} \FunctionTok{lm}\NormalTok{(testscore }\SpecialCharTok{\textasciitilde{}}\NormalTok{ income, }\AttributeTok{data=}\NormalTok{df)}
\FunctionTok{summary}\NormalTok{(m1)}
\end{Highlighting}
\end{Shaded}

\begin{verbatim}
## 
## Call:
## lm(formula = testscore ~ income, data = df)
## 
## Residuals:
##    Min     1Q Median     3Q    Max 
## -39.57  -8.80   0.60   9.03  32.53 
## 
## Coefficients:
##             Estimate Std. Error t value Pr(>|t|)    
## (Intercept) 625.3836     1.5324   408.1   <2e-16 ***
## income        1.8785     0.0905    20.8   <2e-16 ***
## ---
## Signif. codes:  0 '***' 0.001 '**' 0.01 '*' 0.05 '.' 0.1 ' ' 1
## 
## Residual standard error: 13.4 on 418 degrees of freedom
## Multiple R-squared:  0.508,  Adjusted R-squared:  0.506 
## F-statistic:  431 on 1 and 418 DF,  p-value: <2e-16
\end{verbatim}

\normalsize
\end{block}
\end{frame}

\begin{frame}[fragile]{Tidy up with broom}
\protect\hypertarget{tidy-up-with-broom}{}
Once we have estimated a model, it is convenient to use the
\texttt{broom} library to tidy things up:

\scriptsize

\begin{Shaded}
\begin{Highlighting}[]
\FunctionTok{library}\NormalTok{(broom)}
\FunctionTok{tidy}\NormalTok{(m1)}
\end{Highlighting}
\end{Shaded}

\begin{verbatim}
## # A tibble: 2 x 5
##   term        estimate std.error statistic  p.value
##   <chr>          <dbl>     <dbl>     <dbl>    <dbl>
## 1 (Intercept)   625.      1.53       408.  0       
## 2 income          1.88    0.0905      20.8 2.75e-66
\end{verbatim}

\normalsize
\end{frame}

\begin{frame}[fragile]{Nonlinear Regression of Test Score on District
Income}
\protect\hypertarget{nonlinear-regression-of-test-score-on-district-income}{}
\begin{block}{The quadratic regression.}
\protect\hypertarget{the-quadratic-regression.}{}
To run a nonlinear regression on \(var\), we can use the \(I()\)
wrapper:

\scriptsize

\begin{Shaded}
\begin{Highlighting}[]
\FunctionTok{lm}\NormalTok{(testscore }\SpecialCharTok{\textasciitilde{}}\NormalTok{ income }\SpecialCharTok{+} \FunctionTok{I}\NormalTok{(income}\SpecialCharTok{\^{}}\DecValTok{2}\NormalTok{), }\AttributeTok{data=}\NormalTok{df)}
\end{Highlighting}
\end{Shaded}

\normalsize or use the more versatile \(poly(var)\):

\scriptsize

\begin{Shaded}
\begin{Highlighting}[]
\NormalTok{m2 }\OtherTok{\textless{}{-}} \FunctionTok{lm}\NormalTok{(testscore }\SpecialCharTok{\textasciitilde{}} \FunctionTok{poly}\NormalTok{(income,}\DecValTok{2}\NormalTok{,}\AttributeTok{raw=}\ConstantTok{TRUE}\NormalTok{), }\AttributeTok{data=}\NormalTok{df)}
\FunctionTok{tidy}\NormalTok{(m2)}
\end{Highlighting}
\end{Shaded}

\begin{verbatim}
## # A tibble: 3 x 5
##   term                         estimate std.error statistic  p.value
##   <chr>                           <dbl>     <dbl>     <dbl>    <dbl>
## 1 (Intercept)                  607.       3.05       199.   0       
## 2 poly(income, 2, raw = TRUE)1   3.85     0.304       12.7  2.69e-31
## 3 poly(income, 2, raw = TRUE)2  -0.0423   0.00626     -6.76 4.71e-11
\end{verbatim}

\normalsize but note that we set the \texttt{raw=TRUE} option: ``if
true, use raw and not orthogonal polynomials.'\,'
\end{block}
\end{frame}

\begin{frame}[fragile]{Nonlinear Regression of Test Score on District
Income}
\protect\hypertarget{nonlinear-regression-of-test-score-on-district-income-1}{}
\begin{block}{The cubic regression.}
\protect\hypertarget{the-cubic-regression.}{}
\scriptsize

\begin{Shaded}
\begin{Highlighting}[]
\NormalTok{m3 }\OtherTok{\textless{}{-}} \FunctionTok{lm}\NormalTok{(testscore }\SpecialCharTok{\textasciitilde{}} \FunctionTok{poly}\NormalTok{(income,}\DecValTok{3}\NormalTok{,}\AttributeTok{raw=}\ConstantTok{TRUE}\NormalTok{), }\AttributeTok{data=}\NormalTok{df)}
\FunctionTok{tidy}\NormalTok{(m3)}
\end{Highlighting}
\end{Shaded}

\begin{verbatim}
## # A tibble: 4 x 5
##   term                           estimate std.error statistic   p.value
##   <chr>                             <dbl>     <dbl>     <dbl>     <dbl>
## 1 (Intercept)                  600.        5.83        103.   4.61e-298
## 2 poly(income, 3, raw = TRUE)1   5.02      0.859         5.84 1.06e-  8
## 3 poly(income, 3, raw = TRUE)2  -0.0958    0.0374       -2.56 1.07e-  2
## 4 poly(income, 3, raw = TRUE)3   0.000685  0.000472      1.45 1.47e-  1
\end{verbatim}

\normalsize
\end{block}
\end{frame}

\begin{frame}[fragile]{Linear-Log Regression}
\protect\hypertarget{linear-log-regression}{}
\begin{block}{The linear-log regression.}
\protect\hypertarget{the-linear-log-regression.}{}
\scriptsize

\begin{Shaded}
\begin{Highlighting}[]
\NormalTok{m4 }\OtherTok{\textless{}{-}} \FunctionTok{lm}\NormalTok{(testscore }\SpecialCharTok{\textasciitilde{}} \FunctionTok{log}\NormalTok{(income), }\AttributeTok{data=}\NormalTok{df)}
\FunctionTok{tidy}\NormalTok{(m4)}
\end{Highlighting}
\end{Shaded}

\begin{verbatim}
## # A tibble: 2 x 5
##   term        estimate std.error statistic  p.value
##   <chr>          <dbl>     <dbl>     <dbl>    <dbl>
## 1 (Intercept)    558.       4.20     133.  0       
## 2 log(income)     36.4      1.57      23.2 4.77e-77
\end{verbatim}

\normalsize
\end{block}
\end{frame}

\begin{frame}[fragile]{Log-Linear Regression}
\protect\hypertarget{log-linear-regression}{}
\begin{block}{The linear-log regression.}
\protect\hypertarget{the-linear-log-regression.-1}{}
\scriptsize

\begin{Shaded}
\begin{Highlighting}[]
\NormalTok{m5 }\OtherTok{\textless{}{-}} \FunctionTok{lm}\NormalTok{(}\FunctionTok{log}\NormalTok{(testscore) }\SpecialCharTok{\textasciitilde{}}\NormalTok{ income, }\AttributeTok{data=}\NormalTok{df)}
\FunctionTok{tidy}\NormalTok{(m5)}
\end{Highlighting}
\end{Shaded}

\begin{verbatim}
## # A tibble: 2 x 5
##   term        estimate std.error statistic  p.value
##   <chr>          <dbl>     <dbl>     <dbl>    <dbl>
## 1 (Intercept)  6.44     0.00236     2724.  0       
## 2 income       0.00284  0.000140      20.4 1.41e-64
\end{verbatim}

\normalsize
\end{block}
\end{frame}

\begin{frame}[fragile]{Log-Log Regression}
\protect\hypertarget{log-log-regression}{}
\begin{block}{The log-log regression.}
\protect\hypertarget{the-log-log-regression.}{}
\scriptsize

\begin{Shaded}
\begin{Highlighting}[]
\NormalTok{m6 }\OtherTok{\textless{}{-}} \FunctionTok{lm}\NormalTok{(}\FunctionTok{log}\NormalTok{(testscore) }\SpecialCharTok{\textasciitilde{}} \FunctionTok{log}\NormalTok{(income), }\AttributeTok{data=}\NormalTok{df)}
\FunctionTok{tidy}\NormalTok{(m6)}
\end{Highlighting}
\end{Shaded}

\begin{verbatim}
## # A tibble: 2 x 5
##   term        estimate std.error statistic  p.value
##   <chr>          <dbl>     <dbl>     <dbl>    <dbl>
## 1 (Intercept)   6.34     0.00645     982.  0       
## 2 log(income)   0.0554   0.00241      23.0 4.52e-76
\end{verbatim}

\normalsize
\end{block}
\end{frame}

\end{document}
