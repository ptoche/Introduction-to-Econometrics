

%%%%%%%%%%%%%%%%%%%%%%%%%%%%%%%%%%%%%%%%%%%%%%%%%%%%%%%%%
\begin{frame}
\frametitle{Model Specification}
\emph{Choose a regression specification}
\begin{itemize}
\item \emph{Starting point:} 
\newlinequad
Think through the possible sources of omitted variable bias. 
\item \emph{Base specification:} 
\newlinequad
Include the variables of primary interest and the control variables suggested by economic theory.
\item \emph{Alternative specifications:} 
\newlinequad
If the estimates of the coefficients of interest are numerically similar across the alternative specifications, then this provides evidence that the estimates of the base specification are reliable. 
\end{itemize}
\end{frame}
%%%%%%%%%%%%%%%%%%%%%%%%%%%%%%%%%%%%%%%%%%%%%%%%%%%%%%%%%


%%%%%%%%%%%%%%%%%%%%%%%%%%%%%%%%%%%%%%%%%%%%%%%%%%%%%%%%%
\begin{frame}
\frametitle{Interpreting R-Squared}
\emph{Interpreting $R^2$ and Adjusted-$R^2$}
\begin{itemize}
\item An increase in the $R^2$ or Adjusted-$R^2$ does not necessarily mean that an added variable is statistically significant. Perform a hypothesis test using the t-statistic.
\item A high $R^2$ or Adjusted-$R^2$ does not mean that the regressors are a true cause of the dependent variable.
\item A high $R^2$ or Adjusted-$R^2$ does not mean that there is no omitted variable bias.
\item A high $R^2$ or Adjusted-$R^2$ does not mean that you have the most appropriate set of regressors
--- A low $R^2$ or Adjusted-$R^2$ does not mean that you have an inappropriate set of regressors. 
\end{itemize}
\end{frame}
%%%%%%%%%%%%%%%%%%%%%%%%%%%%%%%%%%%%%%%%%%%%%%%%%%%%%%%%%
