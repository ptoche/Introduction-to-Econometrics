

%%%%%%%%%%%%%%%%%%%%%%%%%%%%%%%%%%%%%%%%%%%%%%%%%%%%%%%%%
\begin{frame}
\frametitle{Summary}
\begin{itemize}
\item Hypotheses involving more than one restriction on the coefficients are called joint hypotheses. Joint hypotheses can be tested using an F-statistic.
\item Regression specification proceeds by first determining a base specification chosen to address concern about omitted variable bias. The base specification can be modified by including additional regressors that control for other potential sources of omitted variable bias. 
\item Choosing the specification with the highest $R^2$ can lead to regression models that do not estimate the causal effect of interest.
\item A study is internally valid if the statistical inferences about causal effects are valid for the population being studied. A study is externally valid if its inferences and conclusions can be generalized from the population and setting studied to other populations and settings.
\item Regressors and error terms may be correlated when there are omitted variables, an incorrect functional form is used, one or more of the regressors are measured with error, the sample is not randomly selected, or there is simultaneous causality between the regressors and dependent variables.
\item When regression models are used for prediction, it is not necessary for regression coefficients to be unbiased estimates of causal effects, but the model must be externally valid.
\end{itemize}
\end{frame}
%%%%%%%%%%%%%%%%%%%%%%%%%%%%%%%%%%%%%%%%%%%%%%%%%%%%%%%%%

